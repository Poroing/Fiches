\documentclass{article}

\usepackage[utf8]{inputenc}
\usepackage[francais]{babel}
\usepackage{amssymb}
\usepackage{amsmath}
\usepackage{amsthm}
\usepackage{color}
\usepackage{centernot}

\everymath{\displaystyle}

\newcommand{\ssi}{si et seulement si}
\newcommand{\R}{\mathbb{R}}
\newcommand{\C}{\mathbb{C}}
\newcommand{\N}{\mathbb{N}}
\newcommand{\Z}{\mathbb{Z}}
\newcommand{\resu}{(u_n)_{n \in \N} \in \R^\N}
\newcommand{\allent}{\forall n \in \N}
\newcommand{\et}{\;\;\text{ et }\;\;}
\newcommand{\ou}{\;\;\text{ ou }\;\;}
\newcommand{\tq}{\;\;\text{ tel que }\;\;}
\newcommand{\lm}{\lim\limits}
\newcommand{\voi}[1]{\text{ au voisinage de }#1}
\newcommand{\bs}[1]{\boldsymbol{#1}}
\newcommand{\equ}[1][]{\underset{#1}{\sim}}
\newcommand{\ex}{\mathrm{e}}
\newcommand{\co}[1]{\mathrm{C^{#1}}}
\newcommand{\lminf}{\lm_{n \to \infty}}
\newcommand{\notimplies}{\centernot\implies}
\newcommand{\en}{\text{ en }}
\newcommand{\est}{\text{ est }}
\newcommand{\drt}{\text{ à droite de }}
\newcommand{\gch}{\text{ à gauche de }}
\newcommand{\strmo}{\text{ strictement monotone }}
\newcommand{\mo}{\text{ monotone }}
\newcommand{\sur}{\text{ sur }}
\newcommand{\drv}{\text{ dérivable }}
\newcommand{\e}{\!\!}
\newcommand{\exs}{\text{ existe }}
\newcommand{\fn}{\text{ finie }}
\newcommand{\ngl}[2]{\negl_{#1}\!\!\!(#2)}
\newcommand{\lips}{\text{ lipschitzienne }}

\DeclareMathOperator{\ch}{ch}
\DeclareMathOperator{\tah}{th}
\DeclareMathOperator{\sh}{sh}
\DeclareMathOperator{\dl}{DL}
\DeclareMathOperator*{\negl}{o}

\newenvironment{att}
{\bgroup \color{red}{\Large\textbf{Attention}}\\}
{\egroup}

\theoremstyle{definition}
\newtheorem*{prop}{Proposition}
\newtheorem*{defin}{Définition}
\theoremstyle{remark}
\newtheorem*{rema}{Remarque}
\theoremstyle{plain}
\newtheorem*{them}{Théorème}
\newtheorem*{coro}{Corollaire}

\newenvironment{prp}[1][]
{\begin{prop}[#1]\quad\\}
{\end{prop}}
\newenvironment{dfn}[1][]
{\begin{defin}[#1]\quad\\}
{\end{defin}}
\newenvironment{rem}[1][]
{\begin{rema}[#1]\quad\\}
{\end{rema}}
\newenvironment{thm}[1][]
{\begin{them}[#1]\quad\\}
{\end{them}}
\newenvironment{cor}[1][]
{\begin{coro}[#1]\quad\\}
{\end{coro}}


\title{Derivation des fonctions à valeurs réelles}

\begin{document}
\maketitle
\pagebreak

\section{Definition et premières propriétés}
\quad
\begin{dfn}
\begin{align*}
        &f \drv \e \en a \\
\iff    &\lm_{x \to a} \frac{f(x) - f(a)}{x - a} = l \exs \e \et \e \fn
\end{align*}
$f'(a) = l$
\end{dfn}

\begin{dfn}
\begin{align*}
        &f \drv \e \drt a \\
\iff    &\lm_{x \to a^+} \frac{f(x) - f(a)}{x - a} = l \exs \e \et \e \fn
\end{align*}
$f'_d(a) = l$
\end{dfn}

\begin{prp}
\begin{align*}
        &f \drv \e \drt a \et \e \gch a \et f'_g(a) = f'_d(a) \\
\iff    &f \drv \e \en a
\end{align*}
$f'(a) = f'_g(a) = f'_d(a)$
\end{prp}

\begin{prp}
\[
    f \drv \iff f \text{ admet un } \dl_1(a) \\
\]
$f(x) = a_0 + a_1(x - a) + \ngl{x \to a}{x - a}$ \\
$
\begin{cases}
    f(a) = a_0 \\
    f'(a) = a_1
\end{cases}
$
\end{prp}

\begin{prp}
$a \in I$ et n'est pas une extrémité de $I$ \\
\[
    f \drv \e \et \text{ présente un extremum locale en } a
    \implies f'(a) = 0
\]
\end{prp}

\pagebreak
\section{Operations sur les fonctions dérivable}

\begin{prp}
$(\lambda, \mu) \in \R^2$ \\
\begin{tabular}{c|c}
Fonction &  Dérivée \\
\hline
$\lambda f + \mu g$ & $\lambda f' + \mu g'$ \\
$fg$ & $f'g + g'f$ \\
$^1/_f$ & $\frac{-f'}{f^2}$ \\
$^f/_g$ & $\frac{f'g - g'f}{g^2}$ \\
$f\circ g$ & $g'(f'\circ g)$ \\
$f^{-1}$ & $\frac{1}{f\circ f^{-1}}$ \\
\end{tabular}
\end{prp}

\begin{dfn}
$f$ est un foncion $n$ fois dérivable et $f^{(n)}$ est sa dérivée n-ème.
si $f^{(n)}$ est $\co{0}$, $f$ est de casse $\co{n}$
\end{dfn}

\begin{prp}
$(\lambda, \mu) \in \R^2$ \\
\begin{tabular}{c|c}
Fonction & Dérivée n-ème \\
\hline
$\lambda f + \mu g$ & $\lambda f^{(n)} + \mu g^{(n)}$ \\
$fg$ & $\sum_{k = 0}^{n} \binom{n}{k} f^{(k)}g^{(n - k)}$ \\
$x \mapsto {^1/_x}$ & $x \mapsto (-1)^n \frac{n!}{x^{n + 1}}$ \\
\end{tabular}
\end{prp}

\begin{prp}
Soit $f, g$ de classe $\co{n}$ et $(\lambda, \mu) \in \R^2$,
les fonction suivante sont de classe $\co{n}$
\begin{itemize}
    \item $fg$
    \item $f \circ g$
    \item $\lambda f + \mu g$
    \item $^1/_f$
    \item $^f/_g$
\end{itemize}
\end{prp}

\section{Théorème fondamentaux}

\begin{thm}[de Rolle]
\begin{align*}
            &f \;\; \co{0} \sur {[a, b]}, \drv \e \sur {]a, b[} \\
            &\et f(a) = f(b) \\
\implies    &\exists c \in {]a, b[} \tq f'(c) = 0 \\
\end{align*}
\end{thm}

\begin{thm}[\'Egalité des accroissements finis]
\begin{align*}
            &f \;\; \co{0} \sur {[a, b]}, \drv \e \sur {]a, b[} \\
\implies    &\exists c \in {]a, b[}
                \tq f'(c) = \frac{f(a) - f(b)}{a - b}
\end{align*}
\end{thm}

\begin{prp}[Inégalité des acroissements finis]
\begin{align*}
            &f \;\; \co{0} \sur {[a, b]}, \drv \e \sur {]a, b[} \\
            &\et \exists (m, M) \in \R^2 \tq \forall x \in {]a, b[}, \;\;
                m \leq f'(x) \leq M \\
\implies    &m(b - a) \leq f(b) - f(a) \leq M(b - a)
\end{align*}
\end{prp}

\begin{prp}
Toute fonction dérivable sur un segment et admetant une dérivée bornée
sur ce segment est une fonction lipschitzienne sur ce segment.
\end{prp}


\section{Extensions aux fonctions à valeurs complexes}

\begin{dfn}
\begin{align*}
        &f : I \to \C \drv \e \en a \\
\iff    &\Im(f), \Re(f) \drv \e \en a
\end{align*}
\[
    f'(a) = (\Re(f))'(a) + i(\Im(f))'(a) 
\]
\end{dfn}

\subsection{Operation restant valable}
\begin{itemize}
    \item Les opérations
    \begin{itemize}
        \item somme
        \item produit
        \item quotient
    \end{itemize}
    \item Notion de classe $\co{k}$
    \begin{itemize}
        \item somme
        \item produit
        \item quotient
    \end{itemize}
    \item Formule de Leibniz
    \item Les propriété de la composition
    \item Théorème fondamentale
    \begin{itemize}
        \item Théorème de Rolle non valable
        \item \'Egalité des accroissement finis fausse également
        \item Inagalité des accroissement finis reste valable
        \item Théorème de la limite de la dérivée reste valable
        \item Prolongement de classe $\co{k}$ reste valable
    \end{itemize}
\end{itemize}

\end{document}
