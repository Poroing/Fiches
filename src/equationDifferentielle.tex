\documentclass{article}

\usepackage[utf8]{inputenc}
\usepackage[francais]{babel}
\usepackage{amssymb}
\usepackage{amsmath}
\usepackage{amsthm}
\usepackage{color}

\everymath{\displaystyle}

\newcommand{\ssi}{si et seulement si}
\newcommand{\R}{\mathbb{R}}
\newcommand{\N}{\mathbb{N}}
\newcommand{\Z}{\mathbb{Z}}
\newcommand{\resu}{(u_n)_{n \in \N} \in \R^\N}
\newcommand{\allent}{\forall n \in \N}
\newcommand{\et}{\text{ et }}
\newcommand{\ou}{\text{ ou }}
\newcommand{\tq}{\text{ tel que }}
\newcommand{\lm}{\lim\limits}
\newcommand{\voi}[1]{\text{ au voisinage de }$#1$}
\newcommand{\bs}[1]{\boldsymbol{#1}}

\newenvironment{att}
{\bgroup \color{red}{\Large\textbf{Attention}}\\}
{\egroup}

\theoremstyle{definition}
\newtheorem*{prop}{Proposition}
\newtheorem*{defin}{Définition}
\theoremstyle{remark}
\newtheorem*{rema}{Remarque}
\theoremstyle{plain}
\newtheorem*{them}{Théorème}

\newenvironment{prp}[1][]
{\begin{prop}[#1]\quad\\}
{\end{prop}}
\newenvironment{dfn}[1][]
{\begin{defin}[#1]\quad\\}
{\end{defin}}
\newenvironment{rem}[1][]
{\begin{rema}[#1]\quad\\}
{\end{rema}}
\newenvironment{thm}[1][]
{\begin{them}[#1]\quad\\}
{\end{them}}

\title{Title}

\begin{document}
\maketitle
\pagebreak

\end{document}


\title{\'Equations différentielle}

\begin{document}

\maketitle
\pagebreak
\tableofcontents

Dans tout le chapitre, $\K$ désigne $\R$ ou $\C$ et $I$
désigne un intervalle de $\R$ contenant au moins deux points

\section{\'Equation différentielle linéaire du premier ordre}

On résout l'équation
\begin{equation} \label{eq:E1}
    y' + a(x)y = b(x)
\end{equation}

\subsection{Résolution de l'équation homogène}

\begin{thm}
Soit $I$ un intervalle de $\R$, $a : I \to \K$ un fonction $\co{0}$ sur
$I$. On note $A$ une primitive de $a$ sur $I$. Les solutions de
l'équation
\begin{equation} \label{eq:E10}
    y' + a(x) y = 0
\end{equation}
Sont les fonctions de la forme
\begin{align*}
I &\to \K \\
x &\mapsto \lambda \mathrm{e}^{-A(x)}
\end{align*}
où $\lambda \in \K$
\end{thm}

\subsection{Forme des solutions de l'équation avec second membre}

\begin{prp}
Soit $\varphi_1$ est une solution de \eqref{eq:E1}.

L'ensemble des solutions de \eqref{eq:E1} est l'ensemble des fonctions
$\varphi_1 + \varphi_0$ où $\varphi_0$ parcours l'enemble des solutions
de \eqref{eq:E10}
\end{prp}

\subsection{Méthode de la variation de la constante}

Lorsque \eqref{eq:E1} n'admet pas de solution évidente.
Il faut chercher une solution de la forme
\[
    \varphi_1(x) = \lambda(x)\mathrm{e}^{-A(x)}
\]
où $\lambda$ et une fonction dérivable à determiner.

Pour determiner $\lambda$ on injecte $\varphi_1$ sous la forme
précendente dans l'équation \eqref{eq:E1}

\section{\'Equations linéaire du second ordre à coefficients constant}

On résout
\begin{equation} \label{eq:E}
    a y'' + b y' + y c = f(x)
\end{equation}

où $a$, $b$ et $c$ sont des constantes appartenants à $\K$
et $f$ est une fonction d'une certaine forme, définie sur $\R$
à valeurs dans $\K$.

\subsection{Résolution de l'équation homogène}

On résout
\begin{equation} \label{eq:E0}
    a y'' + b y' + y c = 0
\end{equation}

\begin{prp}
Si $\psi_1$ et $\psi_2$ sont deux solutions de \eqref{eq:E0}.

\[
    \forall (\lambda, \mu) \in \K^2, \lambda \psi_1 + \mu \psi_2
\]
est aussi solution de \eqref{eq:E0}
\end{prp}

\begin{prp}
Soit $r \in \K$

la fonction
\[
    x \mapsto \mathrm{e}^{r x}
\]
 est solution de \eqref{eq:E0} \ssi
\begin{equation} \label{eq:car}
    a r^2 + b^r + c = 0
\end{equation}

L'équation \eqref{eq:car} est appelée l'équation caractéristique
de l'équation différentielle \eqref{eq:E0}
\end{prp}

\subsubsection{Résolution dans le cas complexe}

\begin{thm}
\begin{itemize}
    \item Si l'équation caractéristique admet deux racines complexes
        distinctes, $r_1$ et $r_2$

        L'ensemble des solutions de \eqref{eq:E0} est l'ensemble des
        fonctions
        \begin{align*}
            \R &\to \C \\
            x &\mapsto \lambda \mathrm{e}^{r_1 x} + \mu \mathrm{e}^{r_2 x}
        \end{align*}
        où $(\lambda, \mu) \in \C^2$
    \item Si l'équation caractéristique admet une seul racine complexe
        $r_0$

        L'ensemble des solutions de \eqref{eq:E0} est l'ensemble des
        fonctions
        \begin{align*}
            \R &\to \C \\
            x &\mapsto \lambda \mathrm{e}^{r_0 x} + \mu x \mathrm{e}^{r_0 x}
        \end{align*}
\end{itemize}
\end{thm}

\subsubsection{Résolution dans le cas réel}

\begin{thm}
\begin{itemize}
    \item Si l'équation caractéristique admet deux racines réels
        distinctes, $r_1$ et $r_2$.

        L'ensemble des solutions de \eqref{eq:E0} est l'ensemble des
        fonctions
        \begin{align*}
            \R &\to \R \\
            x &\mapsto \lambda \mathrm{e}^{r_1 x} + \mu \mathrm{e}^{r_2 x}
        \end{align*}
        où $(\lambda, \mu) \in \\R^2$
    \item Si l'équation caractéristique admet une seul racine réelle
        $r_0$

        L'ensemble des solutions de \eqref{eq:E0} est l'ensemble des
        fonctions
        \begin{align*}
            \R &\to \R \\
            x &\mapsto \lambda \mathrm{e}^{r_0 x} + \mu x \mathrm{e}^{r_0 x}
        \end{align*}
        où $(\lambda, \mu) \in \\R^2$
    \item Si l'équation caractéristique admet deux racines complexes,
        $p \pm i \omega$ où $(p, \omega) \in \R^2$

        l'ensemble des solutions de \eqref{eq:E0} est
        \begin{align*}
            \R &\to \R \\
            x &\mapsto \mathrm{e}^{p x} (\lambda \cos (\omega x) \mu \sin (\omega x))
        \end{align*}
\end{itemize}
\end{thm}

\begin{prp}
Soit $\varphi : \R \to \C$ est une solutio de l'équation différentielle
\eqref{eq:E}.

$\Re(\varphi)$ est solution de l'équation
\begin{equation}
    a y'' + b y' + c y = \Re(f(x))
\end{equation}

$Im(\varphi)$ est solution de l'équation
\begin{equation}
    a y'' + b y' + c y = \Im(f(x))
\end{equation}
\end{prp}

\subsection{$f(x)$ est une fonction polynomiale}

\begin{prp}
Si $f$ est une fonction polynomiale de degrée $n$.

L'équation différentielle \eqref{eq:E} admet une solution polynomiale de
de degrée
\begin{itemize}
    \item $n$ si $c \neq 0$
    \item $n + 1$ si $c = 0 \et b \neq 0$
    \item $n + 2$ si $c = 0 \et b = 0$
\end{itemize}
\end{prp}
    

\subsection{$f(x)$ de la forme $P(x) \mathrm{e}^x$}

On cherche une solution particulière de l'équation différentielle
\eqref{eq:E} où $f(x) = P(x)\mathrm{e}^{\alpha x}$ où $P$ est une
fonction polynomiale et $\alpha \in \K$.

Posons
\begin{gather*}
    z(x) = \mathrm{e}^{-\alpha x}y(x) \\
    \iff y(x) = z(x)\mathrm{e}^{\alpha x}
\end{gather*}

Ainsi \eqref{eq:E} équivaut à 
\begin{equation} \label{eq:E'}
    a z'' + (2 \alpha a + b) z' + (a \alpha^2 + b \alpha + c) z = P
\end{equation}

\eqref{eq:E'} admet une solution particulière polynomiale $Q$.
$x \mapsto Q(x)\mathrm{e}^{\alpha x}$ est ainsi une solution particulière de
\eqref{eq:E}.

\section{Problème de Cauchy}

\begin{thm}
Soit $t_0 \in \R$ et $(y_0, y_1) \in \K^2$.

Il existe un unique solution, $y$, de l'équation différentielle
\eqref{eq:E} vérifie
\[
    y(t_0) = y_0 \et y'(t_0) = y_1
\]
\end{thm}

\end{document}
