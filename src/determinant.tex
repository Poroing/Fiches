\documentclass{article}

\usepackage[utf8]{inputenc}
\usepackage[francais]{babel}
\usepackage{amssymb}
\usepackage{amsmath}
\usepackage{amsthm}
\usepackage{color}

\everymath{\displaystyle}

\newcommand{\ssi}{si et seulement si}
\newcommand{\R}{\mathbb{R}}
\newcommand{\N}{\mathbb{N}}
\newcommand{\Z}{\mathbb{Z}}
\newcommand{\resu}{(u_n)_{n \in \N} \in \R^\N}
\newcommand{\allent}{\forall n \in \N}
\newcommand{\et}{\text{ et }}
\newcommand{\ou}{\text{ ou }}
\newcommand{\tq}{\text{ tel que }}
\newcommand{\lm}{\lim\limits}
\newcommand{\voi}[1]{\text{ au voisinage de }$#1$}
\newcommand{\bs}[1]{\boldsymbol{#1}}

\newenvironment{att}
{\bgroup \color{red}{\Large\textbf{Attention}}\\}
{\egroup}

\theoremstyle{definition}
\newtheorem*{prop}{Proposition}
\newtheorem*{defin}{Définition}
\theoremstyle{remark}
\newtheorem*{rema}{Remarque}
\theoremstyle{plain}
\newtheorem*{them}{Théorème}

\newenvironment{prp}[1][]
{\begin{prop}[#1]\quad\\}
{\end{prop}}
\newenvironment{dfn}[1][]
{\begin{defin}[#1]\quad\\}
{\end{defin}}
\newenvironment{rem}[1][]
{\begin{rema}[#1]\quad\\}
{\end{rema}}
\newenvironment{thm}[1][]
{\begin{them}[#1]\quad\\}
{\end{them}}

\title{Title}

\begin{document}
\maketitle
\pagebreak

\end{document}


\title{Groupes symétriques et determinants}

\begin{document}

\section{Groupes symétriques}

Dans tous le paragraphe, $n$ est un entier naturel supérieur ou égal à 1

\subsection{Généralitées}

\begin{dfn}
  Soit $X$ un ensemble.

  Toutes bijection de $X$ est \voc{une permuation de $X$}
\end{dfn}

\begin{dfn}
  Le groupe des permutations de $\lib 1, n \rib$ est appelé
  \voc{le groupe symétrique d'indice $n$}.
  Il est noté $\sym_n$ 
\end{dfn}

\begin{dfn}
  \[
    \forall (\sigma, \sigma') \in \sym_n^2, \sigma \sigma' = \sigma \circ \sigma'
  \]
\end{dfn}

\begin{prp}
  \[
    \Card(\sym_n) = n!
  \]
\end{prp}

\begin{dfn}
  Soit $n \geq 2$ et $\sigma \in \sym_n$.

  On appelle $\sigma$ un cycle si et seulement si il existe
  $p \in \lib 1, n \rib$ tel que
  \begin{itemize}
    \item il existe $(a_1, \ldots, a_p)$ appartenant $\lib 1, n \rib^p$ tel que
      \[
        \forall k \in \lib 1, p - 1 \rib, \sigma(a_k) = a_{k + 1} \et \sigma(a_p) = a_1
      \]
    \item $\forall a \in \lib 1, n \rib - \{ a_1, \ldots, a_n \}, \sigma(a) = a$
  \end{itemize}
  
  On dit que $\sigma$ est un \voc{$p$-cycle} ou \voc{cycle de longueur $p$} et
  l'ensemble $\{a_1, \ldots, a_p \}$ est appelée \voc{support du cycle $\sigma$}

  On note
  \[
    \sigma = (a_1, \ldots, a_p)
  \]
\end{dfn}

\begin{dfn}
  Un cycle de longueur $2$ est appelé \voc{transposition}
\end{dfn}

\begin{prp}
  Soit $(\sigma, \sigma') \in \sym_n$ tel que $\sigma$ et $\sigma'$ soient à supports disjoints.
  
  $\sigma$ et $\sigma'$ commutent.
\end{prp}

\begin{cor}
  Soit $\sigma$ et $\sigma'$ deux cycles à supports disjoints.
  \[
    \forall n \in \N, (\sigma \sigma')^m = \sigma^m \sigma'^m
  \]
\end{cor}

\begin{thm}
  Soit $n \geq 2$.

  Toutes permutations de $\sym_n$ se décompose en un produit unique de cycles à
  supports disjoints.
\end{thm}

\begin{prp}
  Soit $\sigma \in \sym_n$ et si $i \in \lib 1, n \rib$ et $i \in \lib 1, n \rib$
  
  Il existe $p \in \N^*$ tel que $i, \sigma(i), \ldots, \sigma^{p - 1}(i)$ soient deux à deux
  distincts et $\sigma^p(i) = i$
\end{prp}

\subsection{Signature d'une permutation}

\begin{thm}
  Tout élément de $\sym_n$ est un produit de transposition.
\end{thm}

\begin{dfn}
  Il existe une unique application $\epsilon : \sym_n \to \{ 1, -1 \}$
  tel que
  \begin{itemize}
    \item Pour toute transposition $\tau$, $\epsilon(\tau) = -1$
    \item $\forall (\sigma, \sigma') \in \sym_n^2$, $\epsilon(\sigma \sigma') = \epsilon(\sigma)\epsilon(\sigma')$
  \end{itemize}
  Cette application est apelée \voc{signature}.
\end{dfn}

\begin{prp}
  \begin{itemize}
    \item La signature d'un $p$-cycle est $(-1)^{p - 1}$
    \item $\epsilon(\Id) = 1$
    \item $\forall \sigma \in \sym_n, \epsilon(\sigma^{-1}) = \epsilon(\sigma)$
  \end{itemize}
\end{prp}

\begin{dfn}
  Soit $\sigma \in \sym_n$.

  Si la signature de $\sigma$ est $1$, On dit que $\sigma$ est une permutation paire sinon on dit
  qu'elle est impaire.

  l'ensemble des permutation paire est un sous groupe de $(\sym_n, \circ)$
\end{dfn}

\section{Determinant d'une famille de vecteurs et determinant d'un endomorphisme}

\subsection{Forme n-linéaire alternée}

Dans ce paragraphe, $E$ désigne un $\K$-espace vectoriel et $E \neq \left\{ \vec{0} \right\}$

\begin{dfn}
  Soit $p \in \N^*$. On appelle \voc{forme $p$-linéaire} toute application
  \[
    f : E^p \to \K
  \]
  tel que
  \[
    \forall i \in \lib 1, n \rib, \forall (x_1, \ldots, x_{i - 1}, x_{i + 1}, \ldots, x_p) \in E^{p - 1}
  \]
  l'application
  \begin{align*}
    E &\to \K \\
    x &\mapsto f(x_1, \ldots, x_{i - 1}, x, x_{i + 1}, \ldots, x_p)
  \end{align*}

  Les formes $p$-linéaires pour $p \in \N^*$ sont appelées les \voc{formes multiliéaires}
\end{dfn}

\begin{dfn}
  Soit $n \in \N^*$, Soit $f$ une forme $n$-linéaire de $E$.

  Si
  \[
      \forall (x_i)_{i \in \lib 1, n \rib} \in E^n \tq 
    \exists (i, j) \in \lib 1, n \rib^2 \tq i \neq j \et x_i = x_j,
    f(x_1, \ldots, x_n) = 0 
  \]
  
  $f$ est dite \voc{alternée}
\end{dfn}

\end{document}
