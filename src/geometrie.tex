\documentclass{article}

\usepackage[utf8]{inputenc}
\usepackage[francais]{babel}
\usepackage{amssymb}
\usepackage{amsmath}
\usepackage{amsthm}
\usepackage{color}

\everymath{\displaystyle}

\newcommand{\ssi}{si et seulement si}
\newcommand{\R}{\mathbb{R}}
\newcommand{\N}{\mathbb{N}}
\newcommand{\Z}{\mathbb{Z}}
\newcommand{\resu}{(u_n)_{n \in \N} \in \R^\N}
\newcommand{\allent}{\forall n \in \N}
\newcommand{\et}{\text{ et }}
\newcommand{\ou}{\text{ ou }}
\newcommand{\tq}{\text{ tel que }}
\newcommand{\lm}{\lim\limits}
\newcommand{\voi}[1]{\text{ au voisinage de }$#1$}
\newcommand{\bs}[1]{\boldsymbol{#1}}

\newenvironment{att}
{\bgroup \color{red}{\Large\textbf{Attention}}\\}
{\egroup}

\theoremstyle{definition}
\newtheorem*{prop}{Proposition}
\newtheorem*{defin}{Définition}
\theoremstyle{remark}
\newtheorem*{rema}{Remarque}
\theoremstyle{plain}
\newtheorem*{them}{Théorème}

\newenvironment{prp}[1][]
{\begin{prop}[#1]\quad\\}
{\end{prop}}
\newenvironment{dfn}[1][]
{\begin{defin}[#1]\quad\\}
{\end{defin}}
\newenvironment{rem}[1][]
{\begin{rema}[#1]\quad\\}
{\end{rema}}
\newenvironment{thm}[1][]
{\begin{them}[#1]\quad\\}
{\end{them}}

\title{Title}

\begin{document}
\maketitle
\pagebreak

\end{document}


\title{Géometrie}

\begin{document}

\maketitle
\pagebreak
\tableofcontents

Dans tout le chapitre, $E$ désigne un $\R$-espace vectoriel.

\section{Translations}

\subsection{Notations}

\begin{dfn}
Soit $(A, B, C) \in E^2$.

\begin{gather*}
\vect{AB} = B - A \\
\vect{AB} = \vect{0} \iff A = B \\
\vect{AB} = - \vect{AB} \\
\vect{AC} = \vect{AB} + \vect{BC}
\end{gather*}

\subsection{Translations}

\begin{dfn}
Soit $\vect{u} \in E$.

On appelle \voc{translation de vecteur $\vect{u}$ l'application
\begin{align*}
    \tau_{\vect{u}} : E &\to E \\
        M &\mapsto M + \vect{u}
\end{align*}

On a donc $\forall M \in E$
\[
    \vec{M\tau_{\vect{u}}(M)} = \vect{u}
\]
\end{dfn}

\begin{prp}
Soit $(\vect{u}, \vect{v}) \in E^2$.
\begin{gather*}
    \tau_{\vect{u}} \circ \tau_{\vect{v}} = \tau_{\vect{u} + \vect{v}} \\
    \tau_{\vect{0}} = \Id_E \\
\end{gather*}

$\tau_{\vect{u}}$ est bijective.
\[
    \tau_{\vect{u}} = \tau_{-\vect{u}}
\]
\end{prp}

\begin{prp}
L'ensemble des translations de $E$ muni de la loi de composition interne
$\circ$ est un groupe abélien.
\end{prp}

\section{Sous-espaces affines}

\subsection{Définition}

\begin{dfn}
On appelle, $W$, \voc{sous-espace affine} de $E$ l'image d'un sous-espace
vectoriel $F$ par une translation de vecteur $A$.

On note $W = A + F$
\end{dfn}

\begin{prp}
Soit $(A, A') \in E^2$ et $F$ et $F'$ des sous-espace vectorielle de
$E$.

\[
    A + F = A' + F' \iff \{ F = F' \et \vect{AA'} \in F \}
\]
\end{prp}

\begin{prp}
Soit $W$ une sous-espace affine de $E$.

Il existe un unique sous-espace vectoriel $F$ de $E$ tel que
\[
    \exists A \in E \tq W = A + F
\]
\end{prp}

\begin{dfn}
Soit $W$ un sous-espace affine de $E$.

On appelle \voc{direction de $W$} le sous-espace vectoriel $F$ de $E$
vérifiant
\[
    \exists A \in E \tq W = A + F
\]

On dit aussi que \voc{$W$ est dirigée par $F$}.

Si $F$ est de dimension finie, on appelle \voc{dimension de $W$} la
dimension de $F$.

On appelle
\begin{description}
    \item[droite affine] de $E$ tout sous-espace affine ayant pour
        direction une droite vectoriel.
    \item[plan affine] de $E$ tout sous-espace affine ayant pour
        direction un plan vectoriel.
    \item[hyperplan affine] de $E$ tout sous-espace affine ayant pour
        direction un hyperplan vectorielle
\end{description}
\end{dfn}

\subsection{Parallélisme et intersection de sous-espace affine}

\begin{dfn}
Soient $W$ et $W'$ des sous-espaces affines de $E$ de direction
respective $F$ et $F'$.

On dit que \voc{$W$ est parallèle à $W'$} lorsque $F \subset F'$
On note dans ce cas $W // W'$
\end{dfn}

\begin{prp}
Soient $W$ et $W'$ des sous-espaces affines de $E$ tel que
$W // W'$.
\[
    W \inter W' \ou W \subset W'
\]
\end{prp}

\subsection{Repère cartésien}
