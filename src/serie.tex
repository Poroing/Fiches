\documentclass{article}

\usepackage[utf8]{inputenc}
\usepackage[francais]{babel}
\usepackage{amssymb}
\usepackage{amsmath}
\usepackage{amsthm}
\usepackage{color}

\everymath{\displaystyle}

\newcommand{\ssi}{si et seulement si}
\newcommand{\R}{\mathbb{R}}
\newcommand{\N}{\mathbb{N}}
\newcommand{\Z}{\mathbb{Z}}
\newcommand{\resu}{(u_n)_{n \in \N} \in \R^\N}
\newcommand{\allent}{\forall n \in \N}
\newcommand{\et}{\text{ et }}
\newcommand{\ou}{\text{ ou }}
\newcommand{\tq}{\text{ tel que }}
\newcommand{\lm}{\lim\limits}
\newcommand{\voi}[1]{\text{ au voisinage de }$#1$}
\newcommand{\bs}[1]{\boldsymbol{#1}}

\newenvironment{att}
{\bgroup \color{red}{\Large\textbf{Attention}}\\}
{\egroup}

\theoremstyle{definition}
\newtheorem*{prop}{Proposition}
\newtheorem*{defin}{Définition}
\theoremstyle{remark}
\newtheorem*{rema}{Remarque}
\theoremstyle{plain}
\newtheorem*{them}{Théorème}

\newenvironment{prp}[1][]
{\begin{prop}[#1]\quad\\}
{\end{prop}}
\newenvironment{dfn}[1][]
{\begin{defin}[#1]\quad\\}
{\end{defin}}
\newenvironment{rem}[1][]
{\begin{rema}[#1]\quad\\}
{\end{rema}}
\newenvironment{thm}[1][]
{\begin{them}[#1]\quad\\}
{\end{them}}

\title{Title}

\begin{document}
\maketitle
\pagebreak

\end{document}


\title{Séries}

\begin{document}

\maketitle
\pagebreak
\tableofcontents

\section{Généralités}

\subsection{Définition}

\begin{dfn}
Soit $(u_n)_{n \in \N}$ une suite d'éléments de $\K$.

On appelle \voc{série de terme général $u_n$}, et on note $\sum u_n$
la suite $(S_n)_{n \in \N}$ définie par
\[
    S_n = \sum^n_{k = 0} u_k
\]
Pour $n \in \N$, le nombre $S_n$ est appelé \voc{la somme partielle
d'indice $n$ de la série $\sum u_n$}

L'élément $u_n$ est appelé \voc{le terme général de la série $\sum u_n$}
\end{dfn}

\begin{dfn}[Série harmonique]
La série harmonique et la série de terme général $\frac{1}{n}$
\end{dfn}

\begin{dfn}
On dit que $\sum u_n$ \voc{converge} lorsque la suite des sommes partielle
converge.

Dans ce cas $\lminf S_n = \lminf \sum^n_{k = 0} u_k$ est notée
$\sum^{+ \infty}_{n = 0} u_n$ et ce nombre est appelé la somme de la
série $\sum u_n$

Si on pose $\forall n \in \N, \;\; R_n = \sum^{+\infty}_{k = n + 1} u_k$,
$R_n$ est appelé le \voc{$n$-ème reste de la série $\sum u_n$}.

Lorsque $\sum u_n$ ne converge pas, on dit qu'elle diverge.
\end{dfn}

\subsection{Séries convergentes}

\begin{prp}
Soit $\sum u_n$ et $\sum v_n$ des séries convergentes.

\[
    \forall (\lambda, \mu) \in \K^2, \sum (\lambda u_n + \mu v_n)
    \text{ converge}
\]
\end{prp}

\begin{prp}
Soit $\sum u_n$ une série à valeurs complèxes.

$\sum u_n$ converge \ssi $\sum \Re(u_n)$ et $\sum \Im(u_n)$ convergent.

Dans ce cas:
\[
    \sum^{+\infty}_{n = 0} u_n = \sum^{+\infty}_{n = 0} \Re(u_n) 
    + i \sum^{+\infty}_{n = 0} \Im(u_n)
\]
\end{prp}

\begin{prp}
Le terme géneral d'une série convergente converge vers $0$
\end{prp}

\begin{prp}
Soit $(u_n)$ une suite qui ne converge pas vers $0$.

La série $\sum u_n$ diverge.

On dit que \voc{$\sum u_n$ diverge grossiérement}
\end{prp}

\begin{prp}
\[
\sum q^n \begin{cases}
    \text{converge vers } \frac{1}{1 - q}   & si |q| < 1 \\
    \text{diverge}                          & si |q| \geq 1
\end{cases}
\]
\end{prp}

\begin{prp}
Soit $(u_n)$ une suite d'éléments de $\K$.

$(u_n)$ converge \ssi la série $\sum(u_{n + 1} - u_n)$ converge.

Dans ce cas
\[
    \sum^{+\infty}_{n = 0} (u_{n + 1} - u_n) = \lminf u_n - u_0
\]
\end{prp}

\section{Série à termes positifs ou nuls}

\subsection{condition de convergence}

\begin{prp}
Une série à termes positifs converge \ssi la suite des sommes partielle
est majorée
\end{prp}

\begin{prp}
Soient $\sum u_n$ et $\sum v_n$ des séries \voc{à termes positifs} tel que
$\forall n \in \ N, \;\; u_n \leq v_n$.

\begin{itemize}
    \item si $\sum v_n$ converge, alors $\sum u_n$ converge
        et $\sum^{+\infty}_{n = 0} u_n \leq \sum^{+\infty}_{n = 0} v_n$
    \item si $\sum u_n$ diverge, alor $\sum v_n$ diverge
\end{itemize}
\end{prp}

\begin{prp}
Soient $\sum u_n$ et $\sum v_n$ des séries à termes positifs tel que
$u_n = \dom(v_n)$

\begin{itemize}
    \item si $\sum v_n$ converge alors $\sum u_n$ converge
    \item si $\sum u_n$ diverge alors $\sum v_n$ diverge
\end{itemize}
\end{prp}

\begin{prp}
Soient $\sum u_n$ et $\sum v_n$ deux séries à termes strictement positifs
à partir d'un certain rang tel que $u_n \equ v_n$.

Les séries $\sum u_n$ et $\sum v_n$ sont de même nature
\end{prp}

\subsection{Série de Riemann}

\begin{prp}
Soit $n_0 \in \N$, $f : [n_0, +\infty[ \to \R_+$ une fonction $\co{0}$ sur
$[n_0, +\infty[$, $\forall n \geq n_0, I_n = \int^n_{n_0} f(t) \mathrm{d}t$,

\[
    \sum_{n \geq n_0} f(n) \text{ converge} \iff (I_n) \text{ converge}
\]
\end{prp}

\begin{prp}
Soit $\alpha \in \R$. La série $\sum_{n \geq 1} \frac{1}{n^\alpha}$
\begin{itemize}
    \item diverge grossiérement, si $\alpha \in ]-\infty, 0]$
    \item diverge, si $\alpha \in ]0, 1]$
    \item converge, si $\alpha \in ]1, +\infty[$
\end{itemize}
\end{prp}

\section{Séries absolument convergentes}

\begin{dfn}
Soit $(u_n) \in \K^\N$.

On dit que $\sum u_n$ \voc{ converge absolument ou est absolument
    convergente} si $\sum \abs*{u_n}$ converge
\end{dfn}

\begin{prp}
Soit $(u_n) \in \K^\N$ tel que $\sum u_n$ est absolument convergente.
\begin{gather*}
    \sum u_n \text{ converge} \\
    \abs{\sum^{+\infty}_{n = 0} u_n} \leq \sum^{+\infty}_{n = 0}
        \abs*{u_n }
\end{gather*}
\end{prp}

\section{Application ou dévelopement décimal d'un nombre réel}

\begin{dfn}
Un nombre est dit \voc{décimal} lorsqu'il peut s'écrire sous la forme
\[
    \frac{a}{10^n}
\]
où $a \in \Z$ et $n \in \N$

On note $\De$ l'ensemble des nombres décimaux.
\end{dfn}

\begin{thm}
Soit $x \in [0, 1[$ il existe une unique suite $(a_n)_{n \in \N}$
\begin{itemize}
    \item $\forall n \in \N^*, \;\; a_n \in \lib 0, 9 \rib$
    \item $x = \sum^{+\infty}_{n = 1} a_n \times 10^{-n}$
    \item la suite $(a_n)$ n'est pas stationaire égale à $9$
\end{itemize}
L'écriture ''$x = \sum^{+\infty}_{n = 1} a_n \times 10^{-n}$'' est appelée
\voc{développement décimal propre de $x$}
\end{thm}

\end{document}
