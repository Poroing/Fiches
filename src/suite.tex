\documentclass{article}

\usepackage[utf8]{inputenc}
\usepackage[francais]{babel}
\usepackage{amssymb}
\usepackage{amsmath}
\usepackage{amsthm}
\usepackage{color}

\everymath{\displaystyle}

\newcommand{\ssi}{si et seulement si}
\newcommand{\R}{\mathbb{R}}
\newcommand{\N}{\mathbb{N}}
\newcommand{\Z}{\mathbb{Z}}
\newcommand{\resu}{(u_n)_{n \in \N} \in \R^\N}
\newcommand{\allent}{\forall n \in \N}
\newcommand{\et}{\text{ et }}
\newcommand{\ou}{\text{ ou }}
\newcommand{\tq}{\text{ tel que }}
\newcommand{\lm}{\lim\limits}
\newcommand{\voi}[1]{\text{ au voisinage de }$#1$}
\newcommand{\bs}[1]{\boldsymbol{#1}}

\newenvironment{att}
{\bgroup \color{red}{\Large\textbf{Attention}}\\}
{\egroup}

\theoremstyle{definition}
\newtheorem*{prop}{Proposition}
\newtheorem*{defin}{Définition}
\theoremstyle{remark}
\newtheorem*{rema}{Remarque}
\theoremstyle{plain}
\newtheorem*{them}{Théorème}

\newenvironment{prp}[1][]
{\begin{prop}[#1]\quad\\}
{\end{prop}}
\newenvironment{dfn}[1][]
{\begin{defin}[#1]\quad\\}
{\end{defin}}
\newenvironment{rem}[1][]
{\begin{rema}[#1]\quad\\}
{\end{rema}}
\newenvironment{thm}[1][]
{\begin{them}[#1]\quad\\}
{\end{them}}

\title{Title}

\begin{document}
\maketitle
\pagebreak

\end{document}


\title{Suite}

\begin{document}

\maketitle
\pagebreak
\tableofcontents
\pagebreak

\section{Généralité sur les suites réelles}

\begin{dfn}
Une suite est une fonction de $\N$ dans $\R$. cette suite est notée
\begin{align*}
    (u_n)_{n \in \N} : \N &\to \R \\
                        n &\mapsto u_n
\end{align*}
$\R^\N$ est l'ensemble des suites réelles.
\end{dfn}

\begin{dfn}
\begin{align*}
        &(u_n) \text{ est stationnaire } \\
\iff    &\exists n_0 \tq \forall n \geq n_0, \;\; u_n = u_{n_0}
\end{align*}
\end{dfn}

\begin{dfn}
$(u_n)$ est
\begin{description}
    \item[croissante] \quad \\
    \ssi $\forall n \in \N, \;\; u_{n + 1} \geq u_n$
    \item[décroissante] \quad \\
    \ssi $\forall n \in \N, \;\; u_{n + 1} \leq u_n$
    \item[monotone] \quad \\
    \ssi $(u_n)$ est croissante ou décroissante.
\end{description}
\end{dfn}

\begin{rem}[étude de la monotonie d'une suite $(u_n)$]
\begin{itemize}
    \item étudier le signe de $u_{n + 1} - u_n$
    \item Si $\forall n \in \N, \;\; u_n > 0 \ou u_n < 0$,
    on étudie le signe de $\frac{u_{n + 1}}{u_n} - 1$
\end{itemize}
\end{rem}

\begin{dfn}
$(u_n)$ est
\begin{description}
    \item[majorée] \quad \\
    \ssi $\exists M \in \R \tq \allent, \;\; u_n \leq M$
    \item[minorée] \quad \\
    \ssi $\exists m \in \R \tq \allent, \;\; u_n \geq m$
    \item[bornée] \quad \\
    \ssi $\exists A \in \R \tq \allent, \;\; |u_n| \leq A$
\end{description}
\end{dfn}


\section{Limite d'une suite réelles}

\subsection{Définition et premières propriétés}

\begin{dfn}
$(u_n)$ converge vers $l \in \R$ \ssi \\
$\forall \epsilon > 0, \;\; \exists n_0 \in \N \tq \forall n \geq n_0, \;\;
|u_n - l| \leq \epsilon$
\end{dfn}

\begin{prp}[unicitée de la limite]
Si $(u_n)$ \cv vers $l \in \R$, $l$ est unique.
\end{prp}

\begin{prp}
Si $(u_n)$ \cv vers un réelle alors $(u_n)$ est bornée
\end{prp}

\begin{prp}
    Si $(u_n)$ \cv vers $l > 0$ alors $(u_n)$ est minorée à partir d'un
    certain rang par un réelle supérieur à $0$
\end{prp}

\begin{dfn}
    Si $(u_n)$ \cv vers $l \in \R$, $l$ est la limite de la suite et
    est notée $\lminf u_n$
\end{dfn}

\begin{dfn}
$(u_n)$ tend vers
\begin{description}
    \item[$\bs{+\infty}$] \quad \\
    \ssi $\forall A \in \R, \;\; \exists n_0 \in \N \tq \forall n \geq n_0, \;\;
    u_n \geq A$
    \item[$\bs{-\infty}$] \quad \\
    \ssi $\forall A \in \R, \;\; \exists n_0 \in \N \tq \forall n \geq n_0, \;\;
    u_n \leq A$
\end{description}
\end{dfn}

\begin{rem}
Si $(u_n)$ tends vers $+\infty$ alors
\begin{itemize}
    \item $(u_n)$ n'est pas majorée donc non bornée donc divergente
    \item On note $\lminf u_n = +\infty$
    \item $(u_n)$ est minorée
\end{itemize}
\end{rem}

\subsection{Operations sur les limites}

\begin{prp}
Soit $(u_n, v_n)$ une suite convergeant vers $(l_1, l_2) \in \R^2$
\begin{itemize}
    \item $(u_n + v_n)$ \cv vers $l_1 + l_2$
    \item $\forall \lambda \in \R, \;\; (\lambda u_n)$
    \cv vers $\lambda l_1$
    \item $(u_n v_n)$ \cv vers $l_1 \times l_2$
\end{itemize}
\end{prp}

\begin{rem}
Soit $(u_n)$ une suite convergente et $(v_n)$ une suite divergente,
$(u_n + v_n)$ est une suite divergente
\end{rem}

\begin{prp}
Soit $(u_n)$ une suite tendant vers $+\infty$
\begin{itemize}
    \item $(v_n)$ minorée $\implies (u_n + v_n)$ tend vers $+\infty$
    \item $(v_n)$ minorée par un réelle strictement positif
    $\implies (v_n u_n)$ tend vers $+\infty$
\end{itemize}
\end{prp}

\begin{prp}
\begin{itemize}
    \item Si $(u_n)$ \cv vers $l \in \R^*$ alors $u_n \neq 0$ à partir
    d'un certain rang $n_0$ et $\left(\frac{1}{u_n}\right)_{n \geq n_0}$ \cv vers
    $\frac{1}{l}$
    \item Si $(u_n)$ \cv vers $0 \et u_n > 0$ à partir d'un certain rang
    alors $\left(\frac{1}{u_n}\right)$ tend vers $+\infty$
    \item Si $\lminf u_n = +\infty$ alors $u_n \neq 0$ à partir d'un
    certain rang $n_0$ et $\left(\frac{1}{u_n}\right)_{n \geq n_0}$ \cv vers 0
\end{itemize}
\end{prp}

\subsection{Propriétées liées à la relation d'ordre}

\begin{prp}
Soit $(u_n) \in \R^\N$, $(\alpha_n) \in \R^\N$ tel que 
$(\alpha_n)$ converge vers $0$, $l \in \R$ tel qu'à partir
d'un certain rang $|u_n - l| \leq \alpha_n$.

Alors $(u_n)$ converge vers $l$
\end{prp}


\begin{prp}
Soit $(u_n)$ et $(v_n)$ deux suites convergentes
\begin{itemize}
    \item $\exists n_0 \in \N \tq \forall n \geq n_0, \;\; u_n \geq 0
    \implies \lminf u_n \geq 0$
    \item $\exists n_0 \in \N \tq \forall n \geq n_0, \;\; u_n \geq v_n
    \implies \lminf u_n \geq \lminf v_n$
\end{itemize}
\end{prp}

\begin{thm}[d'encadrement]
$\apcr v_n \leq u_n \leq w_n \et (w_n) \text{ \cv vers } l \in \R
\et (v_n) \text{ \cv vers } l \implies u_n \text{ \cv vers } l$
\end{thm}

\begin{prp}
$((u_n), (v_n)) \in (\R^\N)^2 \tq u_n \leq v_n$ à partir d'un certain rang
\begin{itemize}
    \item $\lminf u_n = +\infty \implies \lminf v_n = +\infty$
    \item $\lminf v_n = -\infty \implies \lminf u_n = -\infty$
\end{itemize}
\end{prp}

\begin{cor}
$q \in \R$
\begin{itemize}
    \item $|q| < 1 \implies \lminf q^n = 0$
    \item $q > 1 \implies \lminf q^n = +\infty$
\end{itemize}
\end{cor}

\subsection{Densité}

\begin{prp}
Soit $X$ une partie de $\R$, $X$ est dense dans $\R$ \ssi
$\forall x \in \R, \;\; \exists (x_n) \in X^\N \tq (x_n)$ converge vers
$x$
\end{prp}

\begin{prp}
\begin{align*}
            &X \subset \R \et X \neq \emptyset \et X \text{ majorée}
\implies    &\exists (u_n) \in X^\N \tq (u_n) \text{ converge vers } \sup(X)
\end{align*}
\end{prp}



\begin{dfn}
Soit $(u_n)$ et $(v_n)$ deux suites réelles,
$(v_n)$ est une sous-suite de $(u_n)$ \ssi
$\exists \varphi \in \N^\N \tq \varphi \text{ est strictement croissante}
\et \forall n \in \N , v_n = u_{\varphi(n)}$
\end{dfn}

\begin{prp}
Soit $(u_n)$ une suite admetant pour limite $l \in \bar{\R}$,
Toutes sous-suites de $(u_n)$ ademetent pour limite $l$
\end{prp}

\begin{prp}
Soit $(u_n)$ une suite, si $(u_{2n})$ et $(u_{2n + 1})$ admettent une même
limite $l \in \bar{\R}$ alors $\lminf u_n = l$
\end{prp}

\section{Suites Monotones, Suites adjacentes}

\subsection{Suites monotones}

\begin{thm}
Soit $(u_n)$ une suite réelle croissante
\begin{itemize}
    \item Si $(u_n)$ est majorée, alors $(u_n)$ converge vers 
    $\sup\left\{ u_n \middle| n \in \N \right\}$
    \item Si $(u_n)$ n'est pas majorée alors $(u_n)$ tend vers $+\infty$
\end{itemize}
\end{thm}

\begin{cor}
Soit $(u_n)$ décroissante
\begin{itemize}
    \item si $(u_n)$ est minorée alors $(u_n)$ \cv vers $\inf\{u_n|n \in N\}$
    \item si $(u_n)$ n'est pas minorée alors $\lminf u_n = -\infty$
\end{itemize}
\end{cor}

\subsection{Suites adjacentes}

\begin{dfn}
$(u_n)$ et $(v_n)$ sont des suites adjacantes \ssi
\begin{itemize}
    \item $(u_n)$ est croissante et $(v_n)$ est décroissante
    \item $\lminf (u_n - v_n) = 0$
\end{itemize}
\end{dfn}

\begin{thm}
Deux suite adjacentes $(v_n)$ et $(u_n)$ convergent vers la même limite
$l \in \R$ vérifiant $\forall n \in \N, \;\; u_n \leq l \leq v_n$
\end{thm}

\subsection{Theorème des segments emboités}

\begin{thm}[des segments emboités]
Si $({[a_n, b_n]})_{n \in \N}$ est une suite décroissante de ségments
tel que $\forall n \in \N, \;\; a_n \leq b_n$ dont la longueur des
segments converge vers $0$ alors $\bigcap_{n \in \N} {[a_n, b_n]}$ est
réduit à un point
\end{thm}

\begin{thm}[de \emph{Bolzanot-Weissertrass}]
Toute suite bornée admet au moin une suite extraite convegente.
\end{thm}

\subsection{Suites récurrentes usuelles}

\begin{dfn}[Suite arithmétique]
$(u_n)$ est une suite arithmétique \ssi
\[
    u_0 \in \R \et \exists r \in \R \tq u_{n + 1} = u_n + r
\]
\end{dfn}

\begin{prp}
$(u_n)$ est une suite arithmétique de raison r,
\[
    \forall n \in \N, \;\; u_n = u_0 + n r
\]
\end{prp}

\begin{prp}
\[
    \sum_{k = 0}^{n} u_k = (n + 1) u_0 + r \frac{n (n + 1)}{2}
\]
\end{prp}

\begin{dfn}[Suite géométrique]
$(u_n)$ est une suite arithmétique \ssi
\[
    u_0 \in \R \et \exists q \in \R \tq u_{n + 1} = q u_n
\]
\end{dfn}

\begin{prp}
$(u_n)$ est une suite géométrique de raison q,
\[
    \forall n \in \N, \;\; u_n = u_0 q^n
\]
\end{prp}

\begin{prp}
\[
    \sum_{k = 0}^{n} u_k = 
    \begin{cases}
        u_0 \frac{1 - q^{n + 1}}{1 - q} &\text{si } q \neq 1 \\
        (n + 1) u_0                     &\text{si } q = 1
    \end{cases}
\]
\end{prp}

\begin{dfn}[Suites arithmético-géométrique]
$(u_n)$ est une suite arithmético-géométrique \ssi
\[
    u_0 \in \R \et \exists (a, b) \in \R^2 \tq u_{n + 1} = a u_n + b
\]
\end{dfn}


\section{Relation de comparaison}

\begin{dfn}
$(u_n)$ et $(v_n)$ sont des suites réelles, $v_n \neq 0$ à partir d'un
certain rang $n_0$. \\
$(u_n)$ est
\begin{description}
    \item[dominée par $(v_n)$] \quad \\
    \ssi la suite $\left(\frac{u_n}{v_n}\right)_{n \geq n_0}$ est bornée.
    On note $u_n = \dom(v_n)$
    \item[négligeable devant $(v_n)$] \quad \\
    \ssi $\lminf \frac{u_n}{v_n} = 0$.
    On note $u_n = \negl(v_n)$
    \item[équivalente à $(v_n)$] \quad \\
    \ssi $\lminf \frac{u_n}{v_n} = 1$.
    On note $u_n \equ v_n$
\end{description}
\end{dfn}

\begin{prp}
\begin{itemize}
    \item $u_n \equ u_n$
    \item $u_n \equ v_n \iff v_n \equ u_n$
    \item $u_n \equ v_n \et v_n \equ w_n \implies u_n \equ w_n$
    \item $u_n \equ v_n \et \lminf v_n = l \in \bar\R \implies
    \lminf u_n = l$
    \item $u_n \equ v_n \implies u_n$ est de même sique que $v_n$ à partir
    d'un certain rang
\end{itemize}
\end{prp}

\begin{rem}
\begin{att}
Ne jammais écrire $u_n \equ 0$.
\end{att}
\end{rem}

\begin{prp}
$(u_n), (v_n), (u'_n)$ et $(v'_n)$ des suites réelles ne s'annulant pas
à partir d'un certain rang.
\begin{itemize}
    \item $u_n \equ v_n \et u'_n \equ v'_n \implies u_n u'_n \equ v_n v'_n$
    \item $u_n \equ v_n \et u'_n \equ v'_n \implies 
    \frac{u_n}{u'_n} \equ \frac{v_n}{v'_n}$
\end{itemize}
\end{prp}

\begin{rem}
\begin{att}
On ne peut additioner des équivalents.
\end{att}
\end{rem}

\begin{prp}
\[
    u_n = \negl(v_n) \iff u_n + v_n \equ v_n
\]
\end{prp}

\begin{prp}
\[
    \forall \alpha > 0, \beta > 0, a > 1, \;\; 
    (\ln n)^\alpha = \negl(n^\beta), n^\beta = \negl(a^n), a^n = \negl(n!)
\]
\end{prp}


\section{Extension aux suites complexes}

Pour $(z_n)_{n \in \N} \in \C^\N$, on considére les suites
$(\Im(z_n))_{n \in \N}$ et $(\Re(z_n))_{n \in \N}$.

\begin{dfn}
$(z_n)$ est bornée \ssi $(|z_n|)_{n \in \N}$ est bornée.
\end{dfn}

\begin{prp}
$(z_n)$ est bornée \ssi $(\Im(z_n))$ et $(\Re(z_n))$ sont bornée.
\end{prp}

\begin{dfn}
$(z_n)$ \cv vers $l \in \C$ \ssi $(|z_n - l|)$ \cv vers $0$.
\end{dfn}

\begin{prp}
La limite d'une suite à valeur complexe est unique.
\end{prp}

\begin{prp}
$(z_n)$ \cv vers $l$ \ssi $(\Re(z_n), \Im(z_n))$ \cv vers
$(\Re(l), \Im(l))$.
\end{prp}

\begin{prp}
Si $(z_n)$ \cv alors $(z_n)$ est bornée.
\end{prp}

\begin{prp}
On peut également
\begin{itemize}
    \item prouver $(z_n)$ \cv vers $l \implies (|z_n|)$ \cv vers $|l|$.
    \item prouver Les opération sur les limites.
    \item définir les suites extraites.
    \item prouver $z_n$ \cv vers $l$ alors toute suite-extraite \cv vers $l$.
    \item prouver le théorème de \emph{Bolzanot-Weiersstrass}.
    \item définir les suite récurente usuelles.
\end{itemize}
\end{prp}


\section{Méthode d'étude de suite récurrente du type $u_{n + 1} = f(u_n)$}

\begin{enumerate}
    \item Verifier l'existence de chaque terme de la suite
    \item \'Etudier la monotonie de $f$
    \item Verifier la continuité de $f$
    \begin{description}
        \item[$\bs{f \cro}$] \quad \\
        \begin{enumerate}
            \item étudier la monotonie de $(u_n)$
            \item Chercher les points fixes de $f$
            \item trouver une éventuelle majoration / minoration
            \item déduire la limite de la suite $(u_n)$ à l'aide des points
            fixes
        \end{enumerate}
        \item[$\bs{f \dec}$] \quad \\
        \begin{enumerate}
            \item étudier la monotonie des suite extraite $(u_{2n})$ et
            $(u_{2n + 1})$ à l'aide de la fonction $g = f\circ f$
            \item trouver les points fixes de $g$
            \item trouver une éventuelle majoration / minoration des deux
            suites
            \item déduire la limite des suites et donc celle de $(u_n)$
        \end{enumerate}
        \item[$\bs{f \text{ non monotone}}$] \quad \\
        \begin{enumerate}
            \item démontrer que $f$ est une fonction contractante
            \item en déduire la limite de $(u_n)$
        \end{enumerate}
    \end{description}
\end{enumerate}


\end{document}
