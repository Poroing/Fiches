\documentclass{article}

\usepackage[utf8]{inputenc}
\usepackage[T1]{fontenc}
\usepackage[frenchb]{babel}
\usepackage{amssymb}
\usepackage{amsmath}
\usepackage{amsthm}
\usepackage{color}
\usepackage{centernot}
\usepackage{xspace}
\usepackage{stmaryrd}
\usepackage[pdfborder={0 0 0}]{hyperref}
\usepackage{mathtools}

\everymath{\displaystyle}

% Ensembles
\newcommand{\R}{\mathbb{R}}
\newcommand{\C}{\mathbb{C}}
\newcommand{\N}{\mathbb{N}}
\newcommand{\Z}{\mathbb{Z}}
\newcommand{\Q}{\mathbb{Q}}
\newcommand{\Pri}{\mathbb{P}}
\newcommand{\De}{\mathbb{D}}
\newcommand{\K}{\mathbb{K}}
\newcommand{\co}[1]{\mathrm{C^{#1}}}
\newcommand{\sym}{\mathcal{S}}
\newcommand{\groupeLineaire}{\mathcal{GL}}

% Abreviation de declaration
\newcommand{\resu}{(u_n)_{n \in \N} \in \R^\N}
\newcommand{\allent}{\forall n \in \N}
\newcommand{\apcr}{\exists n_0 \in \N \tq \forall n \geq n_0, \;\;}

% Operaeur logique
\newcommand{\et}{\;\;\text{ et }\;\;}
\newcommand{\ou}{\;\;\text{ ou }\;\;}
\newcommand{\tq}{\;\;\text{ tel que }\;\;}
\newcommand{\notimplies}{\centernot\implies}

% Abreviation d'operateur numérique
\newcommand{\lm}{\lim\limits}
\newcommand{\lminf}{\lm_{n \to +\infty}}
\newcommand{\ngl}[2]{\negl_{#1}\!\!\!(#2)}

% Abreviation textuelle
\newcommand{\ssi}{si et seulement si\xspace}
\newcommand{\voi}[1]{\text{ au voisinage de }#1}
\newcommand{\en}{\text{ en }}
\newcommand{\est}{\text{ est }}
\newcommand{\drt}{\text{ à droite de }}
\newcommand{\gch}{\text{ à gauche de }}
\newcommand{\strmo}{\text{ strictement monotone }}
\newcommand{\mo}{\text{ monotone }}
\newcommand{\cro}{\text{ croissante }}
\newcommand{\dec}{\text{ décroissante }}
\newcommand{\sur}{\text{ sur }}
\newcommand{\drv}{\text{ dérivable }}
\newcommand{\exs}{\text{ existe }}
\newcommand{\fn}{\text{ finie }}
\newcommand{\lips}{\text{ lipschitzienne }}
\newcommand{\lci}{loi de composition interne\xspace}
\newcommand{\cv}{converge\xspace}
\newcommand{\dv}{diverge\xspace}

% Abreviation de commande
\newcommand{\bs}[1]{\boldsymbol{#1}}
\newcommand{\equ}[1][]{\underset{#1}{\sim}}

% Constante Mathematique
\newcommand{\ex}{\mathrm{e}}

\newcommand{\e}{\!\!}
\newcommand{\mind}[4]{
    \substack {#1 \in \lib 1, #3 \rib \\ #2 \in \lib 1, #4 \rib}
}
\newcommand{\lib}{\llbracket}
\newcommand{\rib}{\rrbracket}

\newcommand{\colonne}{\mathrm{C}}
\newcommand{\ligne}{\mathrm{L}}

\newcommand{\voc}[1]{\textit{#1}}
\newcommand{\hyp}[1]{\textbf{#1}}

\DeclareMathOperator{\ch}{ch}
\DeclareMathOperator{\tah}{th}
\DeclareMathOperator{\sh}{sh}
\DeclareMathOperator{\dl}{DL}
\DeclareMathOperator*{\negl}{o}
\DeclareMathOperator*{\dom}{O}
\DeclareMathOperator{\diez}{\#}
\DeclareMathOperator{\point}{.}
\DeclareMathOperator{\D}{\mathfrak{D}}
\DeclareMathOperator{\divise}{|}
\DeclareMathOperator{\pgcd}{\wedge}
\DeclareMathOperator{\ppcm}{\vee}
\DeclareMathOperator{\union}{\cup}
\DeclareMathOperator{\inter}{\cap}
\DeclareMathOperator{\Vect}{Vect}
\DeclareMathOperator{\Lin}{\mathcal{L}}
\DeclareMathOperator{\Ker}{Ker}
\DeclareMathOperator{\Ima}{Im}
\DeclareMathOperator{\Id}{Id}
\DeclareMathOperator{\dime}{dim}
\DeclareMathOperator{\rang}{rg}
\DeclareMathOperator{\mat}{mat}
\DeclareMathOperator{\Tri}{\mathcal{T}}
\DeclareMathOperator{\Dia}{\mathcal{D}}
\DeclareMathOperator{\Card}{\mathrm{Card}}
\DeclareMathOperator{\M}{\mathcal{M}}
\DeclareMathOperator{\com}{\mathrm{com}}

\DeclarePairedDelimiter\abs{\lvert}{\rvert}

\newenvironment{att}
{\bgroup \color{red}{\Large\textbf{Attention}}\\}
{\egroup}

\theoremstyle{definition}
\newtheorem*{prop}{Propriétée}
\newtheorem*{defin}{Définition}
\theoremstyle{remark}
\newtheorem*{rema}{Remarque}
\newtheorem*{meth}{Méthode}
\theoremstyle{plain}
\newtheorem*{them}{Théorème}
\newtheorem*{coro}{Corollaire}
\newtheorem*{lemm}{Lemme}

\newenvironment{prp}[1][]
{\begin{prop}[#1]\quad\\}
{\end{prop}}
\newenvironment{dfn}[1][]
{\begin{defin}[#1]\quad\\}
{\end{defin}}
\newenvironment{rem}[1][]
{\begin{rema}[#1]\quad\\}
{\end{rema}}
\newenvironment{thm}[1][]
{\begin{them}[#1]\quad\\}
{\end{them}}
\newenvironment{cor}[1][]
{\begin{coro}[#1]\quad\\}
{\end{coro}}
\newenvironment{met}[1][]
{\begin{meth}[#1]\quad\\}
{\end{meth}}
\newenvironment{lem}[1][]
{\begin{lemm}[#1]\quad\\}
{\end{lemm}}


\title{Arithmétique dans $\Z$}

\begin{document}

\maketitle
\pagebreak
\tableofcontents

\section{Divisibilité et division euclidienne}

\begin{dfn}[Divisibilité]
Soit $(a, b) \in \Z^2$, on dit que \emph{$a$ divise $b$}
\ssi 
\[
    \exists k \in \Z \tq a = k b
\]
On note alors $a \divise b$
\end{dfn}

\begin{rem}
Dans le chapitre on note $\D(n)$ l'ensemble des diviseurs de $n$.
\end{rem}

\begin{prp}
\begin{itemize}
    \item $\forall a \in \Z, \;\; a \divise a$
    \item $\forall (a, b) \in \Z, \;\; \left\{ a \divise b \et b \divise a \right\} \iff
    |a| = |b|$
    \item $\forall (a, b, c) \in \Z, \;\; \left\{ a \divise b \et b \divise c \right\}
    \implies a \divise c$
\end{itemize}
\end{prp}

\begin{rem}
la relation `divise' est une relation d'ordre dans $\N$.
\end{rem}

\begin{prp}
\begin{itemize}
    \item $\forall d \in \Z, \;\; \forall (\alpha, \beta) \in \Z^2, \;\;
    \left\{ d \divise a \et d \divise b \right\} \implies d \divise (\alpha a + \beta b)$
    \item $\forall x \in \Z^*, \;\; a \divise b \iff a x \divise b x$
\end{itemize}
\end{prp}

\begin{thm}
\[
    \forall (a, b) \in \Z \times \N^*, \;\; \exists! \, (q, r) \in \Z^2 \tq
    \left\{ a = b q + r \et 0 \leq r < b \right\}
\]
$q$ et $r$ sont appelés respectivement le quotient et le reste de la
division euclidienne de a par b.
\end{thm}


\section{PGCD et agorithme d'Euclide}

\begin{dfn}[PDCD]
Soit $(a, b) \in \Z^2 \tq (a, b) \neq (0, 0)$.
Le plus grand diviseur commun de $a$ et de $b$ est noté $a \pgcd b$.
Par convention $0 \pgcd 0 = 0$
\end{dfn}

\begin{rem}
\[
    \forall (a, b) \in \Z^2, \;\; a \divise b \iff a \pgcd b = |a|
\]
\end{rem}

\begin{prp}
\[
    \forall (a, b, q, r) \in \Z^4, \;\; a = b q + r
    \implies \D(a) \inter \D(b) = \D(b) \inter \D(r)
    \implies a \pgcd b = b \pgcd r
\]
\end{prp}

\begin{prp}
\[
    \forall (a, b, d) \in \Z^3, \;\;
    \left\{ d \divise a \et d \divise b \right\}
    \iff d \divise (a \pgcd b)
\]
\end{prp}

\begin{prp}[relation de Bézout]
\[
    \forall (a, b) \in \Z^2, \;\; \exists \, (u, v) \in \Z^2 \tq
    u a + v b = a \pgcd b
\]
On appelle $(u, v)$ \emph{un couple de coefficient de Bézout de $a$ et $b$}
\end{prp}

\begin{dfn}[PPCM]
Soit $(a, b) \in \left(\Z^*\right)^2$.
Le plus petit des multiples strictment positifs communs à $a$ et $b$ est
noté $a \ppcm b$.
Par convention $0 \ppcm 0 = 0$.
\end{dfn}

\begin{prp}
\[
    \forall (a, b, m) \in \Z^3, \;\;
    \left\{ a \divise m \et b \divise m \right\}
    \implies (a \ppcm b) \divise m
\]
\end{prp}

\begin{prp}
\[
    \forall (a, b) \in \Z^2, \;\; (a \ppcm b) (a \pgcd b) = |a b|
\]
\end{prp}

\section{Entiers premiers entre eux}

\begin{dfn}
Soit $(a, b) \in \Z^2$, $a$ et $b$ sont \emph{premiers entre eux} \ssi
leurs seuls diviseurs communs sont $1$ et $-1$.
\end{dfn}

\begin{prp}
\begin{align*}
\forall (a, b) \in \Z^2 \\
        &a \text{ et } b \text{ sont premiers entre eux} \\
\iff    &a \pgcd b = 1 \\
\iff    &a \ppcm b = |a b|
\end{align*}
\end{prp}

\begin{thm}[de Bézout]
\[
    \forall (a, b) \in \Z^2, \;\; a \pgcd b = 1 \iff
    \exists \, (u, v) \in \Z^2 \tq u a + v b = 1
\]
\end{thm}

\begin{prp}
\[
    \forall (a, b) \in \Z^2, \;\; \exists (a', b') \in \Z^2 \tq
    \left\{a' \pgcd b' = 1 \et a = (a \pgcd b) a'
    \et b = (a \pgcd b) b' \right\}
\]
\end{prp}

\begin{cor}
\[
    \forall r \in \Q, \;\; \exists \, (p , q) \in \Z^2 \tq 
    \left\{ r = \frac{p}{q} \et p \pgcd q = 1 \right\}
\]
\end{cor}

\begin{prp}
\[
    \forall n \in \N^* - \{1\}, \;\;
    \forall (a, b_1, b_2, \ldots, b_n) \in \Z^{n + 1}, \;\;
    \forall i \in \lib 1, n \rib, \;\; a \pgcd b_i = 1
    \implies a \pgcd \prod^n_{i  = 1} b_i = 1
\]
\[
    \forall (a, b, n, m) \in \Z^2 \times \N^2, \;\;
    a \pgcd b = 1 \implies a^n \pgcd b^m = 1
\]
\end{prp}

\begin{thm}[de Gausse]
\[
    \forall (a, b, c) \in \Z^3, \quad
    \left\{ a \divise b c \et a \pgcd b = 1 \right\}
    \implies a \divise c
\]
\end{thm}

\begin{met}
Resolution d'une équation diophantienne
\begin{equation}\label{eq:dio}
    a x + b y = c
\end{equation}
On pose $d = a \pgcd b$.
\begin{enumerate}
    \item Si $\neg \left\{d \divise c\right\}$ l'équation n'a aucune
    solution.
    \item On pose $c' = \frac{c}{d}$.
    On cherche un couple de bézout, $(u, v)$, de $a$ et $b$. On a ainsi
    $(x_0, y_0) = \left(uc', vc'\right)$ une solution particulière de
    \eqref{eq:dio}.
    \item On pose $(a', b') = \left(\frac{a}{d}, \frac{b}{d}\right)$.
    L'ensemble des solutions de \eqref{eq:dio} est donc
    $\left\{ (x_0 + k b', y_0 - k a') \; \middle| \; k \in \Z \right\}$.
\end{enumerate}
\end{met}

\section{PGCD de plusieurs entiers}

Ici, $p \in \N^*-\{1\}, \quad (a_1, a_2, \ldots, a_p) \in \Z^p$

\begin{dfn}[PGCD d'un nombre fini d'entier]
Le nombre $d = a_1 \pgcd a_2 \pgcd \ldots \pgcd a_p$
est défini par:
\[   
    d = 
    \begin{cases}
        0   &\text{si } a_1 = a_2 = \ldots = a_p = 0 \\
        \text{Le plus grand des diviseurs communs à } 
        a_1, a_2, \ldots \text{ et } a_p\text{.} &\text{sinon}
    \end{cases}
\]
\end{dfn}

\begin{rem}
On définie de même le PPCM de plus de deux entiers.
\end{rem}

\begin{prp}[Relation de Bézout]
\[
    \exists \, (u_1, u_2, \ldots, u_p) \in \Z^p \tq
    a_1 u_1 + a_2 u_2 + \ldots + a_p u_p
    = a_1 \pgcd a_2 \pgcd \ldots \pgcd a_p
\]
\end{prp}

\begin{dfn}
$a_1, a_2, \ldots, a_p$ sont dits
\emph{premiers entre eux dans leur ensemble} \ssi
$a_1 \pgcd a_2 \pgcd \ldots \pgcd a_p$
\end{dfn}

\begin{prp}[Theorème de Bézout]
\[
    a_1 \pgcd a_2 \pgcd \ldots \pgcd a_p = 1 \iff
    \exists \, (u_1, u_2, \ldots, u_p) \in \Z^p \tq
    a_1 u_1 + a_2 u_2 + \ldots + a_p u_p = 1
\]
\end{prp}

\section{Nombres premiers}

\begin{dfn}
On appelle \emph{nombre premier} tout entier naturel possédent
2 diviseurs strictement positif.
\end{dfn}

\begin{prp}
Soit $p$ un nombre premier et $a \in \Z$,
\[
    \neg \left\{p \divise a\right\} \implies a \pgcd p = 1
\]
\end{prp}

\begin{cor}
\begin{itemize}
    \item Si $p$ est un nombre premier,
    $p$ est premier avec tout les entiers appartenant à
    $\left\{1, 2, \ldots, p - 1 \right\}$.
    \item deux nombre premier distinct sont premiers entre eux.
\end{itemize}
\end{cor}

\begin{prp}
Tout entier supérieur ou égale à $2$ est divisible par au moin un nombre
premier.
\end{prp}

\begin{cor}
L'ensemble des nombres premiers est infini.
\end{cor}

\begin{thm}[Fondamentale de l'arithmétique]
$
            \forall n \in \N, \quad n \geq 2  \\
\implies    \exists \, m \in \N^* \tq \exists \,
            (p_1, p_2, \ldots, p_m, \alpha_1, \alpha_2, \ldots, \alpha_m) 
    \in \Z^m \times (\N^*)^m \\ 
            \tq
    \{ n = p_1^{\alpha_1} p_2^{\alpha_2} \ldots p_m^{\alpha_m} 
    \et p_1, p_2, \ldots, p_m \text{ soient premiers deux à deux} \}
$
\end{thm}


\section{Congruences}

\begin{dfn}
Soit $(n, a, b) \in \N^* \times \Z^2$.
On dit que \emph{$a$ est congru à $b$ modulo $n$} \ssi
\[
    \exists k \in \Z \tq a = b + k n \iff n \divise a - b
\]
On note alors $a \equiv b [n]$
\end{dfn}

\begin{prp}
La relation de congruence modulo $n$ est une relation d'équivalence sur $\Z$
\end{prp}

\begin{prp}
\begin{itemize}
    \item 
    \begin{align*}
        &a \equiv b [n] \et a' \equiv b' [n] \implies \\
        &a + a' \equiv b + b' [n] \\
    \et &a - a' \equiv b - b' [n] \\
    \et &a a' \equiv b b' [n]
    \end{align*}
    \item $a \equiv b [n] \implies \forall k \in \N, \;\;
        a^k \equiv b^k [n]$
\end{itemize}
\end{prp}

\begin{lem}
$p \text{ est premier} \implies \forall k \in \lib 1, p - 1 \rib, \;\;
p \divise \binom{p}{k}$
\end{lem}

\begin{prp}[Petit théorème de Ferma]
$\forall p \in \Pri, \;\; \forall a \in \Z$
\begin{itemize}
    \item $a^p \equiv a [p]$
    \item $\neg p \divise a \implies a^{p - 1} \equiv 1 [p]$
\end{itemize}
\end{prp}

\end{document}
