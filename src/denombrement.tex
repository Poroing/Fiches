\documentclass{article}

\usepackage[utf8]{inputenc}
\usepackage[francais]{babel}
\usepackage{amssymb}
\usepackage{amsmath}
\usepackage{amsthm}
\usepackage{color}

\everymath{\displaystyle}

\newcommand{\ssi}{si et seulement si}
\newcommand{\R}{\mathbb{R}}
\newcommand{\N}{\mathbb{N}}
\newcommand{\Z}{\mathbb{Z}}
\newcommand{\resu}{(u_n)_{n \in \N} \in \R^\N}
\newcommand{\allent}{\forall n \in \N}
\newcommand{\et}{\text{ et }}
\newcommand{\ou}{\text{ ou }}
\newcommand{\tq}{\text{ tel que }}
\newcommand{\lm}{\lim\limits}
\newcommand{\voi}[1]{\text{ au voisinage de }$#1$}
\newcommand{\bs}[1]{\boldsymbol{#1}}

\newenvironment{att}
{\bgroup \color{red}{\Large\textbf{Attention}}\\}
{\egroup}

\theoremstyle{definition}
\newtheorem*{prop}{Proposition}
\newtheorem*{defin}{Définition}
\theoremstyle{remark}
\newtheorem*{rema}{Remarque}
\theoremstyle{plain}
\newtheorem*{them}{Théorème}

\newenvironment{prp}[1][]
{\begin{prop}[#1]\quad\\}
{\end{prop}}
\newenvironment{dfn}[1][]
{\begin{defin}[#1]\quad\\}
{\end{defin}}
\newenvironment{rem}[1][]
{\begin{rema}[#1]\quad\\}
{\end{rema}}
\newenvironment{thm}[1][]
{\begin{them}[#1]\quad\\}
{\end{them}}

\title{Title}

\begin{document}
\maketitle
\pagebreak

\end{document}


\title{Dénombrement}

\begin{document}

\section{Ensembles finis}

\subsection{Cardinal d'un ensemble fini}

\begin{dfn}
  Soit $E$ un ensemble.

  On dit que $E$ est \voc{fini} lorsque
  $E = \emptyset$ ou il existe $n \in \N^*$ tel qu'il
  existe une application bijective de $\lib 1, n \rib$ dans $E$.

  Dans le cas contraire on dit que $E$ est infini
\end{dfn}

\begin{prp}
  Soit $(p, q) \in (N^*)^2$.

  Si $\lib 1, p \rib$ et $\lib 1, q \rib$ sont en bijection alors
  $p = q$.
\end{prp}

\begin{dfn}
  Soit $E$ un ensemble fini non vide.

  Il existe un unique entier naturelle $n \in \N^*$
  tel que $E$ et $\lib 1, n \rib$ soient en bijection.
  Cette entier est appelé \voc{le cardinal de $E$} et
  est noté $\Card(E)$ ou $|E|$ ou $\#E$.

  Par convention $\Card(E) = 0$
\end{dfn}

\subsection{Propriétés}

\begin{prp}
  Soit $E$ un enselmble fini non vide et $a \in E$.

  $E \prive \{a\}$ est fini et $\Card(E \prive \{a\}) = \Card(E) - 1$
\end{prp}

\begin{prp}
  Soit $E$ un ensemble fini.

  Toute partie $E'$ de $E$ est finie et
  $\Card(E') \leq \Card(E)$ avec égalité \ssi
  $E' = E$
\end{prp}

\begin{prp}
  Soient $E$ un ensemble fini non vide et $F$ un ensemble.

  Si $E$ et $F$ sont en bijection alors $F$ est fini et
  $\Card(E) = \Card(E')$
\end{prp}

\begin{prp}
  Soit $E$ et $F$ des ensemble finis non vides.

  Si il existe une application de $E$ dans $F$
  \begin{description}
    \item[injective] alors $\Card(E) \leq \Card(F)$
    \item[surjective] alors $\Card(E) \geq \Card(F)$
    \item[bijective] alors $\Card(E) = \Card(F)$
  \end{description}
\end{prp}

\begin{prp}
  Soient $E$ et $F$ des ensembles finis non vides tel que $\Card(E) = \Card(F)$
  et $f : E \to F$.

  $f$ est injective $\iff$ $f$ est surjective $\iff$ $f$ est bijective.
\end{prp}

\subsection{Opérations}

\end{document}
