\documentclass{article}

\usepackage[utf8]{inputenc}
\usepackage[francais]{babel}
\usepackage{amssymb}
\usepackage{amsmath}
\usepackage{amsthm}
\usepackage{color}

\everymath{\displaystyle}

\newcommand{\ssi}{si et seulement si}
\newcommand{\R}{\mathbb{R}}
\newcommand{\N}{\mathbb{N}}
\newcommand{\Z}{\mathbb{Z}}
\newcommand{\resu}{(u_n)_{n \in \N} \in \R^\N}
\newcommand{\allent}{\forall n \in \N}
\newcommand{\et}{\text{ et }}
\newcommand{\ou}{\text{ ou }}
\newcommand{\tq}{\text{ tel que }}
\newcommand{\lm}{\lim\limits}
\newcommand{\voi}[1]{\text{ au voisinage de }$#1$}
\newcommand{\bs}[1]{\boldsymbol{#1}}

\newenvironment{att}
{\bgroup \color{red}{\Large\textbf{Attention}}\\}
{\egroup}

\theoremstyle{definition}
\newtheorem*{prop}{Proposition}
\newtheorem*{defin}{Définition}
\theoremstyle{remark}
\newtheorem*{rema}{Remarque}
\theoremstyle{plain}
\newtheorem*{them}{Théorème}

\newenvironment{prp}[1][]
{\begin{prop}[#1]\quad\\}
{\end{prop}}
\newenvironment{dfn}[1][]
{\begin{defin}[#1]\quad\\}
{\end{defin}}
\newenvironment{rem}[1][]
{\begin{rema}[#1]\quad\\}
{\end{rema}}
\newenvironment{thm}[1][]
{\begin{them}[#1]\quad\\}
{\end{them}}

\title{Title}

\begin{document}
\maketitle
\pagebreak

\end{document}


\title{Espaces vectoriels}

\begin{document}

\maketitle
\pagebreak
\tableofcontents



\part{Généralité}


\section{Définition}

\begin{dfn}
On appelle $\K$-espace vectoriel un ensemble $E$ muni d'une \lci $+$ et
d'une loi de composition externe $\point$; i-e d'une application
\begin{align*}
            \K \times E &\to E \\
    (\lambda, \vec{x})  &\mapsto \lambda \point \vec{x}
\end{align*}
vérifiant
\begin{itemize}
    \item $(E, +)$ est un groupe abélien
    \item $\forall (\vec{x}, \vec{y}) \in E^2, \forall (\lambda, \mu) \in \K^2$
    \begin{itemize}
        \item $\lambda \point (\vec{x} + \vec{y})
        = \lambda \point \vec{x} + \lambda \point \vec{y}$
        \item $(\lambda + \mu) \point \vec{x}
        = \lambda \point \vec{x} + \mu \point \vec{y}$
        \item $\lambda \point (\mu \point \vec{x})
        = (\lambda \times \mu) \point \vec{x}$
        \item $1 \point \vec{x} = \vec{x}$
    \end{itemize}
\end{itemize}
Les éléments de $E$ sont alors appellée \emph{vecteurs} et les éléments de
$\K$ sont appellée des \emph{scalaires}
\end{dfn}

\begin{dfn}
Soit $(E, +, \;\point\,)$ un $\K$-espace vectoriel, $n \in \N^*, \;\; \forall 
(\vec{x}_1, \vec{x}_2, \ldots, \vec{x}_n) \in E^n$, on appelle
\emph{combinaison linéaire de $\vec{x}_1, \vec{x}_2, \ldots, \vec{x}_n$}
tout vecteurs de la forme:
\[
    \lambda_1 \point \vec{x}_1 + \lambda_2 \point \vec{x}_2 + \ldots +
    \lambda_n \point \vec{x}_n
\]
où $(\lambda_1, \lambda_2, \ldots, \lambda_n) \in \K^n$
\end{dfn}

\begin{dfn}[Notations]
\begin{itemize}
    \item L'élément neutre pour $+$ dans $E$ est notée $\vec{0}$
    (ou $\vec{0}_E)$, il est appellé \emph{le vecteur nul de $E$}
    \item le symétrique de $\vec{x} \in E$ pour $+$ est noté $-\vec{x}$
    Aussi on pose $\forall (\vec{x}, \vec{y}) \in E^2, \;\; \vec{x} - \vec{y}
    = \vec{x} + (-\vec{y})$
\end{itemize}
\end{dfn}

\begin{prp}
$\forall \lambda \in \K^2, \;\; \forall (\vec{x}, \vec{y}) \in E^2$
\begin{itemize}
    \item $\lambda \point \vec{x} = \vec{0} \iff \{\lambda = 0 \ou
    \vec{x} = \vec{0}\}$
    \item $(\lambda - \mu) \point \vec{x} = \lambda \point \vec{x} - 
    \mu \point \vec{x}$
    \item $\lambda \point (\vec{x} - \vec{y}) = \lambda \point \vec{x} - 
    \lambda \point \vec{y}$
\end{itemize}
\end{prp}


\subsection{Sous-espace vectoriel}

\begin{dfn}[Sous-espace vectoriel]
On appelle \emph{sous-espace vectoriel de $E$}, toute partie $F$ de $E$ tel que
$(F, +, \;\point\,)$ soit un $\K$-espace vectoriel, de même $\K$ que $E$.
\end{dfn}

\begin{prp}[Caractérisation des sous-espaces vectoriels]
Soit $(E, +, \;\point\,)$ un $\K$-espace vectoriel, $F$ est un
sous-espace vectoriel de $(E, +, \;\point\,)$ \ssi
\begin{itemize}
    \item $F \subset E$
    \item $F \neq \emptyset$
    \item $\forall (\vec{x}, \vec{y}) \in F^2, \;\; \vec{x} + \vec{y} \in F
    \et \forall \lambda \in \K, \;\; \lambda \point \vec{x} \in F$
\end{itemize}
On peut remplacer la troisième proposition par:
\begin{itemize}
    \item $\forall (\vec{x}, \vec{y}) \in F^2, \;\;
    \forall (\lambda, \mu) \in \K^2, \;\; \lambda \point \vec{x} +
    \mu \point \vec{y} \in F$
    \item $\forall (\vec{x}, \vec{y}) \in F^2, \;\;
    \forall \lambda \in \K, \;\;
    \vec{x} + \lambda \point \vec{y} \in F$
\end{itemize}
\end{prp}

\begin{rem}
un sous-espace vectoriel d'un espace vectoriel $E$ est donc une partie non vide
de $E$ qui est stable par combinaisons linéaires.
\end{rem}

\begin{prp}
Une intersection de sous-espace vectoriel d'un espace vectoriel $E$ est un
sous-espace vectoriel de $E$
\end{prp}

\begin{rem}
Soit un espace vectoriel $(E, +, \;\point\,)$,
\[
    E \neq \{\vec{0}\} \implies E \text{ est infini}
\]
\end{rem}


\subsection{Sous-espace vectoriel engendré}

\begin{prp}
Soit $E$ un espace vectoriel, $n \in \N^*$ et
$(\vec{x_1}, \ldots, \vec{x}_n) \in E^n$,
Le plus petit sous-espace vectoriel de $E$ contenant
$\{ \vec{x}_1, \ldots, \vec{x}_n \}$ est l'ensemble des combinaisons
linéaires des vecteurs $\vec{x_1}, \ldots, \vec{x_n}$ qui est appellé
\emph{le sous-espace vectoriel de $E$ engendré par
$\vec{x}_1, \ldots, \vec{x}_n$} et est notée
\[
    \Vect(\{\vec{x}_1, \ldots, \vec{x}_n\})
\]
\end{prp}

\begin{prp}
Soit $E$ un $\K$-espace vectoriel et $A$ un partie quelconque de $E$,
Le plus petit sous-espace vectoriel contenant $A$ est l'ensemble notée
\[
    \Vect(A) = \bigcap_{\substack{F \text{ sev de } E \\ A \subset F}} F
\]
$\Vect(A)$ est appellée le \emph{sous-espace vectoriel de $E$
engendré par $A$}
\end{prp}


\section{Application Linéaire}

\subsection{Definition}

\begin{dfn}
Soit $E$ et $F$ des $\K$-espace vectoriel et $f : E \to F$,
$f$ est dite \emph{linéaire} lorsque
\[
    \forall (\vec{x}, \vec{y}) \in E^2, \;\;
    \forall (\lambda, \mu) \in \K^2, \;\;
    f(\lambda \point \vec{x} + \mu \point \vec{y}) =
    \lambda \point f(\vec{x}) + \mu \point f(\vec{y})
\]
\end{dfn}

\begin{dfn}
Soit $E$ et $F$ des $\K$-espace vectoriel et $f : E \to F$ linéaire,
$f$ est un
\begin{description}
    \item[isomorphisme d'espaces vectoriels] \quad \\
    \ssi $f$ est bijective
    \item[endomorphisme du $\K$-espace vectoriel $E$] \quad \\
    \ssi $F = E$
    \item[automorphisme du $\K$-espace vectoriel $E$] \quad \\
    \ssi $F = E$ et $f$ est bijective
    \item[une forme linéaire de $E$] \quad \\
    \ssi $F = \K$
\end{description}
\end{dfn}

\begin{dfn}
Soit $E$ et $F$ deux $\K$-espaces vectoriels.
\[
    \Lin(E, F) \text{est l'ensemble des fonction linéaires de } E
    \text{ dans } F
\]
\end{dfn}


\subsection{Premières propriétés}

\begin{prp}
Soient $E$ et $F$ des $\K$-espace vectoriel et $f \in \Lin(E, F)$,
\begin{gather*}
    f(\vec{0}_E) = \vec{0}_F \\
    \forall \vec{x} \in E, \;\; f(-\vec{x}) = -f(\vec{x})
\end{gather*}
\end{prp}

\begin{prp}
Une composée de fonction linéaires est une application linéaire.
\end{prp}

\begin{prp}
L'application réciproque d'un isomorphisme d'espace vectoriel est linéaire.
\end{prp}

\begin{prp}
Soient $E$ et $F$ des $\K$-espace vectoriel, $\Lin(E, F)$ est un
$\K$-espace vectoriel.
\end{prp}

\begin{prp}
Soit $E$ et $F$ des espaces vectoriel et $f \in \Lin(E, F)$
\begin{itemize}
    \item Soit $E'$ un sous-espace vectoriel de $E$, $f(E')$ est un
    sous-espace vectoriel de $F$.
    \item Soit $F'$ un sous-espace vectoriel de $F$, $f^{-1}(F')$ est un
    sous-espace vectoriel de $E$
\end{itemize}
\end{prp}

\begin{prp}
Soient $E$, $F$ et $G$ des $\K$-espace vectoriel
\begin{itemize}
    \item pour $u \in \Lin(E, F)$ fixée l'application
    \begin{align*}
        \Lin(F, G) &\to \Lin(E, G) \\
        v &\mapsto v \circ u
    \end{align*}
    est linéaire
    \item pour $v \in \Lin(F, G)$ fixée, l'application
    \begin{align*}
        \Lin(E, F) &\to \Lin(E, G) \\
        u &\mapsto v \circ u
    \end{align*}
    est linéaire
\end{itemize}
\end{prp}


\subsection{Noyau et Image d'une application linéaire}

\begin{dfn}
Soit $E$ et $F$ des espaces vectoriels et $f \in \Lin(E, F)$,
On appelle
\begin{description}
    \item[noyau de $\bs f$] l'ensemble notée
    \[
        \Ker(f) = \left\{ \vec{x} \in E \middle|
        f(\vec{x}) = \vec{0}_F \right\} = f^{-1}(\{\vec{0}_F\})
    \]
    \item[image de $\bs f$] l'ensemble notée
    \[
        \Ima(f) = \left\{ f(\vec{x}) \middle| \vec{x} \in E \right\}
        = f(E)
    \]
\end{description}
\end{dfn}

\begin{prp}
Soit $F$ et $E$ deux espaces vectoriels et $f \in \Lin(E, F)$,
$\Ker(f)$ est un sous-espace vectoriel de $E$ et $\Ima(f)$ est un
sous-espace vectoriel de $F$.
\end{prp}

\begin{prp}
Soit $f \in \Lin(E, F)$
\begin{align*}
    f \text{ est injective} &\iff \Ker(f) = \{ \vec{0}_E \} \\
    f \text{ est surjective} &\iff \Ima(f) = F
\end{align*}
\end{prp}


\section{Somme de sous-espaces vectoriels}


\subsection{Définition}

\begin{dfn}
Soit $F$ et $G$ des sous espaces vectoriel, la somme de $F$ et $G$ est
notée $F + G$ et est l'ensemble définie par
\[
    F + G = \left\{ \vec{x} + \vec{y} \middle|
    (\vec{x}, \vec{y}) \in F \times G \right\}
\]
\end{dfn}

\begin{prp}
Soit $F$ est $G$ des sous-espaces vectoriels de l'espace vectoriel $E$
\[
    F + G = \Vect( F \union G )
\]
\end{prp}


\subsection{Somme direct de sous-espace vectoriel}

\begin{dfn}
Soit $F$ et $G$ deux sous-espaces vectoriel de l'espace vectoriel $E$,
$F$ et $G$ sont dits somme direct \ssi il y a unicité pour tout vecteurs
de $F + G$ de la décomposition comme somme d'un vecteur de $F$ et d'un
vecteur de $G$ i-e
\[
    \forall \vec{z} \in F + G, \;\;
    \exists! (\vec{x}, \vec{y}) \in F \times G \tq
    \vec{z} = \vec{x} + \vec{y}
\]
On note alors $F \oplus G$
\end{dfn}

\begin{prp}
Soit $F$ et $G$ deux sous-espaces vectoriel de l'espace vectoriel $E$
\[
    F \oplus G \iff F \inter G = \{ \vec{0} \}
\]
\end{prp}

\begin{dfn}
Soient $F$ et $G$ des sous-espace vectoriel du $\K$-espace vectoriel $E$,
On dit que $F$ et $G$ sont supplémentaire dans $E$ lorsque
\[
    E = F \oplus G
\]
On dit dans ce cas que \emph{$G$ est un supplémentaire de $F$ dans $E$}
\end{dfn}

\begin{prp}
Soient $F$ et $G$ des sous-espace vectoriel du $\K$-espace vectoriel
$E$,
\[
    E = F \oplus G \iff
    \begin{cases}
        F + G = E \\
        F \inter G = \{\vec{0}\}
    \end{cases}
\]
\end{prp}

\begin{prp}
Soient $E$ et $F$ des $\K$-espaces vectoriels et $f \in \Lin(E, F)$,
On suppose que $G$ est un supplémentaire de $\Ker(f)$ dans $E$,
$f_{|G}$ réalise un isomorphisme d'espace vectoriel de $G$ sur $\Ima(f)$
\end{prp}


\subsection{Somme de plus de deux sous-espaces vectoriels}

\begin{dfn}
Soient $p \in \N^*$ et $F_1, F_2, \ldots, F_p$ des sous-espace vectoriel
du $\K$-esapce vectoriel $E$, on appelle somme de $F_1, F_2, \ldots F_p$,
l'ensemble
\[
    \sum^p_{i = 1} F_i = \left\{ \sum^p_{i = 1} \vec{x}_i \middle|
    (\vec{x}_1, \ldots, \vec{x}_p) \in \prod^p_{i = 1} F_i \right\}
\]
\end{dfn}

\begin{prp}
Soit $p \in \N^*$ et $F_1, F_2, \ldots, F_p$ sont des
sous-espaces vectoriel du $\K$-espace vectoriel $E$,
\[
    \sum^p_{i = 1} F_i = \Vect\left( \bigcup^p_{i = 1} F_i \right)
\]
\end{prp}

\begin{dfn}
Soit $p \in N^*$ et $F_1, \ldots, F_p$ des sous-espaces vectoriels
du $\K$-espace vectoriel $E$, $\sum^p_{i = 1} F_i$ est \emph{directe}
lorsque
\[
    \forall \vec{x} \in \sum^p_{i = 1} F_i, \;\;
    \exists! (\vec{x}_1, \ldots, \vec{x}_p) \in \prod^p_{i = 1} F_i
    \tq \vec{x} = \sum^p_{i = 1} \vec{x}_i
\]
On note alors $\bigoplus^p_{i = 1} F_i$
\end{dfn}

\begin{prp}
Si $p \in \N^*$ et $F_1, \ldots, F_p$ des sous-espace vectoriel du
$\K$-espace vectoriel $E$, $\bigoplus^p_{i = 1} F_i$ \ssi
$\vec{0}_E$ se décompose de manière unique comme somme de vecteurs
appartenant aux $F_i$
\end{prp}

\begin{prp}
Soit $p \in \N^*$, $F_1, \ldots, F_p$ des sous espace vectoriel du
$\K$-espace vectoriel $E$ tel que $E = \bigoplus^p_{i = 1} F_i$, $F$
un $\K$-espace vectoriel et
$\forall i \in \lib 1, p \rib, \;\; u_i \in \Lin(F_i, F)$,
\[
    \exists! u \in \Lin(E, F) \tq \forall i \in \lib 1, p \rib, \;\;
    u_{| F_i} = u_i
\]
\end{prp}


\section{Endomorphismes d'un espace vectoriel}


\subsection{Propriétés algébriques}

\begin{prp}
$(\Lin(E), +, \circ)$ est un anneau.
\end{prp}

\begin{dfn}
L'ensemble des automorphismes de $E$ est un groupe pour la loi $\circ$,
appellé le \emph{groupe linéaire de $E$} et est notée $GL\,(E)$
\end{dfn}


\subsection{Endomorphismes particuliers}

\begin{dfn}[Homethétie]
Soit $E$ un $\K$-espace vectoriel et $f \in Lin(E) \tq \exists k \in \K
\tq f = k \point \Id_E$, $f$ est une homothétie.
\end{dfn}

\begin{prp}
Soit $E$ un $\K$-espace vectoriel et $H$ l'ensemble des homothéties
appartenant à $\Lin(E)$.
$(H, +, \circ)$ est un corps
\end{prp}

\begin{dfn}[Projecteur]
Soit $E$ un $\K$-espace vectoriel, $F$ et $G$ deux sous-espaces vectoriels
\emph{suplémentaires} de $E$ on appelle \emph{projecteur sur $F$
parallélement à $G$} l'application:
\begin{align*}
    p:  E &\to E \\
        x &\mapsto x_1
\end{align*}
Où $x = x_1 + x_2 \tq (x_1, x_2) \in F \times G$
\end{dfn}

\begin{dfn}
Soit $E$ un $\K$-espace vectoriel, $F$ et $G$ des sous-espaces vectoriels
de $E$ tel que $E = F \oplus G$, $p$ la projection sur $F$ parallélement à
$G$ et $q$ la projection sur $G$ parallélement à $F$.
\[
    p + q = \Id_E
\]
$p$ et $q$ sont des \emph{projecteurs addociès}
\end{dfn}

\begin{prp}
Soit $E$ un $\K$-espace vectoriel, $F$ et $G$ des sous-espaces vectoriels
de $E$ tel que $E = F \oplus G$ et $p$ la projection sur $F$ parallélement à
$G$.
\begin{gather*}
    p \in \Lin(E) \\
    p \circ p = p \\
    \Ima(p) = F \et \Ker(f) = G
\end{gather*}
\end{prp}

\begin{prp}
Soit $E$ un $\K$-espace vectoriel, $F$ et $G$ des sous-espaces vectoriels
de $E$ et $p \in \Lin(E) \tq p \circ p = p$.
$p$ est un projecteur sur $\Ima(p)$ parallélement à $\Ker(p)$.
\end{prp}

\begin{dfn}[Symétrie]
Soit $E$ un $\K$-espace vectoriel, $F$ et $G$ des sous-espaces vectoriels
supplémentaire de $E$. On appelle symétrie par rapport à $F$ parallélement
à $G$ l'application:
\begin{align*}
    s:  E &\to E \\
        x &\mapsto x_1 - x_2
\end{align*}
Où $x = x_1 + x_2 \tq (x_1, x_2) \in F \times G$
\end{dfn}

\begin{prp}
Soit $E$ un $\K$-espace vectoriel, $F$ et $G$ des sous-espaces vectoriels
supplémentaires et $s$ ma symétrie par rapport à $F$ parallélement à $G$
\begin{gather*}
s \in \Lin(E) \\
s \circ s = \Id_E \\
F = \Ker(s - \Id_E) \et G = \Ker(s + \Id_E)
\end{gather*}
\end{prp}

\begin{prp}
Soit $E$ un $\K$-espace vectoriel et $s \in \Lin(E) \tq s \circ s = \Id_E$.
Alors $s$ est la symétrie par rapport à $\Ker(s - \Id_E)$ parallélement à
$\Ker(s + \Id_E)$.
\end{prp}



\part{Espaces vectoriels de dimension finie}


\section{Familles libres, famille génératrices, bases}


\subsection{Définition}

\begin{dfn}
Soit $E$ un $\K$-espace vectoriel, $I$ un ensemble non vide et
$(\alpha_i)_{i \in I}$ une famille d'élément de $E$ ou une famille
d'élément de $\K$. On dit que la famille $(\alpha_i)_{i \in I}$ est à
\emph{support finie} ou que les $\alpha_i$ sont \emph{presque tous nul}
lorqu'il existe un nombre fini d'éléments de la famille non nuls.
\end{dfn}

\begin{dfn}
Soit $E$ un $\K$-espace vectoriel, $n \in \N^*$ et 
$(x_1, x_2, \ldots, x_n) \in E^n$. On dit que la famille
$(x_1, \ldots, x_n)$ est \emph{libre} ou que $x_1, x_2, \ldots x_{n - 1} 
et x_n$ sont \emph{linéairement indépendants} lorsque
\[
    \forall (\lambda_1, \lambda_2, \ldots, \lambda_n) \in \K^n, \;\;
    \sum_{i = 1}^n \lambda_i \point x_i = 0 \implies \forall
    i \in \lib 1, n \rib, \;\; \lambda_i = 0
\]
Sinon la famille est dite \emph{liée}.
\end{dfn}

\begin{dfn}
Soit $E$ un $\K$-espace vectoriel, $I$ un ensemble non vide et
$(x_i)_{i \in I}$ une famille de vecteurs de $E$. On dit que la famille
$(x_i)_{i \in I}$ est \emph{libre} lorsque toute les sous-familles finie de
$(x_i)_{i \in I}$ sont libres.
\end{dfn}

\begin{prp}
Soit $E$ un $\K$-espace vectoriel, $I$ un ensemble non vide et
$(x_i)_{i \in I}$ une famille de vecteurs de $E$, $(x_i)_{i \in I}$ est
liée \ssi au moin un des vecteurs de cette famille est une combinaison
linéaire finie d'autres vecteurs de la famille.
\end{prp}

\begin{prp}
Soit $E$ un $\K$-espace vectoriel, $I$ un ensemble non vide et
$(x_i)_{i \in I}$ une famille de vecteurs de $E$. 
$\Vect(\left\{(x_i)_{i \in I}\right\})$ est l'ensemble des combinaisons
linéaires des vecteurs cette famille.
\end{prp}

\subsection{Base d'un espace vectoriel}

\begin{dfn}
Soit $E$ un $\K$-espace vectoriel, $I$ un ensemble non vide et
$(x_i)_{i \in I}$ une famille de vecteurs de $E$. On dit que la famille
$(x_i)_{i \in I}$ est \emph{génératrice dans $E$} lorsque
\[
    \Vect(\left\{(x_i)_{i \in I}\right\}) = E
\]
\end{dfn}

\begin{dfn}[Base]
Soit $E$ un $\K$-espace vectoriel. On appelle \emph{base de $E$} toute
famille de vecteurs de $E$ qui est à la fois \emph{libre} et
\emph{génératrice} dans $E$.
\end{dfn}

\begin{prp}[Caractérisation des bases]
Soit $E$ un $\K$-espace vectoriel, $I$ un ensemble non vide et
$B = (e_i)_{i \in I}$. $B$ est une base de $E$ \ssi $\exists!
(x_i)_{i \in I}$ une famille de scalaires à support fini tel que
$x = \sum_{i \in I} x_i e_i$.
La famille $(x_i)_{i \in I}$ est appellée famille des \emph{coordonée
du vecteur $x$ dans la base $B$}
\end{prp}


\section{Dimension d'un espace vectoriel}

\begin{dfn}
Soit $E$ un $\K$-espace vectoriel, $E$ est dit de \emph{dimension finie}
lorsque $E$ admet au moin une \emph{famille finie génératrice}, sinon
$E$ est dit de dimension infinie.
\end{dfn}

\begin{thm}
Soit $E$ un $\K$-espace vectoriel, $n \in \N^*$. Si $E$ admet une famille
\emph{génératrice} de $n$ vecteurs alors toute famille de $(n + 1)$
vecteurs est \emph{liée}.
\end{thm}

\begin{thm}[de la base incompète]
Soit $E$ un $\K$-espace vectoriel, $(p, q) \in (\N^*)^2$,
$L = (x_1, x_2, \ldots, x_p)$ une famille \emph{libre} de $E$ et
$G = (x_1, x_2, \ldots, x_q)$ une famille \emph{génératrice} de $E$.
Il existe une \emph{base} de $E$ obtenue en ajoutant à $L$ des vecteurs
de la famille de $G$. On dit que l'on a complété $L$ en une base de $E$
à l'aide de vecteurs de $G$.
\end{thm}

\begin{thm}
Soit $E$ un $\K$-espace vectoriel de dimension finie différent de
$\{0\}$ alors
\begin{itemize}
    \item $E$ admet au moin une base finie
    \item toutes les bases de $E$ sont finies et possédent le même nombre
    de vecteurs d'une base de $E$, ce nombre est appellée 
    \emph{dimension de $E$} et est notée $\dime(E)$.
\end{itemize}
On définie $\dime(\{0\}) = 0$
\end{thm}


\section{Sous-espace vectoriel d'un espace vectoriel de dimension finie}

\begin{prp}
Soit $E$ un \hyp{$\K$-espace vectoriel} de \hyp{dimension finie} et $F$
un \hyp{sous-espace vectoriel de $E$}.
\begin{gather*}
    F \text{ est de dimension finie.} \\
    \dime(F) \leq \dime(E) \\
    \dime(F) = \dime(E) \iff E = F
\end{gather*}
\end{prp}

\begin{dfn}
Soit $E$ un \hyp{$\K$-espace vectoriel} et $F$ une \hyp{famille finie de
vecteurs de $E$}.

On appelle \voc{rang de la famille $F$} le nombre notée $\rang(F)$ tel que
\[
    \rang(F) = \dime(\Vect(F))
\]
\end{dfn}

\begin{prp}
Soit $E$ un $\K$-espace vectoriel, $F$ et $F'$ des familles finies
de vecteurs de $E$

\begin{gather*}
    F \subset F' \implies \rang(F) \leq \rang(F') \\
    \rang(F) = \Card(F) \iff F \text{ est libre}
\end{gather*}
\end{prp}


\subsection{Dimension d'une somme d'espace vectoriel}

\begin{prp}
Soit $E$ un \hyp{$\K$-espace vectoriel de} \hyp{dimension finie},
$F$ et $G$ des \hyp{sous-espaces vectoriels de $E$} et $B$ et $B'$
respectivement \hyp{des bases de $F$ et $G$}.
\[
    E = F \oplus G \iff B \union B' \text{ est une base de } E
\]
\end{prp}

\begin{prp}
Soit $E$ un \hyp{$\K$-espace vectoriel} de \hyp{dimension finie}.

Tout sous-espace vectoriel de $E$ admet au moin un sous-espace vectoriel
supplémentaire dans $E$.
\end{prp}

\begin{prp}
Soit $E$ un $\K$-espace vectoriel de dimension finie.
\[
    \forall (G, F) \text{ sous-espaces vectoriels de } E, \;\;
    E = F \oplus G \implies \dime(E) = \dime(F) + \dime(G)
\]
\end{prp}

\begin{prp}
Soit $E$ un $\K$-espace vectoriel de dimension finie, $p \in \N^*-\{1\}$,
$(F_i)_{i \in \lib 1, p \rib}$ une famille de sous-espace vectoriel de
$E$ et $(B_i)_{i \in \lib 1, p \rib}$ une famille de famille de vecteurs
de $E$ tel que $\forall i \in \lib 1, p \rib, \;\; 
B_i$ est une base de $F_i$
\[
    E = \bigoplus_{i = 1}^p F_i \iff \bigcup_{i = 1}^p B_i \text{ est
    une base de } E
\]
\end{prp}

\begin{prp}
Soit $E$ un $\K$-espace vectoriel, $p \in \N^*-\{1\}$ des
sous-espace vectoriel de $E$ de dimension finie.
\begin{gather*}
    \sum_{i = 1}^p F_i \text{ est de dimension finie} \\
    \dime(\sum_{i = 1}^p F_i) \leq \sum_{i = 1}^p \dime(F_i) \\
    \dime(\sum_{i = 1}^p F_i) = \sum_{i = 1}^p \dime(F_i) \iff
    \bigoplus_{i = 1}^p F_i
\end{gather*}
\end{prp}

\begin{prp}[Formule de Grassman]
Soit $E$ un \hyp{$\K$-espace vectoriel} et $F$ et $G$ des \hyp{sous-espace
vectoriel de $E$} de \hyp{dimension finie}.
\begin{gather*}
    F + G \text{ est de dimension finie} \\
    \dime(F + G) = \dime(F) + \dime(G) - \dime(F \inter G)
\end{gather*}
Cette dernière relation est nommée \voc{formule de Grassemann}.
\end{prp}

\begin{prp}
Soit $E$ un \hyp{$\K$-espace vectoriel} de \hyp{dimension finie} et
$F$ et $G$ des \hyp{sous-espace vectoriel de $E$}.
\begin{align*}
        &E = F \oplus G \\
\iff    &F \inter G = \{0\} \et \dime(E) = \dime(F) + \dime(G) \\
\iff    &E = F + G \et \dime(E) = \dime(F) + \dime(G) \\
\end{align*}
\end{prp}


\section{Application linéaires en dimension finie}

\begin{thm}
Soit $E$ un $\K$-espace vectoriel de dimension finie, $F$ un $\K$-espace
vectoriel, $B = (e_1, e_2, \ldots, e_{\dime(E)})$ une base de $E$ et
$e'_1, e'_2, \ldots, e'_{\dime{E}}$ des vecteurs de $F$.
\[
    \exists! f \in \Lin(E, F) \tq \forall i \in \lib 1, \dime(E) \rib,
    \;\; f(e_i) = e_i
\]
\end{thm}


\subsection{Isomorphisme d'espace vectoriel}

\begin{dfn}
Soit \hyp{$E$ et $F$ de $\K$-espaces vectoriels}

ces espaces vectoriels sont dient \voc{isomorphes} \ssi
il existe au moin un isomorphisme d'espace vectoriel de $E$ dans
$F$
\end{dfn}

\begin{prp}
Soit $E$ un $\K$-espace vectoriel de dimension finie $n$ tel que $n \neq 0$
\[
    E \text{ est un isomorphe à } \K^n
\]
\end{prp}

\begin{prp}
Soit $E$ un $\K$-espace vectoriel de dimension finie différente de $0$.
\[
    E \text{ est un isomorphe à } F \iff F \text{ est de dimension finie}
    \et \dime(F) = \dime(E)
\]
\end{prp}


\subsection{Rang d'une application linéaire}

\begin{prp}
Soient \hyp{$E$ et $F$ des $\K$-espaces vectoriels},
\hyp{$f \in \Lin(E, F)$} et \hyp{$B = (e_1, \ldots, e_n)$ une base de $E$}

\begin{gather*}
    (f(e_1), \ldots, f(e_2)) \text{ est génératrice dans } \Ima(f) \\
    f \text{ est surjective} \iff (f(e_1), \ldots, f(e_n)) \text{ est
    génératrice dans } F \\
    f \text{ est injective} \iff (f(e_1), \ldots, f(e_n)) \text{ est
    libre} \\
    f \text{ est bijective} \iff (f(e_1), \ldots, f(e_n)) \text{ est
    une base de} F
\end{gather*}
\end{prp}

\begin{prp}
Soient \hyp{$E$ et $F$ des $\K$-espaces vectoriels} de
\hyp{dimension finie} \hyp{différentes de $0$} tel que
\hyp{$\dime(F) = \dime(E)$} et \hyp{$f \in \Lin(E, F)$}

\[
    f \text{ est injective} \iff f \text{ est surjective}
    \iff f \text{ est bijective}
\]
\end{prp}

\begin{dfn}
Soient $E$ une $\K$-espace vectoriel de dimension finie différente de $0$,
$F$ un $\K$-espace vectoriel et $f \in \Lin(E, F)$. On appelle rang de $f$,
notée $\rang(f)$, le nombre $\dime(\Ima(f))$.
\end{dfn}

\begin{thm}[du rang]
Soit $E$ un $\K$-espace vectoriel de dimension finie différente de $0$,
$F$ une $\K$-espace vectoriel et $f \in \Lin(E, F)$.
\[
    \dime(E) = \dime(\Ker(f)) + \rang(f)
\]
\end{thm}

\begin{prp}
Soit $E$ et $F$ des $\K$-espace vectoriel de dimension finie différente de
$0$ et $f \in \Lin(E, F)$.
\begin{itemize}
    \item $f$ est injective \ssi $\rang(f) = \dime(E)$
    \item $f$ est surjective \ssi $\rang(f) = \dime(F)$
\end{itemize}
\end{prp}

\begin{prp}
Soit $E$, $F$ et $G$ des $\K$-espaces vectoriels de dimension finie
différente de $0$ et $f \in \Lin(E, F)$
\begin{itemize}
    \item Soit $g$ un isomorphisme d'espace vectoriel de $F$ dans $G$
    \[
        \rang(g \circ f) = \rang(f)
    \]
    \item Soit $g$ un isomorphisme d'espace vectoriel de $G$ dans
    $E$
    \[
        \rang(f \circ g) = \rang(f)
    \]
\end{itemize}
\end{prp}



\subsection{Forme linéaire et hyperplan}

\begin{dfn}
Soit $E$ un $\K$-espace vectoriel de dimension finie différente de $0$
et $B = (e_1, \ldots, e_n)$, une base de $E$.

Pour tout $i \in \lib 1, n \rib$ on définit l'application
$e^*_i : E \to \K$ qui, à tout $x \in E$ associe la $i$-eme
coordonnèee de $x$ dans $B$. 

Les applications $e^*_i$ sont les formes linéaires de $E$
appellées \voc{les formes linéaires coordonées sur $E$ relativement
à la base $B$}
\end{dfn}

\begin{prp}
Soit $E$ une $\K$-espace vectoriel de dimension finie différente de $0$
et $B$ une base de $E$.

Les formes linéaires coordonées sur $E$ relativement à $B$ constituent
une base de $\Lin(E, \K)$. Par conséquent $\dime(\Lin(E, \K)) = \dime(E)$
\end{prp}

\begin{dfn}
Soit $E$ un $\K$-espace vectoriel.

On appelle \voc{hyperplan} de $E$ le noyau d'une forme linéaire non nulle
de $E$

Soit $H$ une hyperplan de $E$, et $f \in \Lin(E, \K) \tq H = \Ker(f)$,
l'égalité ``$f(x) = 0$'' est appellée \voc{équation de $H$}.
\end{dfn}

\begin{prp}
Soit $E$ un $\K$-espace vectoriel et $H$ une sous-espace vectoriel de
$E$.
Les propositions suivantes sont équivalentes
\begin{itemize}
    \item $H$ est une hyperplan de $E$
    \item Pour toute droite vetorielle $D$ non include dans $H$,
    $E = H \oplus D$
    \item Il existe une droite vectorielle non incluse dans $H$ tel que
    $E = H \oplus D$.
\end{itemize}
\end{prp}

\begin{prp}
Soit $E$ un $\K$-espace vectoriel de dimension finie, $n$,  différente
de $0$ et $H$ un sous-espace vectoriel de $E$.
\[
    H \text{ est un hyperplan de } E \iff \dime(H) = n - 1
\]
\end{prp}

\begin{prp}
Soit $E$ un $\K$-espace vectoriel, $f$ et $g$ des formes linéaires non 
nulles de $E$.

$f$ et $g$ ont même hyperplan \ssi $f$ et $g$ sont proportionelles
\end{prp}

\begin{prp}
Soit $E$ un $\K$-espace vectoriel de dimension finie différente de $0$ et
$m \in \N^*$

\begin{itemize}
    \item l'intersection de $m$ hyperplan de $E$ est de dimension 
    supérieur ou égale à $n - m$
    \item réciproquement, tout sous-espace vectoriel de dimension
    égale à $n - m$ est l'intersection de $m$ hyperplans de $E$
\end{itemize}
\end{prp}

\end{document}
