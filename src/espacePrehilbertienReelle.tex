\documentclass{article}

\usepackage[utf8]{inputenc}
\usepackage[francais]{babel}
\usepackage{amssymb}
\usepackage{amsmath}
\usepackage{amsthm}
\usepackage{color}

\everymath{\displaystyle}

\newcommand{\ssi}{si et seulement si}
\newcommand{\R}{\mathbb{R}}
\newcommand{\N}{\mathbb{N}}
\newcommand{\Z}{\mathbb{Z}}
\newcommand{\resu}{(u_n)_{n \in \N} \in \R^\N}
\newcommand{\allent}{\forall n \in \N}
\newcommand{\et}{\text{ et }}
\newcommand{\ou}{\text{ ou }}
\newcommand{\tq}{\text{ tel que }}
\newcommand{\lm}{\lim\limits}
\newcommand{\voi}[1]{\text{ au voisinage de }$#1$}
\newcommand{\bs}[1]{\boldsymbol{#1}}

\newenvironment{att}
{\bgroup \color{red}{\Large\textbf{Attention}}\\}
{\egroup}

\theoremstyle{definition}
\newtheorem*{prop}{Proposition}
\newtheorem*{defin}{Définition}
\theoremstyle{remark}
\newtheorem*{rema}{Remarque}
\theoremstyle{plain}
\newtheorem*{them}{Théorème}

\newenvironment{prp}[1][]
{\begin{prop}[#1]\quad\\}
{\end{prop}}
\newenvironment{dfn}[1][]
{\begin{defin}[#1]\quad\\}
{\end{defin}}
\newenvironment{rem}[1][]
{\begin{rema}[#1]\quad\\}
{\end{rema}}
\newenvironment{thm}[1][]
{\begin{them}[#1]\quad\\}
{\end{them}}

\title{Title}

\begin{document}
\maketitle
\pagebreak

\end{document}


\title{Espaces préhilbertiens réelles}

\begin{document}

\maketitle
\tableofcontents

\pagebreak

Dans tous le chapitre, les espaces vectorielle sont des $\R$-espace
vectorielle

\section{Produit scalaire}

\subsection{Définition}

\begin{dfn}
Soit $E$ un $\R$-espace vectorielle.

On appelle produit scalaire sur $E$ toute application $\phi : E^2 \to \R$
tel que
\begin{itemize}
    \item $\phi$ est bilinéaire
    \item $\phi$ est symétrique, i-e
    \[
        \forall (x, y) \in E^2, \;\; \phi(y, x) = \phi(x, y)
    \]
    \item $\phi$ est positive, i-e
    \[
        \forall x \in E, \phi(x, x) \geq 0
    \]
    \item $\phi$ est définie, i-e
    \[
        \forall x \in E, \;\; \phi(x, x) = 0 \iff x = 0
    \]
\end{itemize}

Dans ce cas, $E$ muni de ce produit scalaire est appelé \voc{espace
préhilbertien}. Un espace phréhilbertien réel de dimensio nfinie est
appelé une \voc{espace euclidien}
\end{dfn}

\subsection{Inégalité de Cauchy-Schwarz}

\begin{prp}[Inégalité de Chauchy-Schwarz]
Soit $E$ un espace préhilbertion réel.
\[
    \forall (x, y) \in E^2, \;\; (x|y)^2 \leq (x|x)(y|y)
\]
avec égalité \ssi $x$ et $y$ sont colinéaire
\end{prp}

\subsection{Norme euclidienne}

\begin{dfn}
Soit $E$ un espace préhilbertien réel.

On appelle \voc{norme euclidienne associé au produit scalaire
$(\quad|\quad)$ de $E$} l'application suivante
\begin{align*}
    E &\to \R_+ \\
    x &\mapsto \sqrt{(x|x)} = ||x||
\end{align*}
\end{dfn}

\begin{prp}
Soit $E$ un espace préhilbertien réel, $||\quad||$ la norme euclidienne
associée à son produit scalaire $(\quad|\quad)$.

Alors $\forall (x, y) \in E^2$
\begin{align*}
    ||x + y||^2 &= ||x||^2 + ||y||^2 + 2(x|y) \\
    ||x - y||^2 &= ||x||^2 + ||y||^2 - 2(x|y) \\
    ||x + y||^2 + ||x - y||^2 &= 2 (||x||^2 + ||y||^2) \\
    4 (x|y) &= ||x + y||^2 - ||x - y||^2
\end{align*}
\end{prp}

\begin{prp}
Soit $E$ un espace préhilbertien réel, de norme euclidienne $||\quad||$.

\begin{gather*}
    \forall x \in E, ||x|| > 0 \\
    \forall x \in E, \;\; \forall \lambda \in \R, \;\;  ||\lambda \point x|| =
        |\lambda| \times ||x|| \\
    \forall x \in E, \;\; ||x|| = 0 \iff x = 0
\end{gather*}
\end{prp}


\subsection{Distance euclidienne}

\begin{dfn}
Soit $E$ un espace préhilbertien réel.

On appelle \voc{distance euclidienne} associé au produit scalaire de $E$
l'application
\begin{align*}
    d : E^2 &\to \R_+ \\
        (x, y) &\mapsto || x - y ||
\end{align*}
\end{dfn}

\begin{prp}
Soit $E$ un espace préhilbertien réel, si $\mathrm{d}$ est la distance
euclidienne.

\begin{gather*}
    \forall (x, y) \in E^2, \;\; \mathrm{d}\,(x, y) \in \R_+ \\
    \forall (x, y) \in E^2, \;\; \mathrm{d}\,(x, y) = \mathrm{d}\,(y, x) \\
    \forall (x, y, z) \in E^3, \;\; \mathrm{d}\,(x, z) \leq
        \mathrm{d}\,(x, y) + \mathrm{d}\,(y, z)
\end{gather*}
\end{prp}

\section{Orthogonalité}

Dans tous le paragraphe, on suppose que $E$ est un espace préhilbertien
réel

\subsection{Vecteurs orthogonaux}

\begin{dfn}
Soit $(x, y) \in E^2$.

On dit que $x$ et $y$ sont dit \voc{orthogonaux} si $(x | y) = 0$.
Dans ce cas on note $x \bot y$
\end{dfn}

\begin{dfn}
Soit $I$ un ensemble quelconque et $(x_i)_{i \in I}$ une famille de
vecteurs de $E$.

La famille $(x_i)_{i \in I}$ est dite \voc{orthogonale} lorsque les
vecteurs de la famille sont deux à deux orthogonaux
\end{dfn}

\begin{prp}
Toute famille de vecteurs non nuls qui est orthogonale est libre
\end{prp}

\begin{prp}
Si $E$ est de de dimension finie $n \in \N^*$, toute famille orthogonale
de vecteurs non nuls est finie et de cardinal inférieur ou égale à $n$
\end{prp}

\begin{thm}[de Pythagore]
Soit $(x, y) \in E^2$.
\[
    x \bot y \iff || x + y ||^2 = ||x||^2 + ||y||^2
\]
\end{thm}

\begin{thm}
Soient $n \in \N^*$ et $x_1, \ldots, x_n$ sont des vecteurs de $E$ deux
à deux orthogonaux, alors
\[
    || x_1 + \cdots + x_n ||^2 = || x_1 ||^2 + \ldots + || x_n ||^2
\]
\end{thm}

\subsection{Bases orthogonales base orthonormales}

\begin{dfn}
Un vecteurs $x$ de $E$ est dit \voc{unitaire} lorsque $||x|| = 1$
\end{dfn}

\begin{dfn}[Base orthogonales]
Soit $B$ une base de $E$.

$B$ est une \voc{base orthogonale} lorsque la famille $B$ est
orthogonale
\end{dfn}

\begin{dfn}[Base orthornomale]
Soit $B$ est une base de $E$.

$B$ est une \voc{base orthonormale} lorsque $B$ est une base orthogonale
et que tout les vecteurs de $B$ sont unitaires
\end{dfn}

\begin{prp}
Tout espace vectoriel euclidien de dimention finie différente de $0$ admet au moin
une base orthonormale.
\end{prp}

\begin{prp}[Caractérisation des base orthonormale]
Soit $E$ un espace euclidien de dimension finie $n$ différente
de $0$, $B = (e_1, \ldots, e_n)$ une base de $E$. Pour tout éléments
$x$ de $E$, on note $x_i$ la $i$-ème coordonée de $x$ dans $B$.

Les assertions suivante sont équivalentes
\begin{itemize}
    \item $B$ est une base orthonormale de $E$
    \item $\forall x \in E, \;\; x_i = (x|e_i)$
    \item $\forall x \in E, \;\; ||x||^2 = \sum^n_{i = 1} x_i^2$
    \item $\forall (x, y) \in E^2, \;\; (x|y) = \sum^n_{i = 1} x_i y_i$
\end{itemize}
\end{prp}

\subsection{Orthogonale d'une partie de $E$}

\begin{dfn}
Soit $A$ une partie quelconque de $E$.

On appelle \voc{orthogonale de $A$} l'ensemble
\[
    \left\{ x \in E \middle| \forall a \in A, x \bot A \right\}
\]
On note cette ensemble $A^\bot$
\end{dfn}

\begin{prp}
\begin{gather*}
    \forall (A, B) \in \mathcal{P}(E)^2, \;\; A \subset B \implies
        B^\bot \subset A^\bot \\
    \forall A \in \mathcal{P}(E), \;\;
        A^\bot \text{ est un sous-espace vectoriel de } E \\
    \forall (x_1, \ldots, x_n) \in E^n,
        \Vect(\{x_1, \ldots, x_n\})^\bot = \{x_1, \ldots, x_n\}^\bot
\end{gather*}
\end{prp}




\subsection{Supplémentaire orthogonale, projection orthogonale}

\begin{thm}
Soit $E$ un espace préhilbertien réel, et $F$ un sous-espace vectoriel
de $E$ \hyp{de dimension finie} Alors $F^\bot$ est un sous-espace vectoriel
de $F$ dans $E$, appelé le sous-espace vectoriel
\voc{supplémentaire orthogonal} de $F$ dans $E$. En particulier, si
$E$ est de dimension finie on a $\dime(E) = \dime(F) \oplus \dime(F^\bot)$
\end{thm}

\begin{dfn}
Soit $E$ un espace préhilbertien réel et $F$ un sous-espace vectoriel de $E$
de dimension finie. On appelle \voc{projection orthogonale sur $F$}
la projection sur $F$, parallélement à $F^\bot$
\end{dfn}

\begin{prp}
Soit $E$ un espace préhilbertion réel et $F$ un sous-espace vectoriel de
dimension finie. On suppose que $(e_1, \ldots, e_p)$ est une base orthornormale
de $F$, si $p$ est la projection orthogonale sur $F$.

\[
    \forall x \in E, \;\; p(x) = \sum^p_{i = 1} (x|e_i) e_i
\]
\end{prp}

\begin{prp}
Soit $E$ un espace vectoriel euclidien et $p \in \Lin(E)$.
les propositions suivantes sont équivalentes.
\begin{gather*}
    p \text{ est une projection orthogonale} \\
    p \circ p = p \et
    \forall (x, y) \in E^2, \;\; (x|p(y)) = (p(x)|y)
\end{gather*}
\end{prp}

\begin{prp}[inégalité de Bessel]
Soit $E$ un espace préhilbertien réel, $F$ un sous-espace vectoriel de
$E$ de dimension finie et $p$ la projection orthogonale sur $F$.

\[
    \forall x \in E, ||p(x)|| \leq ||x||
\]
\end{prp}

\begin{dfn}
Soit $E$ un espace préhilbertien réel, $F$ un sous-espace vectoriel
de $E$ de dimension finie, $p$ la projection orhogonale sur $F$ et
$x \in E$.

\[
    \forall y \in F, \;\; \mathrm{d}(x, y) \geq \mathrm{d}(x, p(x))
\]
il y a égalité \ssi $y = p(x)$

On appelle \voc{distance de $x$ à $F$} et on note
$\mathrm{d}(x, F)$ la distance $\mathrm{d}(x, p(x))$
\end{dfn}


\subsection{Produit mixte}
Dans ce paragraphe, $E$ désigne un espace euclidien de dimension $n \geq 1$
.

\begin{dfn}
Soit $(x_1, \ldots, x_n) \in E^n$ et $B$ une base orthonormale
directe de $E$.

$\det_B(x_1, \ldots, x_n)$ ne depends pas de la base choisi pour $B$.

Ce determinant est appelé \voc{le produit mixte de $x_1, \ldots, x_n$}
noté $[x_1, \ldots, x_n]$
\end{dfn}

\section{Automorphisme orthogonaux d'un espace euclidien}

Dans tous le paragraphe, $E$ désigne un espace vectoriel euclidien réel
de dimension finie $n$ non nul.

\subsection{Definition d'un automorphisme orthogonale}

\begin{dfn}
Un automorphisme $f$ de $E$ est dit \voc{orthogonale} lorsque
\[
    \forall (x, y) \in E^2, \;\; (f(x)|f(y)) = (x|y)
\]
On note $\mathcal{O}(E)$ l'ensemble des automorphismes orthogonaux de $E$
\end{dfn}

\begin{prp}[caractérisation des automorphismes orhtogonaux en dimension
    finie]
Soit $f \in \Lin(E)$ les assetions suivante sont équivalentes
\begin{gather*}
f \in \mathcal{O}(E) \\
\forall (x, y) \in E^2, \;\; (x|y) = (f(x)|f(y)) \\
\forall x \in E, \;\; ||f(x)|| = ||x|| \\
\text{pout tout base orthonormale $B$ de $E$, $f(B)$ est une base
    orthonormale de $E$} \\
\text{il existe une base orthonormale $B$ de $E$ tel que $f(B)$ soit une
    base orthonormale de $E$}
\end{gather*}
\end{prp}

\begin{prp}
L'ensemble $\mathcal{O}(E)$, muni de la loi est un groupe, appelé le
\voc{groupe orthogonale de $E$}
\end{prp}

\subsection{Symétries orthogonales}

\begin{dfn}
Si $F$ est un sous-espace vectoriel de $E$.

On appelle \voc{symétrie orthogonale par rapport à $F$} la symétrie par
rapport à $F$ parallélement à $F^\bot$
\end{dfn}

\begin{dfn}
On appelle reflection de $E$ toute symétrie orthogonale par rapport à
un hyperplan de $E$
\end{dfn}

\begin{prp}
Soit $s \in \Lin(E)$.

Les assertions suivantes sont équivalentes.

\begin{gather*}
s \text{ est une symétrie orthogonale} \\
s \circ s = \Id_E \et \forall (x, y) \in E^2, \;\; (s(x)|y) = (x|s(y))
\end{gather*}
\end{prp}

\begin{prp}
Les symétries orthogonales de $E$ sont des automorphismes orthogonaux
\end{prp}

\subsection{Matrices orthogonales}

\begin{dfn}
On appelle matrice orthogonale de $\M_n(\R)$ toute matrice
$M \in \M_n(\R)$ vérifiant $M^\intercal M = \mathrm{I}_n$.

On note $\mathcal{O}(n)$ l'ensemble des matrices orthogonales de
$\M_n(\R)$.
\end{dfn}

\begin{prp}[caractérisation des matrices orthogonales]
Soit $M \in M_n(\R)$.

Les assertions suivantes sont équvalentes:
\begin{itemize}
    \item $M \in \mathcal{O}(n)$
    \item pour toute base orthonormale $B$ de $E$, l'endomorphisme de $E$
        de matrice $M$ dans $B$ est un automorphisme orthogonale.
    \item il existe une base orthonormale $B$ de $E$ tel que
        l'endomorphisme de $E$ de matrice $M$ dans $B$ soit un
        automorphisme orthogonale.
    \item les vecteurs colonnes de $M$ formes une base orthonormale
        de $\R^n$ pour le produit scalaire canonique.
    \item les vecteurs lignes de $M$ formes une base orthonormale
        de $\R^n$ pour le produit scalaire canonique.
\end{itemize}
\end{prp}

\begin{prp}
L'ensemble $\mathcal{O}(n)$ des matrices orthogonales de $\M_n(\R)$
est un groupe par la multiplication appelé le groupe orthogonale
d'ordre $n$.
\end{prp}

\begin{prp}
Soit $B$ une base orthonormale de $E$ et $B'$ une base de $E$ et
$P$ la matrice de passage de $B$ à $B'$.

$B'$ est une base orthonormale de $E$ \ssi $P \in \mathcal{O}(n)$
\end{prp}

\subsection{Groupe spécial orthogonal}

\begin{prp}
\begin{gather*}
    \forall M \in \mathcal{O}(n), \det(M) \in \{-1, 1\} \\
    \forall f \in \mathcal{O}(n), \det(f) \in \{-1, 1\}
\end{gather*}
\end{prp}

\begin{dfn}
L'ensemble des matrices orthogonales de $\mathcal{O}(n)$ de déterminant
égal à $1$ est un sous-groupe de $(\mathcal{O}(n), \times)$, appelé
\voc{le groupe spécial orthogonale d'ordre $n$}, noté $\mathcal{SO}(n)$
\end{dfn}

\begin{dfn}
L'ensemble des automorphismes orthogonales de $E$ de
déterminant égal à $1$ est un sous-groupe de $(\mathcal{O}(E), \circ)$,
appelé \voc{le groupe spécial orthogonale de $E$}, noté
$\mathcal{SO}(n)$
\end{dfn}

\begin{dfn}
On note $\mathcal{O}^-(n)$ l'ensemble $\mathcal{O}(n) - \mathcal{SO}(n)$
et $\mathcal{O}^-(E)$ l'ensemble $\mathcal{O}(E) - \mathcal{SO}(E)$
\end{dfn}

\begin{prp}
Les reflexions de $E$ appartiennent à $\mathcal{O}^-(E)$.
\end{prp}

\begin{prp}
On suppose que $E$ est orienté. Soit $f \in \Lin(E)$.

Les assertions suivantes sont équivalentes:
\begin{itemize}
    \item $f \in \mathcal{SO}(E)$
    \item Pour toute base othonormale direct, $B$, de $E$, $f(B)$ est une
        base orthonormale directe de $E$.
    \item Il exsite une base orthonormale directe de $E$ tel que $f(B)$
        soit une base orthonormale directe de $E$
\end{itemize}
\end{prp}

\begin{prp}
Soit $M \in \M_n(\R)$.

les assertions suivantes sont équivalentes:
\begin{itemize}
    \item $M \in \mathcal{SO}(n)$
    \item Pour toute base orthonormale directe, $B$, de $\R^n$,
        l'endomorphisme de $M$ dans $B$ appartient à $\mathcal{SO}(\R^n)$.
    \item Il existe une base orthonormale directe, $B$, de $\R^n$ tel que
        l'endomorphisme de $M$ dans $B$ appartienne à $\mathcal{SO}(\R^n)$.
    \item Les vecteurs colonnes de $M$ forment une base orthonormale
        directe de $\R^n$ .
    \item Les vecteurs lignes de $M$ forment une base orthonormale
        directe de $\R^n$.
\end{itemize}
\end{prp}

\begin{prp}
On suppose que $E$ est orienté, Soit $B$ est une base orthonormale directe
et $B'$ une base de $E$, en notant $P = P_{B, B'}$ on a $B'$ est une
base orthonormale \ssi $P \in \mathcal{SO}(n)$.
\end{prp}

\subsection{Descrition des automorphismes orthogonaux de dimension $2$}

\begin{prp}
\begin{gather*}
    \mathcal{SO}(2) = \left\{
        \begin{pmatrix}
            \cos \theta & -\sin \theta \\
            \sin \theta & \cos \theta
        \end{pmatrix}
        \middle| \theta \in \R
    \right\} \\
    \mathcal{O}^-(2) = \left\{
        \begin{pmatrix}
            \cos \theta & \sin \theta \\
            \sin \theta & -\cos \theta
        \end{pmatrix}
        \middle| \theta \in \R
    \right\}
\end{gather*}
\end{prp}

\subsubsection{Etude de $\mathcal{O}^-(E)$}

\begin{prp}
Soit $f \in \mathcal{O}(E)$ et $f$ est une symétrie.

$f$ est une symétrie orthogonale.
\end{prp}

\begin{prp}
$\mathcal{O}^-(E)$ est l'ensemble des reflexions de $E$.
\end{prp}

\subsubsection{Etude de $\mathcal{SO}(E)$}

\begin{prp}
Pour tout $\theta \in \R$ on pose
\(
    R_\theta = \begin{pmatrix}
        \cos \theta & -\sin \theta \\
        \sin \theta & \cos \theta
    \end{pmatrix}
\)

\[
    \forall (\theta', \theta), R_\theta \times R_{\theta'} =
    R_{\theta + \theta'} \et R_\theta^{-1} = R_{-\theta}
\]
\end{prp}

\begin{prp}
Soit $f \in \mathcal{SO}(E)$

Il existe un unique réel $\theta$ à $2\pi$ près trl que la matrice
de $f$ dans toute base orthonormale directe de $E$ soit
\[
    R_\theta = \begin{pmatrix}
        \cos \theta & -\sin \theta \\
        \sin \theta & \cos \theta
    \end{pmatrix}
\]
On dit que $f$ est \voc{une rotation vectorielle d'angle $\theta$}
\end{prp}

\begin{prp}
Soit $a$ et $b$ deux vecteus de $E$ tel que $||a|| = ||b||$.

Il existe une unique roation $f$ de $E$ tel que $f(a) = b$
\end{prp}

\begin{dfn}
Soient $a$ et $b$ deux vecteurs non nuls de $E$.

On appelle \voc{mesure l'angle orienté de $a$ et $b$} et on note
$(a, b)$ l'angle définie à $2\pi$ près, de la rotation qui transforme
$\frac{1}{||a||} a$ en $\frac{1}{||b||} b$
\end{dfn}

\begin{prp}
Soit $(a, b) \in (E-\{0\})^2$ et $\theta$ la mesure de l'angle $(a, b)$.
\begin{gather*}
\cos \theta = \frac{(a|b)}{||a|| \point ||b||} \\
\sin \theta = \frac{[a, b]}{||a|| \point ||b||}
\end{gather*}
\end{prp}

\end{document}
