% vim: set spell spelllang=fr:
\documentclass{article}

\usepackage[utf8]{inputenc}
\usepackage[T1]{fontenc}
\usepackage[frenchb]{babel}
\usepackage{amssymb}
\usepackage{amsmath}
\usepackage{amsthm}
\usepackage{color}
\usepackage{centernot}
\usepackage{xspace}
\usepackage{stmaryrd}
\usepackage[pdfborder={0 0 0}]{hyperref}
\usepackage{mathtools}

\everymath{\displaystyle}

% Ensembles
\newcommand{\R}{\mathbb{R}}
\newcommand{\C}{\mathbb{C}}
\newcommand{\N}{\mathbb{N}}
\newcommand{\Z}{\mathbb{Z}}
\newcommand{\Q}{\mathbb{Q}}
\newcommand{\Pri}{\mathbb{P}}
\newcommand{\De}{\mathbb{D}}
\newcommand{\K}{\mathbb{K}}
\newcommand{\co}[1]{\mathrm{C^{#1}}}
\newcommand{\sym}{\mathcal{S}}
\newcommand{\groupeLineaire}{\mathcal{GL}}

% Abreviation de declaration
\newcommand{\resu}{(u_n)_{n \in \N} \in \R^\N}
\newcommand{\allent}{\forall n \in \N}
\newcommand{\apcr}{\exists n_0 \in \N \tq \forall n \geq n_0, \;\;}

% Operaeur logique
\newcommand{\et}{\;\;\text{ et }\;\;}
\newcommand{\ou}{\;\;\text{ ou }\;\;}
\newcommand{\tq}{\;\;\text{ tel que }\;\;}
\newcommand{\notimplies}{\centernot\implies}

% Abreviation d'operateur numérique
\newcommand{\lm}{\lim\limits}
\newcommand{\lminf}{\lm_{n \to +\infty}}
\newcommand{\ngl}[2]{\negl_{#1}\!\!\!(#2)}

% Abreviation textuelle
\newcommand{\ssi}{si et seulement si\xspace}
\newcommand{\voi}[1]{\text{ au voisinage de }#1}
\newcommand{\en}{\text{ en }}
\newcommand{\est}{\text{ est }}
\newcommand{\drt}{\text{ à droite de }}
\newcommand{\gch}{\text{ à gauche de }}
\newcommand{\strmo}{\text{ strictement monotone }}
\newcommand{\mo}{\text{ monotone }}
\newcommand{\cro}{\text{ croissante }}
\newcommand{\dec}{\text{ décroissante }}
\newcommand{\sur}{\text{ sur }}
\newcommand{\drv}{\text{ dérivable }}
\newcommand{\exs}{\text{ existe }}
\newcommand{\fn}{\text{ finie }}
\newcommand{\lips}{\text{ lipschitzienne }}
\newcommand{\lci}{loi de composition interne\xspace}
\newcommand{\cv}{converge\xspace}
\newcommand{\dv}{diverge\xspace}

% Abreviation de commande
\newcommand{\bs}[1]{\boldsymbol{#1}}
\newcommand{\equ}[1][]{\underset{#1}{\sim}}

% Constante Mathematique
\newcommand{\ex}{\mathrm{e}}

\newcommand{\e}{\!\!}
\newcommand{\mind}[4]{
    \substack {#1 \in \lib 1, #3 \rib \\ #2 \in \lib 1, #4 \rib}
}
\newcommand{\lib}{\llbracket}
\newcommand{\rib}{\rrbracket}

\newcommand{\colonne}{\mathrm{C}}
\newcommand{\ligne}{\mathrm{L}}

\newcommand{\voc}[1]{\textit{#1}}
\newcommand{\hyp}[1]{\textbf{#1}}

\DeclareMathOperator{\ch}{ch}
\DeclareMathOperator{\tah}{th}
\DeclareMathOperator{\sh}{sh}
\DeclareMathOperator{\dl}{DL}
\DeclareMathOperator*{\negl}{o}
\DeclareMathOperator*{\dom}{O}
\DeclareMathOperator{\diez}{\#}
\DeclareMathOperator{\point}{.}
\DeclareMathOperator{\D}{\mathfrak{D}}
\DeclareMathOperator{\divise}{|}
\DeclareMathOperator{\pgcd}{\wedge}
\DeclareMathOperator{\ppcm}{\vee}
\DeclareMathOperator{\union}{\cup}
\DeclareMathOperator{\inter}{\cap}
\DeclareMathOperator{\Vect}{Vect}
\DeclareMathOperator{\Lin}{\mathcal{L}}
\DeclareMathOperator{\Ker}{Ker}
\DeclareMathOperator{\Ima}{Im}
\DeclareMathOperator{\Id}{Id}
\DeclareMathOperator{\dime}{dim}
\DeclareMathOperator{\rang}{rg}
\DeclareMathOperator{\mat}{mat}
\DeclareMathOperator{\Tri}{\mathcal{T}}
\DeclareMathOperator{\Dia}{\mathcal{D}}
\DeclareMathOperator{\Card}{\mathrm{Card}}
\DeclareMathOperator{\M}{\mathcal{M}}
\DeclareMathOperator{\com}{\mathrm{com}}

\DeclarePairedDelimiter\abs{\lvert}{\rvert}

\newenvironment{att}
{\bgroup \color{red}{\Large\textbf{Attention}}\\}
{\egroup}

\theoremstyle{definition}
\newtheorem*{prop}{Propriétée}
\newtheorem*{defin}{Définition}
\theoremstyle{remark}
\newtheorem*{rema}{Remarque}
\newtheorem*{meth}{Méthode}
\theoremstyle{plain}
\newtheorem*{them}{Théorème}
\newtheorem*{coro}{Corollaire}
\newtheorem*{lemm}{Lemme}

\newenvironment{prp}[1][]
{\begin{prop}[#1]\quad\\}
{\end{prop}}
\newenvironment{dfn}[1][]
{\begin{defin}[#1]\quad\\}
{\end{defin}}
\newenvironment{rem}[1][]
{\begin{rema}[#1]\quad\\}
{\end{rema}}
\newenvironment{thm}[1][]
{\begin{them}[#1]\quad\\}
{\end{them}}
\newenvironment{cor}[1][]
{\begin{coro}[#1]\quad\\}
{\end{coro}}
\newenvironment{met}[1][]
{\begin{meth}[#1]\quad\\}
{\end{meth}}
\newenvironment{lem}[1][]
{\begin{lemm}[#1]\quad\\}
{\end{lemm}}


\title{Polynômes}

\begin{document}

\section{L'anneau des polynômes}

\subsection{Définition}

\begin{dfn}
  On appelle \voc{polynôme à une indéterminée à coefficient dans $\K$} toute suite $(a_n)_{n \in \N}$
  à support finie.
\end{dfn}

\begin{dfn}
  L'ensemble des polynômes à une indéterminée à coefficient dans $\K$ est notée $\K[X]$ 

  On appelle \voc{polynôme constant} tout polynôme de la forme $(a_0, 0, \ldots, 0)$

  On note $0$ a suite constante nul qui est appelé \voc{le polynôme nul}

  On appelle \voc{monôme} tout polynôme de la forme $(0, \ldots, 0, a_p, 0, \ldots, 0)$

  Deux polynômes $(a_n)$ et $(b_n)$ sont égaux \ssi $\forall n \in \N$
\end{dfn}

\subsection{Structures d'espace vectoriel et d'anneau}

\begin{dfn}
  Soit $(P = (a_n)_{n \in \N}, Q = (b_n)_{n \in \N}) \in \K[X]^2$ et
  $\lambda \in \K$
  \begin{gather*}
    P + Q = (a_n + b_n)_{n \in \N} \\
    P \times Q = (\sum^n_{i = 0} a_i b_{n - i})_{n \in \N} \\
    \lambda \point P = (\lambda \point a_n)_{n \in \N}
  \end{gather*}
\end{dfn}

\begin{prp}
  $(\K[X], +, \quad \point \quad)$ est un $\K$-espace vectoriel et
  $(\K[X], +, \times)$ est un anneau commutatif.
\end{prp}

\begin{prp}
  \[
    \forall (P, Q) \in \K[X]^2, \forall \lambda \in \K, \lambda (P \times Q) = (\lambda P) \times Q = P \times (\lambda Q)
  \]
\end{prp}

\begin{dfn}
  Soit $P \in \K[X]$.

  On note
  \begin{gather*}
    P^0 = 1_{\K[X]} \\
    P^1 = P \\
    P^n = P \times P^{n - 1}
  \end{gather*}

  On note $X^k = (\delta_{k, n})_{n \in \N}$
\end{dfn}

\begin{prp}
  La famille $(X^k)_{k \in \N}$ est une base de $\K[X]$, appelée la base
  canonique de $\K[X]$
\end{prp}
  

\begin{dfn}
  Le polynôme constant $(\lambda, 0, \ldots, 0)$ sont simplement noté
  $\lambda$
\end{dfn}

\begin{dfn}
  Soit $P \in \K[X]$.

  On peu noter
  \[
    P = a_0 + a_1 X + \cdots + a_n X^n
  \]
\end{dfn}

\subsection{Degré d'un polynôme}

\begin{dfn}
  Soit $P = \sum^\infty_{k = 0} a_k X^k \in \K[X] \prive \{0\}$

  On appelle \voc{degré de $P$} et on note $\deg(P)$ le plus grand entier
  $k \in \N \tq a_k \neq 0$

  Le degré du polynôme nul est $- \infty$
\end{dfn}

\begin{dfn}
  Soit $P = \sum^\infty_{k = 0} a_k X^k \in \K[X] \prive \{0\}$ tel que
  $\deg(P) = n$.

  $a_n$ est appelé \voc{le coefficient dominant de $P$}.
\end{dfn}

\begin{dfn}
  Un polynôme est dit \voc{unitaire} lorsqu'il est non nul
  et que son coefficient dominant est $1$
\end{dfn}

\begin{prp}
  $\forall (P, Q) \in (\K[X])^2$
  \[
    \deg(P + Q) \leq \max \left\{ \deg(P), \deg(Q) \right\}
  \]

  Et si $\deg(P) \neq \deg(Q)$
  \[
    \deg(P + Q) = \max \left\{ \deg(P), \deg(Q) \right\}
  \]
\end{prp}

\begin{prp}
\[
  \forall P \in \K[X], \forall \lambda \in \K^*, \deg (\lambda P) = \deg(P)
\]
\end{prp}

\begin{cor}
  \[
    \dim_\K(\K_n[X]) = n + 1
  \]
\end{cor}

\begin{prp}
  Soit $(P, Q) \in \K[X]^2$.
  \[
    \deg(P \times Q) = \deg(P) + \deg(Q)
  \]
\end{prp}

\begin{cor}
  Soit $(P, Q) \in \K[X]^2$.
  \[
    P \times Q = 0 \iff P = 0 \ou Q = 0
  \]
\end{cor}

\begin{cor}
  Les éléments inversible de l'anneau $(\K[X], +, \times)$ sont
  les polynômes constants non nuls.
\end{cor}

\begin{dfn}
  Soit $(A = \sum^\infty_{k = 0} a_k X^k, B) \in \K[X]^2$.

  On appelle \voc{composé de $B$ par $A$} et on note
  $A(B)$ le polynôme
  \[
    A(B) = \sum^\infty_{k = 0} a_k B^k
  \]
\end{dfn}

\begin{prp}
  Soit $(A, B) \in (\K[X] \prive \{0\})^2$.
  \[
    \deg(A(B)) = \deg(A) \times \deg(B)
  \]
\end{prp}

\section{Divisibilité et division euclidienne}

\subsection{Divisibilité}

\begin{dfn}
  Soit $(A, B) \in \K[X]$.

  On dit que \voc{$B$ divise $A$} ou que \voc{$B$ est un multiple de $A$}\ssi
  $\exists Q \in \K[X] \tq A = B \times Q$

  On note alors $B \divise A$ et on dit que
\end{dfn}

\begin{dfn}
  Soit $(P, Q) \in \K[X]^2$.

  $P$ et $Q$ sont dit \voc{associés} \ssi
  $\exists \lambda \in \K^* \tq P = \lambda Q$
\end{dfn}

\begin{prp}
  Soit $(A, B, C) \in \K[X]^2$
  
  \begin{gather*}
    A \divise A \\
    A \divise B \et B \divise C \implies A \divise C \\
    A \divise B \et B \divise A \ssi \text{$A$ et $B$ sont associés} \\
    A \divise B \et \deg(A) = \deg(B) \iff \text{$A$ et $B$ sont associés}
  \end{gather*}
\end{prp}

\subsection{Division euclidienne}

\begin{thm}
  Soit $(A, B) \in \K[X] \times (\K[X] \prive \{0\})$.

  \[
    \exists! (Q, R) \in \K[X]^2 \tq A = B Q + R \et \deg(R) < \deg(B)
  \]
\end{thm}

\section{Arithmétique dans $\K[X]$}

\subsection{Pgcd de deux polynômes}

\begin{dfn}
  Soit $(A, B) \in \K[X]^2$ tel que $A \neq 0$ ou $B \neq 0$.
  
  Tout diviseur commun à $A$ et $B$ de degré maximal est appelée
  \voc{un pgcd de $A$ et $B$}

  On convient que le seul pgcd de $0$ et $0$ est $0$
\end{dfn}

\begin{prp}
  Soit $(A, B, Q, R) \in \K[X]^4$ tel que $A = BQ + R$
  \[
    \diviseurs(A) \inter \diviseurs(B) = \diviseurs(B) \inter \diviseurs(R)
  \]
\end{prp}

\begin{thm}
  Soit $(A, B) \in (\K[X] - \{0\})^2$

  Les pgcd de $A$ et $B$ sont associés et un seul est
  unitaire, il et appelé \voc{le pgcd de $A$ et $B$}
  et est noté $A \pgcd B$
\end{thm}

\begin{thm}
  Soit $(A, B) \in (\K[X] - \{0\})^2$
  \[
    \diviseurs(A) \inter \diviseurs(B) = \diviseurs(A \pgcd B)
  \]
\end{thm}

\begin{prp}[Bézout]
  Soit $(A, B) \in \K[X]^2$
  \[
    \exists (U, V) \in \K[X]^2 \tq A U + B V = A \pgcd B
  \]
\end{prp}

\begin{prp}
  Soit $(A, B, P) \in (\K[X] - \{0\})^3$ tel que $P$ soit unitaire
  \[
    (P A \pgcd P B) = P (A \pgcd B)
  \]
\end{prp}

\subsection{Couples de polynômes premier entre eux}

\begin{dfn}
Soit $(A, B) \in (\K[X])^2$

$A$ et $B$ sont premiers entre eux lorsque
\[
    A \pgcd B = 1
\]
c'est à dire, lorsque les seuls diviseurs communs à $A$ et $B$ sont les
polynômes constant non nuls
\end{dfn}

\begin{thm}[Bézout]
Soit $(A, B) \in (\K[X])^2$

$A$ et $B$ sont premier entre eux \ssi
\[
    \exists (U, V) \in (\K[X])^2 \tq A U + B V = 1
\]
\end{thm}

\begin{thm}[Lemme de Gauss]
Soit $(A, B, C) \in (\K[X])^3$
\[
    A \divise B \times C \et A \pgcd B = 1 \implies A \divise C
\]
\end{thm}

\begin{prp}
Soit $(A, B, C) \in (\K[X])^3$
\[
    A \divise C \et B \divise C \et A \pgcd B = 1 \implies A \times B
    \divise C
\]
\end{prp}

\subsection{Pgcd de plusieurs polynômes}

\begin{prp}
\[
    \forall (A, B, C) \in (\K[X])^3, \;\;
    A \pgcd (B \pgcd C) = (A \pgcd B) \pgcd C
\]
\end{prp}

\begin{dfn}
Soit $n \in \N, n \geq 2 (A_1, \ldots, A_n) \in (\K[X])^n$

Les polynômes $A_1 \pgcd \ldots \pgcd A_n$ est appelé
\voc{le pgcd des polynômes $A_1, \ldots, A_n$}
\end{dfn}

\subsection{Ppcm de deux polynôme}

\begin{dfn}
  Soit $(A, B) \in (\K[X] - \{0\})^2$

  Tout multiple commun de $A$ et $B$ de degré minimal
  est appelé \voc{un ppcm de $A$ et $B$}

  Si $A$ ou $B$ sont nul, leurs ppcm est nul.
\end{dfn}

\begin{prp}
  Soit $(A, B) \in (\K[X] - \{0\})^2$

  Les multiples commun à $A$ et $B$ sont les multiples de $P$

  Les ppcm de $A$ et $B$ sont associés, il existe un unique
  ppcm unitaire il est notée $A \ppcm B$
\end{prp}

\begin{prp}
  Soit $(A, B) \in (\K[X] - \{0\})^2$ et $P \in \K[X]$
  unitaire.
  \[
    (P A) \ppcm (P B) = P (A \ppcm B)
  \]
\end{prp}

\begin{prp}
  Soit $(A, B) \in \K[X]^2$
  
  $(A \pgcd B) \times (A \ppcm B)$ et $A B$ sont associés
\end{prp}

\subsection{Couples de polynômes premier entre eux}

\begin{dfn}
  Soit $(A, B) \in \K[X]^2$.

  $A$ et $B$ sont premiers entre eux si et seulement si $A \pgcd B = 1$
\end{dfn}

\begin{thm}[Bézout]
  Soit $(A, B) \in \K[X]^2$.

  $A$ et $B$ sont premiers entre eux si et seulement si
  $\exists (U, V) \in \K[X]^2 \tq A U + B V = 1$
\end{thm}

\subsection{Pgcd de plusieurs polynômes}

\begin{prp}
  \[
    \forall (A, B, C) \in \K[X]^3, A \pgcd (B \pgcd C) = (A \pgcd B) \pgcd C
  \]
\end{prp}

\begin{dfn}
  Soit $n \geq 2$, $(A_1, \ldots, A_n) \in \K[X]^n$.

  Les polynômes $A_1 \pgcd \ldots \pgcd A_n$ est appelé
  \voc{Le pgcd des polynômes $A_1, \ldots, A_2$}
\end{dfn}

\begin{prp}
  Soit $(A_1, \ldots, A_n) \in \K[X]^n$.
  \[
    \exists (U_1, \ldots, U_n) \in \K[X]^n \tq A_1 U_1 + \cdots + A_n U_n
    = A_1 \pgcd \ldots \pgcd A_n
  \]
\end{prp}

\begin{dfn}
Soit $(A_1, \ldots, A_n) \in (\K[X])^n$ 

On dit que $A_1, \ldots, A_n$ sont premier entre eux dans leurs
ensemble lorsque $A_1 \pgcd \ldots \pgcd A_n = 1$
\end{dfn}

\begin{thm}[Bézout]
Soit $(A_1, \ldots, 1_n) \in (\K[X])^n$

$A_1, \ldots, A_n$ sont premiers entre dans leur ensemble \ssi
\[
    \exists (U_1, \ldots, U_n) \in (\K[X])^n \tq
    A_1 U_1 + \ldots + A_n U_n = 1
\]
\end{thm}

\section{Fonction polynômiales et racines}

\subsection{Fonction polynômiale associée à un polynôme}

\begin{dfn}
  Soit $P = \sum^\infty_{k = 0} a_k X^k \in \K[X]$

  On appelle \voc{fonction polynômiale associé à $P$} l'application
  \begin{align*}
    \tilde{P} : \K &\to \K \\
      x &\mapsto \sum^\infty_{k = 0} a_k x^k
  \end{align*}
\end{dfn}

\begin{prp}
  \begin{gather*}
    \forall (P, Q) \in \K[X]^2, \forall (\lambda, \mu) \in \K^2,
      \widetilde{\lambda P + \mu Q} = \lambda \tilde{P} + \mu \tilde{Q} \\
    \forall (P, Q) \in \K[X]^2, \widetilde{P \times Q} = \tilde{P} \times \tilde{Q} \\
    \forall (P, Q) \in \K[X]^2, \widetilde{P(Q)} = \tilde{P} \circ \tilde{Q}
  \end{gather*}
\end{prp}

\subsection{Racine d'un polynôme}

\begin{dfn}
  Soit $P \in \K[X]$ et $a \in \K$.

  On dit que $a$ est \voc{une racine de $P$} ou
  \voc{un zéro de $P$} lorsque
  \[
    \tilde{P}(a) = 0
  \]
\end{dfn}

\begin{prp}
  Soit $P \in \K[X]$ et $a \in \K$.

  $a$ est racine de $P$ si et seulement si $(X - a) \divise P$.
\end{prp}

\begin{cor}
  Soient $P \in \K[X], n \in \N^*$ et $x_1, \ldots, x_n$ des éléments
  de $\K$ deux à deux distincts.

  $x_1, \ldots, x_n$ sont des racines de $P$ \ssi $\prod^n_{i = 1} (X - x_i) \divise P$
\end{cor}

\begin{cor}
  Soit $P \in \K[X], n \in \N$.

  Si $\deg(P) < n$ et si $P$ admet $n + 1$ racines deux à deux
  distincts alors $P = 0$
\end{cor}

\begin{cor}
  Soit $n \in \N, P \in \K_n[X] \\\ \{0\}$.

  $P$ admet au plus $n$ racines dans $\K$
\end{cor}

\begin{cor}
  Soit $n \in \N, P \in \K_n[X]$

  Si il existe $x_1, \ldots, x_{n + 1}$ valeurs distinctes dans $\K$ tel que
  $\tilde{P}(x_1) = \cdots = \tilde{P}(x_n)$ alors $P$ est constant
\end{cor}

\begin{cor}
  L'application
  \begin{align*}
    \phi : \K[X] &\to \K^\K \\
      P &\mapsto \tilde{P}
  \end{align*}
\end{cor}

\begin{dfn}
  Soit $(x_1, \ldots, x_n) \in \K^n$
  
  On appelle \voc{fonctions symétrique élémentaires de
  $x_1, \ldots, x_n$}
  \[
    \sigma_k = \sum_{1 \leq i_1 < \cdots < i_k \leq n} x_{i_1} \cdots x_{i_n}
  \]
\end{dfn}

\begin{prp}
  Soit $P \in \K[X]$ et $(a_1, \ldots, a_n) \in \K[X]^n$ tel que
  $P = \sum^n_{k = 0} a_k X^k$ et $P$ est scindé sur $\K$,
  $P$ s'écris alors $P = a_n \prod^n_{k = 1} (X - x_k)$ où
  $(x_1, \ldots, x_n) \in \K^n$.

  \[
    \forall k \in \lib 1, n \rib, \sigma_k = (-1)^k \frac{a_{n - k}}{a_n}
  \]
\end{prp}


\subsection{Multiplicité des racines}

\begin{dfn}
  Soit $P \in \K[X]$, $a \in \K$ et $\alpha \in \N$.

  On dit que $a$ est une racine de $P$ d'ordre $\alpha$
  lorsque
  \[
    (X - a)^\alpha \divise P \et \not (X - a)^\alpha \div P
  \]

  C'est-à-dire
  \[
    \exists Q \in \K[X], P = (X - a)^\alpha Q \et Q(a) \neq 0
  \]

  Dans ce cas $\alpha$ est \voc{l'ordre de multiplicité} de la racine $a$
\end{dfn}

\begin{prp}
  Soit $P \in \K[X] - \{0\}$ et $x_1, \ldots, x_n$ des racines
  deux à deux distincts de $P$, d'ordre respectif $\alpha_1, \ldots, \alpha_n$
  \[
    \prod^n_{i = 1} (X - x_i)^{\alpha_i} \divise P
  \]
\end{prp}

\subsection{Polynôme scindés}

\begin{dfn}
  Soit $P \in \K[X]$

  $P$ est dit \voc{scindé} lorsque
  \[
    \exists (\lambda, n, (x_1, \ldots, x_n)) \in \K^* \times \N^* \times \K^n
    \tq P = \lambda \prod^n_{i = 1} (X - x_i)
  \]
\end{dfn}

\section{Dérivation}

\subsection{Dérivation formelle d'un polynôme}

\begin{dfn}
  Soit $P = \sum^\infty_{k = 0} a_k X^k \in \K[X]$ et $n \in \N^*$.

  On appelle \voc{polynôme dérivé de $P$} et on note $P'$ le
  polynôme
  \[
    P' = \sum^\infty_{k = 1} k a_k X^{k - 1}
  \]

  On note
  \begin{gather*}
    P^{(0)} = P \\
    P^{(n)} = (P^{(n - 1)})'
  \end{gather*}
\end{dfn}

\begin{prp}
  Soit $(P, Q) \in (\K[X])^2$ et $(\lambda, \mu) \in \K^2$.
  \begin{gather*}
    (\lambda P + \mu Q)' = \lambda P' + \mu Q' \\
    (P \times Q)' = P' \times Q + P \times Q'
  \end{gather*}
\end{prp}

\subsection{Formule de Leibniz, formule de Taylor}

\begin{prp}[Formule de Leibniz]
  Soit $(P, Q) \in (\K[X]^2)$ et $n \in \N$.
  \[
    (P \times Q)^{(n)} = \sum^n_{k = 0} \binom{n}{k} P^{(k)} Q^{(n - k)}
  \]
\end{prp}

\begin{prp}[Formule de taylor]
  Soit $P \in \K[X], a \in \K$ et $n \in \N$ tel que $\deg(P) \leq n$
  \[
    P = \sum^n_{k = 0} \frac{\tilde{P^{(k)}}(a)]}{k!} (X - a)^k
  \]
\end{prp}

\subsection{Ordre de multiplicité d'une racine et dérivées successives}

\begin{prp}
  Soit $P \in \K[X] \prive \{0\}$, $n \in \N^*$ et $a \in \K$.

  $a$ est racine d'ordre $n$ \ssi $\tilde{P}(a) = \tilde{P'}(a) = \cdots =
  \tilde{P^{(n - 1)}}(a) = 0 \et \tilde{P^{(n)}}(a) \neq 0$
\end{prp}

\section{Polynômes irréductibles de $\C[X]$ et de $\R[X]$}

\subsection{Détermination des polynômes irréductibles de $\R[X]$ et de
$\C[X]$}

\begin{dfn}
Un polynôme $P$ de $\K[X]$ est dit irréductible lorsque $P$ n'est pas
constant et lorsque ses seuls diviseurs de $\K[X]$ sont les polynômes
constant nuls et les polynômes associés à $P$
\end{dfn}

\begin{prp}
Soit $P \in \K[X]$ et $\alpha \in \C$

$\alpha$ est racine \ssi $\bar{\alpha}$ est racine de $P$
\end{prp}

\begin{thm}
Les polynôme irréductibles de $\R[X]$ sont
\begin{itemize}
    \item Les polynômes de degré $1$
    \item Les polynômes de degré $2$ de discriminant négatif
\end{itemize}
\end{thm}

\begin{thm}[de d'Alembert Gauss]
Tout polynôme non constant de $\C[X]$ admet au moins une racine complexe
\end{thm}

\begin{prp}
Tout polynôme de $\C[X]$ de degré supérieur ou égale à $1$ est scindé
sur $\C[X]$
\end{prp}

\begin{prp}
Les polynômes irréductibles de $\C[X]$ sont exactement les polynômes
de degré $1$
\end{prp}

\subsection{Décomposition en produit de facteurs irréductibles}

\begin{thm}
Tout polynôme non constant de $\K[X]$ est le produit d'un scalaire
non nul par un produit de polynômes irréductibles et unitaire $\K[X]$
De plus, il y a unicité de cette décomposition à l'ordre près des
facteurs
\end{thm}

\section{Fractions rationnelles}

\subsection{Le corps des fractions rationnelles}

\begin{dfn}
  On appelle \voc{fraction rationnelle à coefficients dans $\K$} toute
  classe d'équivalence de la relation binaire $\binaryRelation$ sur $\K[X] \times (\K[X] - \{0\})$
  suivante
  \[
    (P_1, Q_1) \binaryRelation (P_2, Q_2) \iff P_1 \times Q_2 = P_2 \times Q_1
  \]
  On note l'ensemble de ces classes d'équivalences $\K(X)$.

  Soit $(P, Q) \in \K[X] \times (\K[X] - \{0\})$. La classe d'équivalence
  de $(P, Q)$ est notée $F = \frac{P}{Q}$, on dit que $(P, Q)$ est un
  \voc{représentant} de la fraction rationnelle $F$.
\end{dfn}


\subsection{Degré d'une fraction rationnelle, partie entière}

\begin{dfn}
  Soit $F \in \K(X)$ et $(P, Q) \in (\K[X])^2 \tq F = \frac{P}{Q}$.

  $\deg(P) - \deg(Q)$ est indépendant de $P$ et $Q$ choisi. On appelle
  ce nombre \voc{degré de $F$}. On le note $\deg(F)$
\end{dfn}

\begin{prp}
  Soit $(F_1, F_2) \in \K(X)^2$
  \begin{gather*}
    \deg(F_1 + F_2) \leq \max\left\{\deg F_1, \deg F_2 \right\} \\
    \deg F_1 F_2 = \deg (F_1) + \deg (F_2)
  \end{gather*}
\end{prp}

\begin{prp}
  Soit $F \in \K(X)$.

  \[
    \exists! (E, G) \in \K[X] \times \K(X) \tq \deg G < 0 \et F = E + G
  \]

  $E$ est appelé \voc{partie entière de la fraction $F$}
\end{prp}

\subsection{Zéros et pôles d'une fraction rationnelle}

\begin{dfn}
  Soit $F \in \K(X)$ et $(P, Q) \in \K[X]^2$ tel que $\frac{P}{Q}$
  soit la forme irréductible de $F$.

  On appelle \voc{zéro} ou \voc{racine} de $F$ toute racine de $P$
\end{dfn}

\section{Formule d'interpolation de Lagrange}

\begin{dfn}
Soit $n \in \N^*$ et $x_1, \ldots, x_n$ des scalaires de $\K$ deux à deux
distincts.

Pour tout $i \in \lib 1, n \rib$, on appelle $i$-\textsuperscript{ème}
polynôme de Lagrange associé à $x_1, \ldots, x_n$ le polynôme
\[
    L_i = \prod_{j \neq i} \frac{X - x_j}{x_i - x_j}
\]
\end{dfn}

\begin{prp}
Soit $n \in \N^*$ et $x_1, \ldots, x_n$ des scalaires deux à deux
distincts.
On note $L_1, \ldots, L_2$ les polynômes de Lagrange associé à
$x_1, \ldots, x_n$ 

Alors $(L_1, \ldots, L_n)$ est une base de $\K_{n - 1}[X]$

De plus pour $P \in \K_{n - 1}[X]$ la $i$-ème coordonnée
de $P$ dans cette base est $P(x_i)$ c'est-à-dire
\[
    \forall P \in \K_{n - 1}[X], \;\; P = \sum^n_{i = 1} P(x_i) L_i
\]
\end{prp}

\begin{prp}[Formule d'interpolation de Lagrange]
Soit $n \in \N^*$, $x_1, \ldots, x_n$ des scalaires deux à deux distincts
et $y_1, \ldots, y_n$ des scalaires.

Il existe $P \in \K_{n - 1}[X]$ vérifiant 
\[
    \forall i \in \lib 1, n \rib, \;\; P(x_i) = y_i
\]
Ce polynôme est
\[
    P = \sum^n_{i = 1} y_i L_i
\]
où $L_i$ est le i\textsuperscript{ème} polynôme de Lagrange associé à
$x_1, \ldots, x_n$.

$P$ est appelé polynôme interpolateur de Lagrange.
\end{prp}

\begin{prp}
Soit $n \in \N^*$, $x_1, \ldots, x_n$ des scalaires deux à deux distincts
et $y_1, \ldots, y_n$ des scalaires.

Les polynôme $P \in \K[X]$ vérifiant
\[
    \forall i \in \lib 1, n \rib, \;\; P(x_i) = y_i
\]
Sont les polynômes de la forme
\[
    P(X) = L(X) + Q(X) \times \left( \prod^n_{i = 1} (X - x_i)
    \right)
\]
où $L$ est le polynôme interpolateur de Lagrange.
\end{prp}


\end{document}
