\documentclass{article}

\usepackage[utf8]{inputenc}
\usepackage[francais]{babel}
\usepackage{amssymb}
\usepackage{amsmath}
\usepackage{amsthm}
\usepackage{color}

\everymath{\displaystyle}

\newcommand{\ssi}{si et seulement si}
\newcommand{\R}{\mathbb{R}}
\newcommand{\N}{\mathbb{N}}
\newcommand{\Z}{\mathbb{Z}}
\newcommand{\resu}{(u_n)_{n \in \N} \in \R^\N}
\newcommand{\allent}{\forall n \in \N}
\newcommand{\et}{\text{ et }}
\newcommand{\ou}{\text{ ou }}
\newcommand{\tq}{\text{ tel que }}
\newcommand{\lm}{\lim\limits}
\newcommand{\voi}[1]{\text{ au voisinage de }$#1$}
\newcommand{\bs}[1]{\boldsymbol{#1}}

\newenvironment{att}
{\bgroup \color{red}{\Large\textbf{Attention}}\\}
{\egroup}

\theoremstyle{definition}
\newtheorem*{prop}{Proposition}
\newtheorem*{defin}{Définition}
\theoremstyle{remark}
\newtheorem*{rema}{Remarque}
\theoremstyle{plain}
\newtheorem*{them}{Théorème}

\newenvironment{prp}[1][]
{\begin{prop}[#1]\quad\\}
{\end{prop}}
\newenvironment{dfn}[1][]
{\begin{defin}[#1]\quad\\}
{\end{defin}}
\newenvironment{rem}[1][]
{\begin{rema}[#1]\quad\\}
{\end{rema}}
\newenvironment{thm}[1][]
{\begin{them}[#1]\quad\\}
{\end{them}}

\title{Title}

\begin{document}
\maketitle
\pagebreak

\end{document}


\title{Polynômes}

\begin{document}


\subsection{Couples de polynômes premier entre eux}

\begin{dfn}
Soit $(A, B) \in (\K[X])^2$

$A$ et $B$ sont premiers entre eux lorsque
\[
    A \pgcd B = 1
\]
c'est à dire, lorsque les seuls diviseurs communs à $A$ et $B$ sont les
polynômes constant non nuls
\end{dfn}

\begin{thm}[Bézout]
Soit $(A, B) \in (\K[X])^2$

$A$ et $B$ sont premier entre eux \ssi
\[
    \exists (U, V) \in (\K[X])^2 \tq A U + B V = 1
\]
\end{thm}

\begin{thm}[Lemme de Gauss]
Soit $(A, B, C) \in (\K[X])^3$
\[
    A \divise B \times C \et A \pgcd B = 1 \implies A \divise C
\]
\end{thm}

\begin{prp}
Soit $(A, B, C) \in (\K[X])^3$
\[
    A \divise C \et B \divise C \et A \pgcd B = 1 \implies A \times B
    \divise C
\]
\end{prp}

\subsection{pgcd de plusieurs polynômes}

\begin{prp}
\[
    \forall (A, B, C) \in (\K[X])^3, \;\;
    A \pgcd (B \pgcd C) = (A \pgcd B) \pgcd C
\]
\end{prp}

\begin{dfn}
Soit $n \in \N, n \geq 2 (A_1, \ldots, A_n) \in (\K[X])^n$

Les polynômes $A_1 \pgcd \ldots \pgcd A_n$ est appelé
\voc{le pgcd des polynômes $A_1, \ldots, A_n$}
\end{dfn}

\begin{dfn}
Soit $(A_1, \ldots, A_n) \in (\K[X])^n$ 

On dit que $A_1, \ldots, A_n$ sont premier entre eux dans leurs
ensemble lorsque $A_1 \pgcd \ldots \pgcd A_n = 1$
\end{dfn}

\begin{thm}[Bézout]
Soit $(A_1, \ldots, 1_n) \in (\K[X])^n$

$A_1, \ldots, A_n$ sont premiers entre dans leur ensemble \ssi
\[
    \exists (U_1, \ldots, U_n) \in (\K[X])^n \tq
    A_1 U_1 + \ldots + A_n U_n = 1
\]
\end{thm}

\section{Polynômes irréductibles de $\C[X]$ et de $\R[X]$}

\subsection{Determination des polynômes irréductibles de $\R[X]$ et de
$\C[X]$}

\begin{dfn}
Un polynôme $P$ de $\K[X]$ est dit irréductible lorsque $P$ n'est pas
constant et lorsque ses seuls diviseurs de $\K[X]$ sont les polynômes
constant nuls et les polynômes associés à $P$
\end{dfn}

\begin{prp}
Soit $P \in \K[X]$ et $\alpha \in \C$

$\alpha$ est racine \ssi $\bar{\alpha}$ est racine de $P$
\end{prp}

\begin{thm}
Les polynôme irréductibles de $\R[X]$ sont
\begin{itemize}
    \item Les polynômes de degré $1$
    \item Les polynômes de degré $2$ de discriminant négatif
\end{itemize}
\end{thm}

\begin{thm}[de d'Alembert Gauss]
Tout polynôme non constant de $\C[X]$ admet au moin une racine complexe
\end{thm}

\begin{prp}
Tout polynôme de $\C[X]$ de degré supérieur ou égale à $1$ est scindé
sur $\C[X]$
\end{prp}

\begin{prp}
Les polynômes irréductibles de $\C[X]$ sont exactement les polynômes
de degré $1$
\end{prp}



\subsection{Décomposition en produit de facteurs irréductibles}

\begin{thm}
Tout polynôme non constant de $\K[X]$ est le produit d'un scalaire
non nul par un produit de polynômes irréductibles et unitaire $\K[X]$
De plus, il y a unicité de cette décomposition à l'ordre près des
facteurs
\end{thm}

\section{Formule d'interpolation de Lagrange}

\begin{dfn}
Soit $n \in \N^*$ et $x_1, \ldots, x_n$ des scalaires de $\K$ deux à deux
distincts.

Pout tout $i \in \lib 1, n \rib$, on appelle i\textsuperscript{éme}
polynôme de Lagrange associé à $x_1, \ldots, x_n$ le polynôme
\[
    L_i = \prod_{j \neq i} \frac{X - x_j}{x_i - x_j}
\]
\end{dfn}

\begin{prp}
Soit $n \in \N^*$ et $x_1, \ldots, x_n$ des scalairs deux à deux
distincts.
On note $L_1, \ldots, L_2$ les polynômes de Lagranges associé à
$x_1, \ldots, x_n$ 

Alors $(L_1, \ldots, L_n)$ est une base de $\K_{n - 1}[X]$

De plus pour $P \in \K_{n - 1}[X]$ la i\textsuperscript{ème} coordonée
de $P$ dans cette base est $P(x_i)$ i-e
\[
    \forall P \in \K_{n - 1}[X], \;\; P = \sum^n_{i = 1} P(x_i) L_i
\]
\end{prp}

\begin{prp}[Formule d'interpolation de Lagrange]
Soit $n \in \N^*$, $x_1, \ldots, x_n$ des scalaires deux à deux distincts
et $y_1, \ldots, y_n$ des scalaires.

Il existe $P \in \K_{n - 1}[X]$ vérifiant 
\[
    \forall i \in \lib 1, n \rib, \;\; P(x_i) = y_i
\]
Ce polynôme est
\[
    P = \sum^n_{i = 1} y_i L_i
\]
où $L_i$ est le i\textsuperscript{ème} polynôme de Lagrange associé à
$x_1, \ldots, x_n$.

$P$ est appellé polynôme interpolateur de Lagrange.
\end{prp}

\begin{prp}
Soit $n \in \N^*$, $x_1, \ldots, x_n$ des scalaires deux à deux distincts
et $y_1, \ldots, y_n$ des scalaires.

Les polynôme $P \in \K[X]$ vérifiant
\[
    \forall i \in \lib 1, n \rib, \;\; P(x_i) = y_i
\]
Sont les polynômes de la forme
\[
    P(X) = L(X) + Q(X) \times \left( \prod^n_{i = 1} (X - x_i)
    \right)
\]
où $L$ est le polynôme interpolateur de Lagrange.
\end{prp}


\end{document}
