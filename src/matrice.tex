\documentclass{article}

\usepackage[utf8]{inputenc}
\usepackage[francais]{babel}
\usepackage{amssymb}
\usepackage{amsmath}
\usepackage{amsthm}
\usepackage{color}

\everymath{\displaystyle}

\newcommand{\ssi}{si et seulement si}
\newcommand{\R}{\mathbb{R}}
\newcommand{\N}{\mathbb{N}}
\newcommand{\Z}{\mathbb{Z}}
\newcommand{\resu}{(u_n)_{n \in \N} \in \R^\N}
\newcommand{\allent}{\forall n \in \N}
\newcommand{\et}{\text{ et }}
\newcommand{\ou}{\text{ ou }}
\newcommand{\tq}{\text{ tel que }}
\newcommand{\lm}{\lim\limits}
\newcommand{\voi}[1]{\text{ au voisinage de }$#1$}
\newcommand{\bs}[1]{\boldsymbol{#1}}

\newenvironment{att}
{\bgroup \color{red}{\Large\textbf{Attention}}\\}
{\egroup}

\theoremstyle{definition}
\newtheorem*{prop}{Proposition}
\newtheorem*{defin}{Définition}
\theoremstyle{remark}
\newtheorem*{rema}{Remarque}
\theoremstyle{plain}
\newtheorem*{them}{Théorème}

\newenvironment{prp}[1][]
{\begin{prop}[#1]\quad\\}
{\end{prop}}
\newenvironment{dfn}[1][]
{\begin{defin}[#1]\quad\\}
{\end{defin}}
\newenvironment{rem}[1][]
{\begin{rema}[#1]\quad\\}
{\end{rema}}
\newenvironment{thm}[1][]
{\begin{them}[#1]\quad\\}
{\end{them}}

\title{Title}

\begin{document}
\maketitle
\pagebreak

\end{document}


\title{Matrice}

\begin{document}

\maketitle
\pagebreak
\tableofcontents

\part{Matrice????}

Dans tout le document, $\K$ désigne soit $\R$ soit $\C$ et $p, q, r, n$
sont des entiers appartenant à $\N^*$.

\section{Définitions}


\subsection{Définition d'une matrice}

\begin{dfn}
On appelle \voc{matrice de type $(p, q)$} ou \voc{matrice à $p$ lignes
et $q$ collones} à coeffictient dans $\K$, toute famille
\[
    (a_{ij})_{\substack{i \in \lib 1, q \rib \\ j \in \lib 1, p \rib}}
    \in \K^{pq}
\]
Cette matrice est notée: 
\[
    \begin{pmatrix}
        a_{11} & a_{12} & \cdots & a_{1q} \\
        a_{21} & a_{22} & \cdots & a_{2q} \\
        \vdots & \vdots & \ddots & \vdots \\
        a_{p1} & a_{p2} & \cdots & a_{pq}
    \end{pmatrix}
\]
\end{dfn}

\begin{dfn}[Notation]
On note $\M_{pq}(\K)$ l'ensemble des matrice de type $(p, q)$ à
coefficient dans $\K$.

Un élément de $\M_{1q}(\K)$ est appellé \voc{matrice ligne}.

Un élément de $\M_{p1}(\K)$ est appellé \voc{matrice collone}.

Soit $A = (a_{ij})$ et $B = (b_{ij})$ deux matrices, ces matrices sont égales \ssi
\[
    A \in \M_{pq}(\K) \et B \in \M_{pq}(\K) \et
    \forall (i, j) \in \lib 1, p \rib \times \lib 1, q \rib, \;\;
    a_{ij} = b_{ij}
\]

On appelle \voc{matrice nulle} la matrice de $\M_{pq}(\K)$ dont touts les
coefficients sont nuls, cette matrice est notée $O_{pq}$
\end{dfn}

\subsection{structures d'espaces vectoriel}

\begin{dfn}
On définit la loi de composition interne $+$ dans $\M_{pq}(\K)$ ainsi:
\[
    \forall (A, B) = 
        ((a_{ij})_{\substack{ i \in \lib 1, p \rib \\ j \in \lib 1, q \rib}},
        (b_{ij})_{\substack{ i \in \lib 1, p \rib \\ j \in \lib 1, q \rib}})
        \in (\M_{pq}(\K))^2, \;\;
    A + B = (a_{ij} + b_{ij})_{\substack{ i \in \lib 1, p \rib \\
        j \in \lib 1, q \rib}}
\]

On définit la loi de composition externe $\point$ dans $\M_{pq}(\K)$ ainsi:
\[
    \forall A =
        (a_{ij})_{\substack {i \in \lib 1, p \rib \\ j \in \lib 1, q \rib}}
        \in \M_{pq}(\K), \;\;
    \forall \lambda \in \K, \;\;
    \lambda \point A = 
        (\lambda \point a_{ij})_{
            \substack {i \in \lib 1, p \rib \\ j \in \lib 1, q \rib}}
\]
\end{dfn}

\begin{prp}
$(\M_{pq}(\K), +, \; \point \;)$ est un $\K$-espace vectoriel
\end{prp}

\begin{dfn}[Notation]
On appelle \voc{matrices élémentaires de $\M_{pq}(\K)$} les matrices de
$\M_{pq}(\K)$ de la forme
\[
    E_{st} = 
    \bordermatrix{
        ~ & ~       & ~         & t         & ~         & ~         \cr
        ~ & 0       & \cdots    & 0         & \cdots    & 0         \cr
        ~ & \vdots  & \ddots    & \vdots    & \ddots    & \vdots    \cr
        s & 0       & \cdots    & 1         & \cdots    & 0         \cr
        ~ & \vdots  & \ddots    & \vdots    & \ddots    & \vdots    \cr
        ~ & 0       & \cdots    & 0         & \cdots    & 0         \cr
    }
    = (\delta_{is} \times \delta_{jt})_{
        \substack {i \in \lib 1, p \rib \\ j \in \lib 1, q \rib}}
\]
\end{dfn}

\begin{prp}
La famille
$(E_{st})_{\substack {s \in \lib 1, p \rib \\ t \in \lib 1, q \rib}}$
est appellée \voc{la base canonique de $\M_{pq}(\K)$}.
\end{prp}

\begin{prp}
$(\M_{pq}(\K), +, \; \point \;)$ est un $\K$-espace vectoriel de
dimension finie et
\[
    \dime_{\K} (\M_{pq}(\K)) = pq
\]
\end{prp}

\subsection{Matrices et applications linéaires}

\begin{dfn}
Soient $E$ et $F$ deux $\K$-espaces vectoriels respectivement de
dimension finie $q$ et $p$, $B = (e_1, \ldots, e_q)$ une base de
$E$, $B' = (e'_1, \ldots, e'_p)$ une base de $F$ et $f \in \Lin(E, F)$

On appelle \voc{matrice de $f$ dans $B$ et $B'$} la matrice de
$\M_{pq}(\K)$ dont les colonnes sont constituées des coordonées des
vecteurs $f(e_1), \ldots, f(e_q)$, cette matrice est notée
$\mat_{B, B'} (f)$.

Plus rigoureusement, on a, 
$\mat_{B, B'}(f) = (a_{ij})_{
    \substack {i \in \lib 1, p \rib \\ j \in \lib 1, q \rib}}$ \\
où $\forall (i, j) \in \lib 1, p \rib \times \lib 1, q \rib, \;\; a_{ij} $
est la $i$-ème coordonée dans $B'$ du vecteur $f(e_j)$.
\end{dfn}

\begin{prp}
Soient $E$ et $F$ des $\K$-espaces vectoriels respectivement de
dimension finie $q$ et $p$, $B$ une base de $E$ et $B'$ une base $F$
\begin{align*}
    \phi : \Lin(E, F)   &\to \M_{pq}(\K) \\
                     f  &\mapsto \mat_{B, B'}(f)
\end{align*}
est un isomorphisme d'espace vectoriel.
\end{prp}

\begin{prp}
Soient $E$ et $F$ des $\K$-espaces vectoriel repectivement de dimension
finie $q$ et $p$.

$\Lin(E, F)$ est un $\K$-espace vectoriel de dimension finie $pq$
\end{prp}

\begin{dfn}
L'application
\begin{align*}
    \phi:   \Lin(\K^q, \K^p)    &\to        \M_{pq}(\K) \\
            f                   &\mapsto    \mat_{B, B'}(f)
\end{align*}
est appellée \voc{l'isomorphisme canonique de $\Lin(\K^q, \K^p)$ vers
$\M_{pq}(\K)$}

Aussi, soit $A \in \M_{pq}(\K)$, \voc{l'application linéaire de $\K^p$ dans
$\K^q$ canoniquement associée à $A$} est l'application
$f \in \Lin(\K^q, \K^p)$ dont la matrice, dans
\hyp{la base canonique de $\K^q$ et $\K^p$} est $A$
\end{dfn}



\section{Produit matriciel}


\subsection{Definition et premières propriétés}

\begin{dfn}
Soit $A = (a_{ij})_ {
    \substack {i \in \lib 1, p \rib \\ j \in \lib 1, q \rib}}
    \in \M_{pq}(\K)$
et $B = (b_{ij})_ {
    \substack {i \in \lib 1, q \rib \\ j \in \lib 1, r \rib}}
    \in \M_{qr}(\K)$

On définit le produit $A \times B$ par $A \times B = (c_{ij})_{
    \substack { i \in \lib 1, q \rib \\ j \in \lib 1, r \rib}}$,
où $c_{ij} = \sum_{k = 1}^q a_{ik} \times b_{kj}$.
\end{dfn}

\begin{prp}
Soient $E$, $F$ et $G$ des $\K$-espace vectoriel de dimension finie
différente de $0$, $B$, $B'$ et $B''$ respectivement des bases de
$E$, $F$ et $G$, $f \in \Lin(F, G)$ et $g \in \Lin(E, G)$.

\[
    \mat_{B, B''}(f \circ g) = \mat_{B', B''}(f) \times 
    \mat_{B, B'}(g)
\]
\end{prp}

\begin{prp}
Soit $E_{s, t}$ et $E'_{u, v}(\K)$ respectivement des matrices élémentaires
de $\M_{p, q}(\K)$ et $\M_{q, r}(\K)$.

\[
    E_{s, t} \times E_{u, v} = \delta_{tu} E''_{s, v}
\]
où $E''_{s, v}(\K)$ est une matrice élémentaires de $\M_{p, r}(\K)$
\end{prp}


\subsection{Matrice carrées}

\begin{dfn}
On appelle \voc{matrice carrée} d'ordre $n$ tout élément de $\M_n(\K)$.
Soit $A = (a_{ij})_{
    \substack {i \in \lib 1, n \rib \\ j \in \lib 1, n \rib}} 
    \in \M_n(\K)$

\begin{itemize}
    \item Les coefficient $a_{ii}$ pour $i \in \lib 1, n \rib$ sont
    appellés \voc{éléments diagonaux}.
    \item La matrice $A$ est appellée \voc{matrice triangulaire supérieur}
    si $\forall (i, j) \in (\lib 1, n \rib)^2, \;\; i > j \implies 
    a_{ij} = 0$. On note $\Tri^+_n(\K)$ l'ensemble des matrices
    triangulaires supérieurs de $\M_n(\K)$
    \item La matrice $A$ est appellée \voc{matrice triangulaire inférieur}
    si $\forall (i, j) \in (\lib 1, n \rib)^2, \;\; i < j \implies 
    a_{ij} = 0$. On note $\Tri^-_n(\K)$ l'ensemble des matrices
    triangulaires inférieur de $\M_n(\K)$
    \item La matrice $A$ est appellée \voc{matrice diagonale}
    si $\forall (i, j) \in (\lib 1, n \rib)^2, \;\; i \neq j \implies 
    a_{ij} = 0$. On note $\Dia_n(\K)$ l'ensemble des matrices diagonales de
    $\M_n(\K)$
\end{itemize}
\end{dfn}

\begin{prp}
$(\M_n(\K), +, \times)$ est un anneau.
\end{prp}

\begin{dfn}
On appelle \voc{matrice inversible} de $\M_n(\K)$ toute matrice
admetant une élément symétrique pour la multiplication dans
$\M_n(\K)$

L'ensemble des matrice inversibles de $\M_n(\K)$ est un groupe pour la
multiplication matricielle, appelé le \voc{groupe linéaire d'ordre $n$},
noté $\mathrm{GL}_n(\K)$
\end{dfn}

\begin{prp}
Soit $E$ un $\K$-espace vectoriel de dimension finie différente de $0$,
$B$ une base de $E$ et
\begin{align*}
    \Phi : \Lin(E) &\to \M_n(\K) \\
        f &\mapsto \mat_B(f)
\end{align*}

$\Phi$ est un isomorphisme d'espace vectoriel vérifiant
\[
    \forall (f, g) \in (\Lin(E))^2, \Phi(f \circ g) =
    \Phi(f) \times \Phi(g)
\]

Si $E = \K^n$ et $B$ est la base canonique de $\K^n$, $\Phi$ est appelée
l'isomorphisme canonique de $\Lin(\K^n)$ sur $\M_n(\K)$
\end{prp}

\begin{prp}
Soit $E$ et $F$ deux $\K$-espaces vectoriels de même dimension finie
différente de $O$, $B$ et $B$ respectivement des bases de $E$ et $F$ et
$f \in \Lin(E, F)$

$f$ est bijective \ssi $\mat_{B, B'}(f)$ est inversible
\end{prp}

\begin{prp}
Soit $M \in \M_n(\K)$.

\begin{align*}
        &M \text{ est inversible} \\
\iff    &\exists A \in M_n(\K) \tq M \times A = I_n \\
\iff    &\exists B \in M_n(\K) \tq B \times M = I_n
\end{align*}
\end{prp}

\begin{prp}
Les ensemble $\Tri_n^+(\K)$, $\Tri_n^-(\K)$ et $\Dia_n(\K)$ sont des
sous-espace vectoriel de l'espace vectoriel $(\M_n(\K), +, \; \point \;)$
et des sous-anneaux de l'anneau $(\M_n(\K), +, \times)$.
\end{prp}


\subsection{Transposée d'une matrice}

\begin{dfn}
Soit $A \in \M_{p, q}(\K)$ tel que $A = (a_{ij})_{\mind{i}{j}{p}{q}}$

On appelle \voc{transoposée de $A$} et on note $A^\intercal$ la matrice
de $\M_{q, p}(\K)$ définie par
\[
    A^\intercal = (a_{ji})_{\mind{i}{j}{q}{p}}
\]
\end{dfn}

\begin{prp}
L'application
\begin{align*}
    \M_{p, q}(\K) &\to \M_{q, p}(\K) \\
    A &\mapsto A^\intercal
\end{align*}
est un isomorphisme d'espace vectoriel.
\end{prp}

\begin{prp}
Soit $(A, B) \in \M_{p, q}(\K) \times \M_{q, r}(\K)$
\[
    (A \times B)^\intercal = B^\intercal \times A^\intercal
\]
\end{prp}

\begin{prp}
Soit $M \in \M_n(\K)$
\begin{gather*}
    M \in \mathrm{GL}_n(\K) \iff M^\intercal \in \mathrm{GL}_n(\K) \\
    M \in \mathrm{GL}_n(\K) \implies (M^\intercal)^{-1} =
    (M^{-1})^\intercal
\end{gather*}
\end{prp}

\begin{dfn}
Soit $M \in \M_n(\K)$
\begin{itemize}
    \item On dit que $M$ est \voc{symétrique} lorsqe $M^\intercal = M$
    \item On dit que $M$ est \voc{antisymétrique} lorsque $M^\intercal = -M$
\end{itemize}
On note $\mathcal{A}_n(\K)$ l'ensemble des matrices antisymétriques de $\M_n(\K)$
et $\mathcal{S}_n(\K)$ est l'ensemble des matrices symétriques de $\M_n(\K)$
\end{dfn}

\begin{prp}
$\mathcal{S}_n(\K)$ et $\mathcal{A}_n(\K)$ sont des sous-espace vectoriel
supplémentaire de $\M_n(\K)$
\end{prp}



\section{Changements de bases}


\subsection{Matrices de passage entre deux bases}

\begin{dfn}
Soit $E$ un $\K$-espace vectoriel de dimension finie, $n$, différente de $0$
et $B = (e_1, \ldots, e_n)$ et $B' = (e'_1, \ldots, e'_n)$ deux bases de $E$.

On appelle \voc{matrice de passage de $B$ à $B'$} la matrice de $\M_n(\K)$ 
dont les colonnes sont constituées des coordonées des vecteurs de $B'$ dans
la base $B$.

\[
    P_{B, B'} =
    \bordermatrix{ 
        ~       & e'_1  & \ldots    & e'_n  \cr
        e_1     & ~     & ~         & ~     \cr
        \vdots  & ~     & ~         & ~     \cr
        e_n     & ~     & ~         & ~     \cr
   }
\]
\end{dfn}

\begin{prp}
Soit $E$ un $\K$-espace vectoriel de dimension finie $n \neq 0$ et $B$, $B'$
et $B''$ des bases de $E$.

\begin{gather*}
    P_{B, B''} = P_{B, B'} \times P_{B', B''} \\
    P_{B, B'} \in \mathrm{GL}_n(\K) \et P^{-1}_{B, B'} = P_{B', B}
\end{gather*}
\end{prp}

\begin{prp}
Soit $E$ un $\K$-espace vectoriel de dimension finie $n \neq 0$,
$B$ et $B'$ des  des bases de $E$, $x \in E$, $X$ la matrice
colonnes des coordonées de $x$ dans $B$ et $X'$ la matrice
colonnes des coordonées de $x$ dans $B'$.
\[
    X = P_{B, B'} \times X'
\]
\end{prp}


\subsection{Changement de base pour une application linéaire}

\begin{prp}
Soit $E$ et $F$ des $\K$-espaces vectoriels de dimension finie
différente de $0$, $B_1$ et $B_2$ des bases de $E$, $B'_1$ et $B'_2$
des bases de $F$ et $f \in \Lin(E, F)$

On note
\begin{gather*}
    A_1 = \mat_{B_1, B'_1}(f) \\
    A_2 = \mat_{B_2, B'_2}(f) \\
    P = P_{B_1, B_2} \\
    Q = P_{B'_1, B'_2}
\end{gather*}

\[
    A_2 = Q^{-1} \times A_1 \times P
\]
\end{prp}

\subsection{Matrices équivalentes et matrices semblables}

\begin{dfn}
Soient $(A, B) \in (\M_{p, q}(\K))^2$

On dit que $A$ et $B$ sont équivalentes lorsque
\[
    \exists (P, Q) \in \M_q(\K) \times \M_p(\K) \tq B = Q \times A \times P
\]
\end{dfn}

\begin{dfn}
Soit $r \in \lib 0, \min \{p, q\} \rib$

On note $J_r$ la matrice de $M_{p, q}(\K)$ définie par
\[
    J_r = \begin{pmatrix}
        I_r & 0 \\
        0   & 0 \\
    \end{pmatrix}
\]
\end{dfn}

\begin{prp}
Soient $E$ et $F$ des $\K$-espaces vectoriels respectivement de dimension
finie $q$ et $p$ et $f \in \Lin(E, F)$

Il existe une base $B$ de $E$ et une base $B'$ de $F$ tel que
\[
    \mat_{B, B'}(f) = J_{\rang(f)}
\]
\end{prp}

\begin{dfn}
Soit $(A, B) \in (\M_n(\K))^2$

On dit que $A$ et $B$ sont semblables lorque
\[
    \exists P \in \mathrm{GL}_n(\K) \tq B = P^{-1} \times A \times P
\]
\end{dfn}



\section{Trace d'une matrice, d'un endomorphisme}

\begin{dfn}
Soit $A = (a_{ij})_{\mind{i}{j}{n}{n}} \in \M_n(\K)$

On note $\mathrm{tr}(A)$ la \voc{trace de $A$} qui est tel que
\[
    \mathrm{tr}(A) = \sum^n_{k = 1} a_{kk}
\]
\end{dfn}

\begin{prp}
L'application trace:
\begin{align*}
    \M_n(\K) &\to \K \\
    A &\mapsto \mathrm{tr}(A) 
\end{align*}
est une forme de $\M_n(\K)$ vérifiant

\[
    \forall (A, B) \in (\M_n(\K))^2, \;\; \mathrm{tr}(AB) = \mathrm{tr}(BA)
\]
\end{prp}

\begin{prp}
Deux matrice carrée semblable ont la même trace
\end{prp}


\subsection{Trace d'un endomorphisme}

\begin{dfn}
Soit $E$ un $\K$-espace vectoriel de dimension finie différente de $0$
et $f \in \Lin(E)$

On appelle \voc{trace de $f$} la trace d'une matrice de $f$ dans une base
quelconque de $E$
\end{dfn}

\begin{prp}
Soit $E$ un $\K$-espace vectoriel de dimension finie différente $0$ et
$(f, g) \in (\Lin(E))^2$

\begin{gather*}
    \forall (\lambda, \mu) \in \K^2, \;\; \mathrm{tr}(\lambda f + \mu g) =
    \lambda \mathrm{tr}(f) + \mu \mathrm{tr}(g) \\
    \mathrm{tr}(f \circ g) = \mathrm{tr}(g \circ f)
\end{gather*}
\end{prp}


\section{Noyaux, image, rang d'une matrice}


\subsection{Application linéaire canoniquement assocée à une matrice}

\begin{dfn}
Soit $A \in \M_{p, q}(\K)$.

On appelle \voc{application linéaire canoniquement associé à $A$}
l'application linéaire de $\Lin(\K^q, \K^p)$ dont la matrice, dans la
base canonique de $\K^q$ et $\K^p$ est la matrice $A$.
\end{dfn}

\begin{dfn}
Soit $A$ une matrice.
On appelle \voc{noyau de $A$} le noyau de l'application linéaire
canoniquement associé à $A$. On note cet ensemble $\Ker(A)$.

On peut définir de la même façon l'image de $A$.
\end{dfn}

\begin{prp}
Soit $A$ une matrice carrée.
\[
    A \text{ est inversible} \iff \Ker(A) = \{0\}
\]
\end{prp}

\begin{prp}
Soit $A$ un matrice triangulaire.

$A$ est inversible \ssi ses éléments diagonaux sont non nul.
Dans ce cas, $A^{-1}$ est triangulaire de même triangularitée que $A$
\end{prp}

\begin{prp}
Soit $D \in \Dia_n(\K) \tq D =
    \begin{pmatrix}
        d_1 & \cdots & 0 \\
        \vdots & \ddots & \vdots \\
        0 & \cdots & d_n
    \end{pmatrix}
$

\[
    D \text{ est inversible} \iff \forall i \in \lib 1, n \rib, d_i \neq 0
\]
Dans ce cas, $D =
    \begin{pmatrix}
        d_1^{-1} & \cdots & 0 \\
        \vdots & \ddots & \vdots \\
        0 & \cdots & d_n^{-1}
    \end{pmatrix}
$
\end{prp}


\subsection{Rang d'une matrice}

\begin{dfn}
Soit $A$ une matrice.

On appelle \voc{rang} de $A$ la dimension de $\Ima{A}$.
On note le rang de $A$ $\rang(A)$.
\end{dfn}

\begin{prp}
Soit $A \in \M_n(\K)$.

\[
    A \text{ est inversible } \iff \rang(A) = n
\]
\end{prp}

\begin{prp}
Soit $B$ une base d'un $\K$-espace vectoriel, $E$, de dimension finie
différente de $0$ et soit $x_1, \ldots, x_q$ des vecteurs de $E$.

Le rang de la famille $(x_1, \ldots, x_q)$ est le rang de la matrice
dont pour tout entier $i$ compris entre $1$ et $q$ la i-eme colonne
est constituée de coordonée de $x_i$ dans $B$.
\end{prp}

\begin{prp}
Soient $E$ et $F$ des $\K$-espaces vectoriels de dimension finie différente
de $0$, $B$ une base de $E$, $B'$ une base de $F$ et $f \in \Lin(E, F)$

\[
    \rang(f) = \rang(\mat_{B, B'}(f))
\]
\end{prp}



\begin{prp}
Soit $A \in \M_{p, q}(\K)$

\begin{gather*}
    \forall Q \in \mathcal{GL}_p(\K), \;\; \rang(QA) = \rang(A) \\
    \forall P \in \mathcal{GL}_q(\K), \;\; \rang(AP) = \rang(A)
\end{gather*}
\end{prp}

\begin{prp}
Soit $A \in \M_{p, q}(\K)$, $r \in \lib 0, \min(p, q) \rib$

\[
    \rang(A) = r \iff A \text{ est équivalente à } J_r
\]
\end{prp}

\begin{prp}
Soit $A \in \M_{p, q}(\K)$

\[
    \rang(A^\intercal) = \rang(A)
\]
\end{prp}


\subsection{Rang de matrices extraites}

\begin{dfn}
Soit $A \in \M_{p, q}(\K)$

On appelle \voc{matrice extraite de $A$} toute matrice obtenue en
supprimant de $0$ à $p$ lignes et $0$ à $q$ colonnes dans $A$
\end{dfn}

\begin{prp}
Soit $A$ une matrice.

\[
    A' \text{ est une matrice extraite de } A \implies \rang(A') \leq
    \rang(A)
\]
\end{prp}

\begin{prp}
Soit $A$ une matrice.

Le rang de $A$ est égal au plus grand ordre des matrices
extraites carrées inversibles de $A$
\end{prp}



\section{Opérations élémentaires et systèmes linéaires}


\subsection{Opération élémentaire sur un matrice}

\begin{dfn}
Soit $A$ un matrice.

On appelle \voc{Opération élémentaire de $A$} l'une des transformatioins
suivantes
\begin{itemize}
    \item Remplacement d'une ligne par un multiple non nul de cette même
    ligne. On note alors $\mathrm{L}_i \gets \alpha \mathrm{L}_i$.
    \item Ajout d'un multiple d'une ligne à une autre ligne.
    On note $\mathrm{L}_i \gets \mathrm{L}_i + \alpha \mathrm{L}_k$.
    \item \'Echange de deux lignes. On note $\mathrm{L}_i \leftrightarrow
    \mathrm{L}_j$
\end{itemize}
Les même opérations sur les colonnes sont considérées comme des
opérations élémentaires.
\end{dfn}

\begin{prp}
Les opérations élémentaires sur un matrice ne changent pas le rang de la
matrice.
\end{prp}

\end{document}
