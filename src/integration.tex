\documentclass{article}

\usepackage[utf8]{inputenc}
\usepackage[T1]{fontenc}
\usepackage[frenchb]{babel}
\usepackage{amssymb}
\usepackage{amsmath}
\usepackage{amsthm}
\usepackage{color}
\usepackage{centernot}
\usepackage{xspace}
\usepackage{stmaryrd}
\usepackage[pdfborder={0 0 0}]{hyperref}
\usepackage{mathtools}

\everymath{\displaystyle}

% Ensembles
\newcommand{\R}{\mathbb{R}}
\newcommand{\C}{\mathbb{C}}
\newcommand{\N}{\mathbb{N}}
\newcommand{\Z}{\mathbb{Z}}
\newcommand{\Q}{\mathbb{Q}}
\newcommand{\Pri}{\mathbb{P}}
\newcommand{\De}{\mathbb{D}}
\newcommand{\K}{\mathbb{K}}
\newcommand{\co}[1]{\mathrm{C^{#1}}}
\newcommand{\sym}{\mathcal{S}}
\newcommand{\groupeLineaire}{\mathcal{GL}}

% Abreviation de declaration
\newcommand{\resu}{(u_n)_{n \in \N} \in \R^\N}
\newcommand{\allent}{\forall n \in \N}
\newcommand{\apcr}{\exists n_0 \in \N \tq \forall n \geq n_0, \;\;}

% Operaeur logique
\newcommand{\et}{\;\;\text{ et }\;\;}
\newcommand{\ou}{\;\;\text{ ou }\;\;}
\newcommand{\tq}{\;\;\text{ tel que }\;\;}
\newcommand{\notimplies}{\centernot\implies}

% Abreviation d'operateur numérique
\newcommand{\lm}{\lim\limits}
\newcommand{\lminf}{\lm_{n \to +\infty}}
\newcommand{\ngl}[2]{\negl_{#1}\!\!\!(#2)}

% Abreviation textuelle
\newcommand{\ssi}{si et seulement si\xspace}
\newcommand{\voi}[1]{\text{ au voisinage de }#1}
\newcommand{\en}{\text{ en }}
\newcommand{\est}{\text{ est }}
\newcommand{\drt}{\text{ à droite de }}
\newcommand{\gch}{\text{ à gauche de }}
\newcommand{\strmo}{\text{ strictement monotone }}
\newcommand{\mo}{\text{ monotone }}
\newcommand{\cro}{\text{ croissante }}
\newcommand{\dec}{\text{ décroissante }}
\newcommand{\sur}{\text{ sur }}
\newcommand{\drv}{\text{ dérivable }}
\newcommand{\exs}{\text{ existe }}
\newcommand{\fn}{\text{ finie }}
\newcommand{\lips}{\text{ lipschitzienne }}
\newcommand{\lci}{loi de composition interne\xspace}
\newcommand{\cv}{converge\xspace}
\newcommand{\dv}{diverge\xspace}

% Abreviation de commande
\newcommand{\bs}[1]{\boldsymbol{#1}}
\newcommand{\equ}[1][]{\underset{#1}{\sim}}

% Constante Mathematique
\newcommand{\ex}{\mathrm{e}}

\newcommand{\e}{\!\!}
\newcommand{\mind}[4]{
    \substack {#1 \in \lib 1, #3 \rib \\ #2 \in \lib 1, #4 \rib}
}
\newcommand{\lib}{\llbracket}
\newcommand{\rib}{\rrbracket}

\newcommand{\colonne}{\mathrm{C}}
\newcommand{\ligne}{\mathrm{L}}

\newcommand{\voc}[1]{\textit{#1}}
\newcommand{\hyp}[1]{\textbf{#1}}

\DeclareMathOperator{\ch}{ch}
\DeclareMathOperator{\tah}{th}
\DeclareMathOperator{\sh}{sh}
\DeclareMathOperator{\dl}{DL}
\DeclareMathOperator*{\negl}{o}
\DeclareMathOperator*{\dom}{O}
\DeclareMathOperator{\diez}{\#}
\DeclareMathOperator{\point}{.}
\DeclareMathOperator{\D}{\mathfrak{D}}
\DeclareMathOperator{\divise}{|}
\DeclareMathOperator{\pgcd}{\wedge}
\DeclareMathOperator{\ppcm}{\vee}
\DeclareMathOperator{\union}{\cup}
\DeclareMathOperator{\inter}{\cap}
\DeclareMathOperator{\Vect}{Vect}
\DeclareMathOperator{\Lin}{\mathcal{L}}
\DeclareMathOperator{\Ker}{Ker}
\DeclareMathOperator{\Ima}{Im}
\DeclareMathOperator{\Id}{Id}
\DeclareMathOperator{\dime}{dim}
\DeclareMathOperator{\rang}{rg}
\DeclareMathOperator{\mat}{mat}
\DeclareMathOperator{\Tri}{\mathcal{T}}
\DeclareMathOperator{\Dia}{\mathcal{D}}
\DeclareMathOperator{\Card}{\mathrm{Card}}
\DeclareMathOperator{\M}{\mathcal{M}}
\DeclareMathOperator{\com}{\mathrm{com}}

\DeclarePairedDelimiter\abs{\lvert}{\rvert}

\newenvironment{att}
{\bgroup \color{red}{\Large\textbf{Attention}}\\}
{\egroup}

\theoremstyle{definition}
\newtheorem*{prop}{Propriétée}
\newtheorem*{defin}{Définition}
\theoremstyle{remark}
\newtheorem*{rema}{Remarque}
\newtheorem*{meth}{Méthode}
\theoremstyle{plain}
\newtheorem*{them}{Théorème}
\newtheorem*{coro}{Corollaire}
\newtheorem*{lemm}{Lemme}

\newenvironment{prp}[1][]
{\begin{prop}[#1]\quad\\}
{\end{prop}}
\newenvironment{dfn}[1][]
{\begin{defin}[#1]\quad\\}
{\end{defin}}
\newenvironment{rem}[1][]
{\begin{rema}[#1]\quad\\}
{\end{rema}}
\newenvironment{thm}[1][]
{\begin{them}[#1]\quad\\}
{\end{them}}
\newenvironment{cor}[1][]
{\begin{coro}[#1]\quad\\}
{\end{coro}}
\newenvironment{met}[1][]
{\begin{meth}[#1]\quad\\}
{\end{meth}}
\newenvironment{lem}[1][]
{\begin{lemm}[#1]\quad\\}
{\end{lemm}}


\title{Intégration}

\begin{document}

\maketitle
\pagebreak
\tableofcontents
\pagebreak

Dans tous le chapitre, On suppose $(a, b) \in \R^2$.

\section{Fonctions continues par morceaux}

Dans ce paragraphe, on suppose $a < b$

\subsection{Fonctions en escalier}

\begin{dfn}
On appelle \voc{subdivision du segment $[a, b]$} toute partie finie
de $[a, b]$ contenant $a$ et $b$
\end{dfn}

\begin{dfn}
On appelle \voc{pas} ou \voc{module} de la subdivision
$\sigma : x_0 < x_1 < \ldots < x_n$ et on note $|\sigma|$, le réel
\[
    \max_{i \in \lib 0, n - 1 \rib} (x_{i + 1} - x_i)
\]
\end{dfn}

\begin{dfn}
Un fonction $f : [a, b] \to \R$ est dite \voc{en escalier sur
$[a, b]$} lorsqu'il existe une subdivision $\sigma : x_0 < \ldots < x_n$
de $[a, b]$ tel que $\forall i \in \lib 0, n- 1 \rib$, $f$ soit
constante sur $]x_i, x_{i + 1}[$.

Dans ce cas $\sigma$ est dite \voc{adaptée} ou \voc{subordonée} à la
fonction $f$
\end{dfn}

\begin{dfn}
On note $\mathcal{E}([a, b], \R)$ l'ensemble des fonctions en escalier
sur $[a, b]$
\end{dfn}

\begin{dfn}
$\mathcal{E}([a, b], \R)$ est un $\R$-espace vectoriel
\end{dfn}

\subsection{Fonctions continues par morceaux}

\begin{dfn}
Un fonction $f : [a, b] \to \R$ est dite \voc{continue par morceaux sur
$[a, b]$} lorsqu'il existe une subdivision $\sigma : x_0 < \ldots < x_n$
de $[a, b]$ tel que $\forall i \in \lib 0, n - 1 \rib$, la restriction
de $f$ à $]x_i, x_{i + 1}[$ soit $\co{0}$ sur $]x_i, x_{i + 1}[$ et
admette des limites finies en $x_i^+$ et $x_{i + 1}^-$.

Une telle subdivision est dite \voc{adaptée à $f$}
\end{dfn}

\begin{dfn}
Soit $I$ un intervalle de $\R$ contenant au moint deux points, $f$ une
fonction de $I$ dans $\R$.

$f$ est dite \voc{continue par morceaux sur $I$}
lorsque $\forall (a, b) \in I^2 \tq a < b, f$ est continue par
morceaux sur $[a, b]$
\end{dfn}

\begin{prp}
Soit $I$ un intervalle de $\R$ contenant au moin deux points.

L'ensemble des fonctions continues par morceaux sur $I$ est un
$\R$-espace vectoriel
\end{prp}

\begin{prp}
Soit $f \in \mathcal{C}([a, b], \R)$.

\[
    \forall \epsilon > 0, \;\;
    \exists \Gamma \in \mathcal{E}([a, b], \R) \tq
    |f - \Gamma| \leq \epsilon
\]
\end{prp}

\begin{thm}
Soit $f$ une fonction continue par morceaux sur $[a, b]$.

\[
    \forall \epsilon > 0, \;\; \exists (\varphi, \psi) \in
    (\mathcal{E}([a, b], \R))^2 \tq \varphi \leq f \leq \psi \et
    \psi - \varphi \leq \epsilon
\]
\end{thm}

\section{Intégrale d'une fonction continue par morceaux sur un segment}

\subsection{Intégrale d'une fonction en escalier}

\begin{dfn}
Soit $f \in \mathcal{E}([a, b], \R)$ et $\sigma : x_0 < \ldots < x_n$ un
subdivision adaptée à $f$. Pour tout $i \in \lib 1, n \rib$ notons
$c_i$ la valeur de $f$ sur $x_{i - 1}, x_i$.

La quantité $\sum_{i = 1}^n (x_i - x_{i - 1}) c_i$ ne dépend pas de la
subdivision de $[a, b]$ adapté à $f$.

On appelle cette quantité intégrale définis sur $[a, b]$ et on la note
$\int_{[a, b]} f$
\end{dfn}

\begin{prp}
L'intégrale est une forme linéaire de $\mathcal{E}([a, b], \R)$
\end{prp}

\begin{prp}
Soit $f \in \mathcal{E}([a, b], \R)$ tel que $f \geq 0$ sur $[a, b]$.

\[
    \int_{[a, b]} = \sum^n_{i = 1} (x_i - x_{i - 1}) c_i
\]
\end{prp}

\begin{prp}
Si $(f, g) \in (\mathcal{E}([a, b], \R))^2$ et si $f \leq g$ sur $[a, b]$.
\[
    \int_{[a, b]} f \leq \int_{[a, b]} g
\]
\end{prp}

\begin{prp}
Soit $c \in ]a, b[$ et $f : [a, b] \to \R$.

$f$ est en escalier \ssi $f_{|_{[a, c]}}$ et $f_{|_{[c, b]}}$ sont en escalier.
Dans ce cas

\[
    \int_{[a, b]} f = \int_{[a, c]} f_{|_{[a, c]}}
        + \int_{[c, b]} f_{|_{[c, b]}}
\]
\end{prp}

\subsection{Intégrale d'une fonction continue par morceaux}

\subsubsection{Définition}

\begin{prp}
Une fonction continue par morceaux sur un segment est bornée sur ce
segment.
\end{prp}

\begin{prp}
Soit $f$ une fonction continue par morceaux sur $[a, b]$
\begin{gather*}
    A = \left\{ \int_{[a, b]} \varphi \quad \middle| \quad \varphi \in
        \mathcal{E}([a, b], \R) \et \varphi \leq f \right\} \\
    B = \left\{ \int_{[a, b]} \psi \quad \middle| \quad \psi \in
        \mathcal{E}([a, b], \R) \et \varphi \geq f \right\}
\end{gather*}
$A$ admet une borne supérieure et $B$ admet une borne inférieure. c'est
borne sont égales.
\end{prp}

\begin{dfn}
Soit $f$ une fonction continue par morceaux sur $[a, b]$.

On appelle intégrale de $f$ sur $[a, b]$ le nombre:
\begin{align*}
    \int_{[a, b]} =
        &\sup \left\{ \int_{[a, b]} \varphi \quad \middle| \quad \varphi \in
        \mathcal{E}([a, b], \R) \et \varphi \leq f \right\} \\
        &\inf \left\{ \int_{[a, b]} \psi \quad \middle| \quad \psi \in
        \mathcal{E}([a, b], \R) \et \varphi \geq f \right\}
\end{align*}
\end{dfn}

\subsubsection{Propriétés}

\begin{prp}
L'intégrale est une forme linéaire de l'éspace des fonctions continues
par morceaux sur $[a, b]$
\end{prp}

\begin{prp}
Soit $f$ une fonction continue par morceaux sur $[a, b]$ et $g$ une
fonction continue par morceaux sur $[a, b]$ tel que $f \leq g$

\[
    \int_{[a, b]} f \leq \int_{[a, b]} g
\]
\end{prp}

\begin{prp}
Soit $f$ continue par morceaux sur $[a, b]$.

$|f|$ est aussi continue par morceaux et
\[
    |\int_{[a, b]} f| \leq \int_{[a, b]} |f|
\]
\end{prp}

\begin{prp}
Soit $c \in ]a, b[$ et $f : [a, b] \to \R$.

$f$ est continue par morceaux \ssi $f_{|_{[a, c]}}$ et $f_{|_{[c, b]}}$
sont en continues par morceaux.
Dans ce cas
\[
    \int_{[a, b]} f = \int_{[a, c]} f_{|_{[a, c]}}
        + \int_{[c, b]} f_{|_{[c, b]}}
\]
\end{prp}

\subsection{Extenion de la définition}

à partir de ce paragraphe, on ne suppose plus $a < b$

\begin{dfn}

Soit $I$ un intervalle de $\R$ contenant au moin deux points, $f$ une fonction
continue par marceaux sur $I$.
Pour tout $(a, b) \in I^2$, on pose

\begin{gather*}
    \text{si } a < b, \int_a^b f(t)\mathrm{d}t = \int_{[a, b]} f \\
    \text{si } a = b, \int_a^b f(t)\mathrm{d}t = 0 \\
    \text{si } a > b, \int_a^b f(t)\mathrm{d}t = - \int_{[a, b]} f
\end{gather*}
\end{dfn}

\section{Sommes de Riemann}

\begin{dfn}
Soit $f$ continue par morceaux sur $[a, b]$ où $a < b$,
$\sigma_n ; x_0 < \ldots < x_n$ la subivision de $[a, b]$
à pas constant.

On appelle \voc{somme de Riemann de $f$ associée à la subdivision $\sigma_n$}
la quantité:
\[
    S_n(f) = \frac{b - a}{n} \sum_{k = 0}^{n - 1} f(a + k\frac{b - a}{n})
\]
\end{dfn}

\begin{prp}
Soit $(a, b) \in \R$ tel que $a < b$, Soit $f$ continue par morceaux sur
$[a, b]$.

\[
    \lminf \frac{b - a}{n} \sum_{k = 0}^{n - 1} f(a + k \frac{b - a}{n}) =
    \int_a^b f(t)\mathrm{d}t
\]
\end{prp}

\section{Intégration d'une fonction continue sur un segment}

\begin{thm}[Fondamentale de l'intégration]
Soit $I$ une intervalle de $\R$ contenant au moin deux points, $a \in I$,
$f$ une fonction continue de $I$ dans $\R$

La fonction
\[
    F : x \mapsto F(x) = \int_a^x f(t)\mathrm{d}t
\]
est l'unique primitive de $f$ sur $I$ s'annulent en $a$
\end{thm}

\begin{prp}
Soit $I$ une intervalle de $\R$ contenant au moin deux points,
$f \in \co{0}(I, \R)$, $(a, b) \in I^2$ et $G$ est une primitive
de $f$ sur $I$.

\[
    \int^b_a f(t)\mathrm{d}t = \left[G(t)\right]^b_a = G(b) - G(a)
\]
\end{prp}

\section{Formule de Taylor}

\subsection{Formule de Taylor avec reste intégrale}

\begin{prp}
Soit $I$ un intervalle de $\R$ contenant au moin deux points, $a \in I$,
$n \in \N$ et $f$ une fonction de classe $\co{n + 1}$ de $I$ dans
$\R$.
\[
    f(x) = \sum^n_{k = 0} \frac{f^{(k)}(a)}{k!} (x - a)^k
    + \int^x_a \frac{(x - t)^n}{n!} f^{(n + 1)}(t) \mathrm{d}t
\]
\end{prp}

\subsection{Inégalité de Taylor-Lagrange}

\begin{prp}
Soient $(a, b) \in \R^2$ tel que $a \neq b$, $n \in \N$, $f$ une fonction
de $[a, b] \to \R$ de classe $\co{n + 1}$ sur $[a, b]$. Soit $M_{n + 1}$
un majorant de $|f^{(n + 1)}|$ sur $[a, b]$.

\[
    |f(b) - \sum^n_{k = 0} \frac{f^{(k)(a)}}{k!}(b - a)^k| \leq
    \frac{|b - a|^{n + 1}}{(n + 1)!} M_{n + 1}
\]
\end{prp}

\subsection{Formule de Taylor-Young}

\begin{prp}
Soit $I$ un intervalle de $\R$ contenant au moin deux points, $a \in I$,
$n \in \N$ et $f : I \to \R$ une fonction $\co{n}$ sur $I$.

\[
    f(x) = \sum_{k = 0}^n \frac{f^{(k)}(a)}{k!} (x - a)^k
    + \ngl{x \to a}{(x - a)^n}
\]
\end{prp}

\end{document}
