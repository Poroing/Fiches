\documentclass{article}

\usepackage[utf8]{inputenc}
\usepackage[francais]{babel}
\usepackage{amssymb}
\usepackage{amsmath}
\usepackage{amsthm}
\usepackage{color}

\everymath{\displaystyle}

\newcommand{\ssi}{si et seulement si}
\newcommand{\R}{\mathbb{R}}
\newcommand{\N}{\mathbb{N}}
\newcommand{\Z}{\mathbb{Z}}
\newcommand{\resu}{(u_n)_{n \in \N} \in \R^\N}
\newcommand{\allent}{\forall n \in \N}
\newcommand{\et}{\text{ et }}
\newcommand{\ou}{\text{ ou }}
\newcommand{\tq}{\text{ tel que }}
\newcommand{\lm}{\lim\limits}
\newcommand{\voi}[1]{\text{ au voisinage de }$#1$}
\newcommand{\bs}[1]{\boldsymbol{#1}}

\newenvironment{att}
{\bgroup \color{red}{\Large\textbf{Attention}}\\}
{\egroup}

\theoremstyle{definition}
\newtheorem*{prop}{Proposition}
\newtheorem*{defin}{Définition}
\theoremstyle{remark}
\newtheorem*{rema}{Remarque}
\theoremstyle{plain}
\newtheorem*{them}{Théorème}

\newenvironment{prp}[1][]
{\begin{prop}[#1]\quad\\}
{\end{prop}}
\newenvironment{dfn}[1][]
{\begin{defin}[#1]\quad\\}
{\end{defin}}
\newenvironment{rem}[1][]
{\begin{rema}[#1]\quad\\}
{\end{rema}}
\newenvironment{thm}[1][]
{\begin{them}[#1]\quad\\}
{\end{them}}

\title{Title}

\begin{document}
\maketitle
\pagebreak

\end{document}


\title{Structures Algébriques Usuelles}

\begin{document}

\section{Lois de Compositons Internes}

\begin{dfn}
Les lois de compositions internes sont les applications de formes.
\begin{align*}
    * : \;\;    E \times E &\to E \\
                (x, y) &\mapsto x * y
\end{align*}
\end{dfn}

\begin{dfn}
La loi de composition interne $*$ dans $E$
\begin{description}
    \item[est commutative] \quad \\
    \[\forall (x, y) \in E^2, \;\; x * y = y * x\]
    \item[est associative] \quad \\
    \[\forall (x, y, z) \in E^3, \;\;\]
    \[(x * y) * z = x * (y * z)\]
    \item[admet un élément neutre $\bs{e}$] \quad \\
    \[\forall x \in E,\;\; x * e \;=\; e * x \;=\; x\]
    \item[assigne un symetrique $\bs{x'}$ à $\bs{x \in E}$] \quad \\
    $E$ admet un élément neutre $e$ pour $x$
    et $x * x' \;=\: x' * x \;=\; e$
\end{description}
\end{dfn}

\begin{prp}
\begin{itemize}
    \item Si $x$ admet un symétrique il est unique.
    \item Si $x$ et $y$ admettent pour symétrique par $*$,
    $x'$ et $y'$, on pose alors $(x * y)'$ le symétrique de $x * y$.
    \[
        (x * y)' = y' * x'
    \]
\end{itemize}
\end{prp}

\begin{prp}
Si $E$ admet un élément neutre pour $*$, il est unique.
\end{prp}

\section{Groupe}

\begin{dfn}[Groupe]
$(G, *)$ est un groupe \ssi
\begin{itemize}
    \item $*$ est une loi de compositon interne de $G$
    \item $*$ est associative
    \item $G$ admet un élément neutre pour $*$
    \item Tout élément de $G$ admet un symétrique pour $*$ dans $G$
\end{itemize}
Si $*$ est commutative, le groupe est abélien
\end{dfn}

\pagebreak

\begin{dfn}
Si $G$ est un groupe muni d'une \lci notée
\begin{description}
    \item[$\bs{+}$] \quad \\
    \begin{itemize}
        \item L'élément neutre est noté $O_G$
        \item Le symétrique de $x \in G$ est noté $-x$
        \item $\underbrace{x + x + \dotsb + x}_{n \text{termes}}$
        est notée $nx$ $(n \in \N^*)$
    \end{itemize}
    \item[$\bs{\times}$, $\bs{*}$ ou $\bs{.}$] \quad \\
    \begin{itemize}
        \item L'élément neutre est noté $e_G$ ou $1_G$
        \item Le symétrique de $x \in G$ est noté $x^{-1}$
        \item $\underbrace{x * x * \dotsb * x}_{n \text{termes}}$
        est noté $x^n$ $(n \in \N^*)$
    \end{itemize}
\end{description}
\end{dfn}

\begin{dfn}[Sous-groupe]
$H$ est un sous-groupe de $(G, *)$ \ssi
\begin{itemize}
    \item $H \in G$
    \item $H \neq \emptyset$
    \item $\forall (x, y) \in H^2, \;\;
    \left\{(x * y) \in H \et x^{-1} \in H\right\}
    \ou \left\{(x * y^{-1}) \in H\right\}$
\end{itemize}
\end{dfn}

\begin{rem}
Si $H$ est un sous-groupe de $(G, *)$ alors $e_H = e_G$
\end{rem}

\end{document}
