\documentclass{article}

\usepackage[utf8]{inputenc}
\usepackage[francais]{babel}
\usepackage{amssymb}
\usepackage{amsmath}
\usepackage{amsthm}
\usepackage{color}

\everymath{\displaystyle}

\newcommand{\ssi}{si et seulement si}
\newcommand{\R}{\mathbb{R}}
\newcommand{\N}{\mathbb{N}}
\newcommand{\Z}{\mathbb{Z}}
\newcommand{\resu}{(u_n)_{n \in \N} \in \R^\N}
\newcommand{\allent}{\forall n \in \N}
\newcommand{\et}{\text{ et }}
\newcommand{\ou}{\text{ ou }}
\newcommand{\tq}{\text{ tel que }}
\newcommand{\lm}{\lim\limits}
\newcommand{\voi}[1]{\text{ au voisinage de }$#1$}
\newcommand{\bs}[1]{\boldsymbol{#1}}

\newenvironment{att}
{\bgroup \color{red}{\Large\textbf{Attention}}\\}
{\egroup}

\theoremstyle{definition}
\newtheorem*{prop}{Proposition}
\newtheorem*{defin}{Définition}
\theoremstyle{remark}
\newtheorem*{rema}{Remarque}
\theoremstyle{plain}
\newtheorem*{them}{Théorème}

\newenvironment{prp}[1][]
{\begin{prop}[#1]\quad\\}
{\end{prop}}
\newenvironment{dfn}[1][]
{\begin{defin}[#1]\quad\\}
{\end{defin}}
\newenvironment{rem}[1][]
{\begin{rema}[#1]\quad\\}
{\end{rema}}
\newenvironment{thm}[1][]
{\begin{them}[#1]\quad\\}
{\end{them}}

\title{Title}

\begin{document}

\maketitle
\pagebreak

\section{Limite d'une fonction en un point}
\begin{dfn} 
$f$ admet une limite $l$ en $a$ lorsque
\begin{description}
    \item[$\bs{a \in \R, l \in \R}$] \hfill \\
    $\forall \epsilon > 0, \exists \eta > 0 \tq \forall x \in I, |x - a|
    \leq \implies |f(x) - l| \leq \epsilon$
    \item[$\bs{a \in \R, l = \infty}$] \hfill \\
    $\forall M \in \R, \exists \eta > 0 \tq \forall x \in I,
    |x - a| \leq \eta \implies |f(x) - a| \leq \epsilon$
    \item[$\bs{a = \infty, l \in \R}$] \hfill \\
    $\forall \epsilon > 0, \exists A \in \R \tq \forall x \leq A,
    |f(x) - a| \leq \epsilon$
\end{description}
\end{dfn}

\begin{prp}[Unicité de la limite] 
\begin{enumerate}
    \item La limite est unique
    \item Si $a \in I$ et si $f$ admet une limite en $a$, alors
    $\lm_{a} f = f(a)$
\end{enumerate}
\end{prp}

\begin{prp} 
Soit $a \in \bar{\R}$. Si $f$ admet une limite finie en $a$, alors
$f$ est bornée au voisinage de $a$
\end{prp}

\begin{dfn} 
\begin{enumerate}
    \item Si $f$ est définie au voisinage à gauche de $a$, on dit que
    $f$ admet $l$ pour limite à gauche lorsque $f_{|I\cap ]-\infty, a[}$
    admet $l$ pour limite en $a$; on note alors
    $l = \lm_{x \to a^-} f(x)$ ou $l = \lm_{a^-} f$
    \item Si $f$ est définie au voisinage à droite de $a$, on dit que
    $f$ admet $l$ pour limite à droite lorsque $f_{|I\cap ]a, +\infty[}$
    admet $l$ pour limite en $a$; on note alors
    $l = \lm_{x \to a^+} f(x)$ ou $l = \lm_{a^+} f$
\end{enumerate}
\end{dfn}

\begin{prp} 
On suppose que $f$ est définie à froite et à gauche au voisinage de
$a \in \R$. Soit $l \in \bar{\R}$.
\begin{enumerate}
    \item $a \in I \et \lm_{x \to a} f(x) = l \iff
    (\lm_{x \to a^-} f(x) = \lm_{x \to a^+} f(x) = l \et l = f(a))$
    \item $a \in I \et \lm_{x \to a} f(x) = l \iff
    (\lm_{x \to a^-} f(x) = \lm_{x \to a^+} f(x) = l)$
\end{enumerate}
\end{prp}

\section{Propriétés sur les limites}
\begin{thm}[Caractérisation séquentielle]
Soit $(a, l) \in \bar{\R}^2$ avec $a$ un point de $I$ ou une extrémité
de $I$. Les deux assertions sont équivalentes
\begin{itemize}
    \item $\lm_{x \to a} f(x) = l$ 
    \item $\forall (u_n) \in I^\N \tq \lm_{n \to +\infty} u_n = a,
    \lm_{n \to +\infty} f(u_n) = l.$
\end{itemize}
\end{thm}

\begin{rem}
Le théorème est souvent utilisé pour prouver qu'une fonction n'admet pas
de limite.
Il suffit de trouver une suite $(u_n)$ qui tend vers $+\infty$ et tel
que $f(u_n)$ diverge ou trouver deux suite $(u_n)$ et $(v_n)$
qui tend vers $+\infty$ telle que $f(u_n)$ et $f(v_n)$ tendent vers des
limites différente.
\end{rem}
 
\begin{rem}[Opération sur les limites]
Les opération sur les limites sont les même qu'avec les suites.
\end{rem}

\begin{prp}[Composition de limite] 
Soit $(a, b, l) \in \bar{\R}^3$
$\lm_{x \to a} f(x) = b \et \lm_{t \to b} g(t) = l
\implies \lm_{x \to a} g \circ f(x) = l$
\end{prp}

\begin{prp} 
Soit $a \in \bar{\R}$, alors
$\lm_{a} f \in ]0, +\infty] \implies f \text{minorée} \voi{a}
\text{ par un réel strictement positif}$
\end{prp}

\begin{prp} 
\begin{itemize}
\item $f(x) \geq 0 \voi{a} \implies \lm_a f \geq 0$
\item $f(x) \leq g(x) \voi{a} \implies
\lm_a f \leq \lm_a g$
\end{itemize}
\end{prp}

\begin{rem} 
\begin{itemize}
    \item 
    \begin{att}
    On ne peut pas passer à la limite dans un inégalité que si l'on
    sait que les limites existent!!
    \end{att}
    \item
    \begin{att}
    $f(x) < g(x) \voi{a} \implies \lm_a f(x) \leq \lm_a g(x)$
    et non $\lm_a f(x) < \lm_a g(x)$
    \end{att}
\end{itemize}
\end{rem}

\end{document}
