\documentclass{article}

\usepackage[utf8]{inputenc}
\usepackage[francais]{babel}
\usepackage{amssymb}
\usepackage{amsmath}
\usepackage{amsthm}
\usepackage{color}

\everymath{\displaystyle}

\newcommand{\ssi}{si et seulement si}
\newcommand{\R}{\mathbb{R}}
\newcommand{\N}{\mathbb{N}}
\newcommand{\Z}{\mathbb{Z}}
\newcommand{\resu}{(u_n)_{n \in \N}}
\newcommand{\allent}{\forall n \in \N}
\newcommand{\et}{\text{ et }}
\newcommand{\ou}{\text{ ou }}
\newcommand{\tq}{\text{ telle que }}

\newenvironment{att}
{\bgroup \color{red}{\Large\textbf{Attention}}\\}
{\egroup}

\theoremstyle{definition}
\newtheorem*{dfn}{Définition}
\theoremstyle{remark}
\newtheorem*{rem}{Remarque}
\theoremstyle{plain}
\newtheorem*{thm}{Théorème}

\title{Suites de nombre réels et complexes}

\begin{document}
\maketitle
\pagebreak

\section{Généralité sur les suites réelles}
\subsection{Suite Réelle}
\begin{dfn}
On appelle suite de nombre réelles toute fonction de $\N$ dans
$\R$. On note une tel suite $\resu$
\begin{align*}
    \N &\to \R\\
    n &\mapsto u_n
\end{align*}
On note $\R^{\N}$ l'ensemble des suites réelles
\end{dfn}

\begin{rem} 
\quad
\begin{enumerate}
    \item 2 suites $(u_n)$ et $(v_n)$ réelles sont égales \ssi 
    \space $\forall n \in \N, u_n = v_n$
    \item Si $n_0 \in \Z$, si on considère $(u_n)_{n \geq n_0}$
    quitte à poser $v_n = u_{n - n_0}$ on peut se ramener à 1
    suite indéxée par $\N$, mais les propriétés restent valables
    pour les suites $(u_n)_{n \geq n_0}$
\end{enumerate}
\end{rem}

\subsection{Suite Stationnaire}
\begin{dfn}
Une suite $\resu$ est dite stationnaire lorsque $(u_n)$ est
constante à partir d'un rang i-e lorsque
\[
    \exists n_0 \in \N / \forall n \geq n_0, u_n = u_{n_0}
\]
\end{dfn}

\subsection{Variation d'une suite}
\begin{dfn}
Une suite réelle $\resu$ est dite
\begin{description}
    \item[croissante] lorsque,
    \[
        \allent, u_{n + 1} \geq u_n
    \]
    \item[décroissante] lorsque,
    \[
        \allent, u_{n + 1} \leq u_n
    \]
    \item[monotone] lorque $(u_n)$ est croissante ou décroissante
\end{description}
\end{dfn}

\begin{rem}
Pour étudier la monotonie d'une suite, on peut 
\begin{itemize}
    \item chercher le signe de $u_{n + 1} - u_n$
    \item si $\allent, u_n > 0 \vee \allent, u_n < 0$ alors on peut
    comparer $\frac{u_{n + 1}}{u_n}$ et $1$
    \begin{itemize}
        \item 
        \begin{align*}
                    &\allent, u_n > 0 \et \frac{u_{n + 1}}{u_n} \leq 1 \\
            \implies& u_{n + 1} \leq u_n \\
            \implies& (u_n) \text{est décroissante}
        \end{align*}
        \item 
        \begin{align*}
                    &\allent, u_n > 0 \et \frac{u_{n + 1}}{u_n} \geq 1 \\
            \implies& u_{n + 1} \geq u_n \\
            \implies& (u_n) \text{est croissante}
        \end{align*}
    \end{itemize}
\end{itemize}
\begin{att}
Si $\allent, u_n < 0$ les résultats sont inversés car
$\frac{u_{n + 1}}{u_n} \leq 1 \implies u_{n + 1} \geq u_n$
\end{att}
\end{rem}

\subsection{Différentes remarques}
\begin{rem}
\quad
\begin{enumerate}
    \item Une suite est dite constante lorque
    \[
        \exists k \in \R \tq \allent, u_n = k
    \]
\end{enumerate}
\end{rem}

\end{document}
